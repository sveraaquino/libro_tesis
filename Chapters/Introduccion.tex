\newpage{\ } 
\thispagestyle{empty} 

\chapter{Introducción}
\lhead{Capítulo 1. \emph{Introducción}} % This is for the header on each page - perhaps a shortened title

La captura de carbono es un servicio ambiental proporcionado por los bosques para mitigar las emisiones de di\'oxido de carbono (CO2) a la atm\'osfera \cite{peralta2013analisis}. La captura de carbono es determinante para disminuir el calentamiento global y estabilizar el cambio clim\'atico producidos por el incremento en la atm\'osfera de los llamados Gases de Efecto Invernadero (GEI) \cite{marquezestimacion}. El CO2 es el gas mas abundante, contribuyendo con un 76\% al GEI \cite{avila2001almacenamiento}, debido principalmente al cambio de paisajes de bosques tropicales maduros a paisajes agr\'icolas.\\~\\
Los bosques tropicales en condiciones naturales contienen m\'as carbono a\'ereo por unidad de superficie que cualquier otro tipo de cobertura terrestre \cite{gibbs2007monitoring}. Por esto, cuando los bosques se convierten a otros usos del suelo, ocurre una gran liberaci\'on neta de carbono a la atm\'osfera. El cambio en el uso del suelo es responsable del 15-20\% de las emisiones totales de gases de efecto invernadero \cite{peralta2013analisis}.\\~\\
Las transformaciones química de compuestos, que contienen carbono en los intercambios entre biosfera, atm\'osfera, hidrosfera y litosfera son conocidas como ciclo de carbono \cite{wikixxx}. La fotosíntesis de las plantas constituye un proceso fundamental en el ciclo, ya que permite separar  el CO2 en oxigeno que consumimos y carbono (C) en materia \'organica, actuando en forma de almacenes de C como biomasa en función a la composición flor\'istica, la edad y la densidad de cada estrato por comunidad vegetal por periodos prolongados \cite{acosta2003diseno}.\\~\\
El t\'ermino biomasa se refiere a toda la materia org\'anica que proviene de \'arboles, plantas y desechos de animales que pueden ser convertidos en energ\'ia. La biomasa forestal se define como la cantidad total de materia org\'anica a\'erea presente en los \'arboles incluyendo hojas, ramas, tronco principal y corteza \cite{garzuglia2003wood}.\\~\\
La teledetecci\'on o percepci\'on remota sin estar en un contacto f\'isico directo permite el uso de informaciones provenientes de sensores instalados en plataformas espaciales, que complementados con sistemas de informaci\'on geogr\'aficas (SIG) permiten an\'alisis mas continuos y din\'amico. Estos sensores remotos captan la energ\'ia reflejada o radiada por la superficie, ya sea emitida por el sol (sensores pasivos) o por el mismo sensor (sensores activos), para ser transformadas a valores digitales (VD) como im\'agenes satelitales, de manera secuencial para cada espacio de la tierra, a intervalos regulares de tiempo.\\~\\
Las coberturas vegetales, a trav\'es de la im\'agenes prove\'idas por los sensores remotos, es posible calcular \'indices que var\'ian dentro de m\'argenes, indicando el vigor de la vegetaci\'on o la densidad de la biomasa forestal. Estos indices junto con la comparaci\'on multitemporal hace posible identificar la evoluci\'on de coberturas vegetales en periodos de tiempos obteniendo resultados cualitativos y/o cuantitativos en espacio y tiempo \cite{martinez2013normalizacion}.\\~\\
En la actualidad existen diferentes m\'etodos para la detecci\'on de cambios de vegetaci\'on. Los m\'etodos requieren una supervisi\'on, un trabajo de campo y la utilizaci\'on de complejos sistemas de informaci\'on geogr\'afica. Estos sistemas de información geográfica son com\'unmente software de pago, elevando de esta manera el costo de dichos estudios. \\~\\
La falta de informaci\'on nos lleva a varios cuestionamientos referente a como estamos manejando nuestro medio ambiente y que efectos acarreara esos usos. El empleo de la teledetecci\'on y las im\'agenes satelitales multitemporales permiten realizar un an\'alisis a lo largo del tiempo de los cambios que el ambiente est\'a experimentando, mas aun en zonas como el Chaco Paraguayo, donde la informaci\'on ambiental es escaso por los altos costos y dificultades en el acceso al realizar controles en el terreno.\\~\\
Se propone dise\~{n}ar e implementar una metodolog\'ia autom\'atica que permita estimar la perdida de carbono a trav\'es de la biomasa, empleando procesamiento digital de im\'agenes satelitales, disponibles de forma libre, din\'amicos y no complejos.

\section{Justificación y Motivación}

REDD+ es una iniciativa que tiene como objetivo reducir la p\'erdida de bosques, teniendo como actividades principales \cite{peralta2013analisis}:
\begin{itemize}
	\item Evitar p\'erdidas como emisiones de gases de efecto invernadero (conservaci\'on, no deforestaci\'on, no degradaci\'on).
	\item Mantiener el dep\'osito o stock de carbono (conservaci\'on, gesti\'on sostenible).
	\item Incrementar el dep\'osito por su efecto de retenci\'on o sumidero de carbono (conservaci\'on, restauraci\'on, gesti\'on sostenible).
\end{itemize}

El Paraguay se ha embarcado en el proceso de preparaci\'on para reducir la deforestaci\'on y degradaci\'on forestal, a fin de disminuir las emisiones de CO2, conservar los bosques y su biodiversidad. Este proceso deriva la necesidad de elaborar una estrategia nacional, con pol\'iticas socios ambientales y econ\'omicos viables.\\~\\
Para medir los beneficios de carbono de un proyecto REDD+ es necesario calcular la cantidad de carbono almacenado en el bosque en cuesti\'on y luego predecir la cantidad de carbono que se podr\'ia conservar si se detiene o reduce la deforestaci\'on y la degradaci\'on forestal \cite{nellemann2009carbono}.\\~\\
Las grandes extensiones de las \'areas de estudio, la dificultad de acceder a las mismas, el alto costo del establecimiento de las parcelas de inventario y su limitada utilidad hacen que la mayoría de las investigaciones para estimar y mapear la biomasa en bosques se centren en las t\'ecnicas de Sensores Remotos. \\~\\
La necesidad de crear metodolog\'ias que ayuden al monitoreo de forma din\'amica y barata, lleva al desarrollo de herramientas libres que permitan estimar focos de alerta para la toma de acciones y controles m\'as rigurosos a tiempo.

\section{Antecedentes}\label{sec:antecedente}

El proyecto del Mapa global de carbono fue desarrollado por el Jet Propulsion Laboratory, California Institute of Technology en el a\~{n}o 2011 \cite{saatchi2011benchmark}, abarcando m\'as de 2.5 millones en hect\'areas de bosques, para tres continentes, trazando el stock total de carbono en la biomasa viva (por debajo y por encima).\\~\\
El estudio utilizo una combinaci\'on de datos con 4079 parcelas de inventario in situ e im\'agenes satelitales provenientes de sensores remotos LIDAR (Light Detection and Ranging o Laser Imaging Detection and Ranging). La combinaci\'on genera muestras de las estructura boscosa para estimar el almacenamiento de carbono y as\'i poder ser extrapolados en toda las superficie terrestre a trav\'es de im\'agenes \'opticas y microondas (resoluci\'on espacial de 1km).\\~\\
En la actualidad existen varios proyectos finales de grado, realizados por estudiantes de la Facultad de Ciencias Agrarias - UNA, que desarrollan metodolog\'ias de detecci\'on de cambio forestal y estimaciones de carbono. Algunos proyectos finales de grado son citados a continuaci\'on:
\begin{itemize}
	\item Detecci\'on de cambios de la cobertura vegetal mediante indices de vegetaci\'on (NDVI), dentro y fuera de la Reserva de la biosfera del Chaco en el periodo 1985-2011 \cite{gustavo2012deteccion}.
	\item An\'alisis del cambio de cobertura de la tierra y estimaci\'on de carbono en el \'area para Parque Nacional San Rafael, a\~{n}o 2008/2013 \cite{peralta2013analisis}.
	\item Estimaci\'on de carbono almacenado en el Parque Nacional Defensores del Chaco seg\'un formaci\'on vegetal mediante im\'agenes satelitales, a\~{n}o 2014 \cite{kris2014estimacion}.
\end{itemize}
Los trabajos relacionados a estimaciones de carbono, implementan una metodolog\'ia base hecha en el marco denominado \textit{Desarrollo del estudio de linea de base para el sitio piloto Bosque atl\'antico de Alto Paran\'a. (BAAPA)} \cite{BAAPA2013} realizado por el Paraguay Land Use (ParLu). El ParLu es una iniciativa de World Wildlife Fund (WWF) Paraguay y WWF Alemania que apoya a las iniciativas REDD+ en Paraguay, enfoc\'andose principalmente a nivel local en comunidades del Bosque Atl\'antico y el Pantanal.\\~\\
Los productos generados por BAAPA, consisten en mapas de stock de carbono con sus correspondientes mapas de cobertura y deforestaci\'on 2000\textendash2005 y 2005\textendash2011. Estos productos fueron realizados a partir de muestreos en parcelas in situ y algoritmos de clasificaci\'on supervisadas, proporcionada por software SIG de pago. El estudio tambi\'en fue hecho conjuntamente con la  Carrera de Ingenier\'ia Forestal de la Facultad de Ciencias Agrarias perteneciente a la Universidad Nacional de Asunci\'on. 

	Un estudio realizado por University of Maryland Institute for Advanced Computer Studies denominado Forest Cover Change in Paraguay, nos muestra el cambio de vegetaci\'on estimado en todo el pa\'is utilizando un m\'etodo iterativo de etiquetado de cambio por clusterizaci\'on supervisada \cite{huang2009assessment}. Este trabajo detecta cambios en los a\~{n}os 1990 al 2000, donde aparte de proveer un etiquetado de cambios de vegetaci\'on fue realizada con im\'agenes de acceso libre. Las validaciones fueron hechas con varias im\'agenes satelitales de alta resoluci\'on espacial(entre 4 y 0.5 metros), no libres, arrojando para todas las escenas precisiones globales mayor al 90\% y errores por comisi\'on y omisi\'on menores al 10\%.

\section{Planteamiento del problema}

Paraguay es un pa\'is que basa su econom\'ia en la agricultura y la ganader\'ia extensiva, actividades que han afectado al recurso forestal, dando como resultado extensas \'areas deforestadas y degradadas \cite{BAAPA2013}.\\~\\
	En el informe realizado por la ENPAB \cite{basualdo2003estrategia} se menciona que existe una fuerte presi\'on pol\'itica y social, proveniente de diversos grupos que buscan transformar las tierras del Chaco paraguayo en unidades econ\'omicas de producci\'on, cuyo enfoque gira en torno al crecimiento econ\'omico antes que al desarrollo sostenible. 
	En muchas zonas del chaco paraguayo, el modelo de desarrollo y uso de la tierra ha producido grandes extensiones de tierras altamente degradadas, arenales, desertificaci\'on y salinizaci\'on.\\~\\	
	A pesar que existen leyes de protecci\'on para evitar la deforestaci\'on y valorar los bosques, los mismos necesitan apoyo para su monitoreo y aplicaci\'on efectiva, debido a que los costos en tiempo y recursos son elevados.\\~\\
	Con el objetivo de implementar pol\'iticas de mitigaci\'on del cambio clim\'atico relativas a reducir las emisiones provenientes de la degradaci\'on y la deforestaci\'on (REDD+), los pa\'ises en desarrollo deben contar con estimaciones robustas s\'olidas en cuanto a las reservas de carbono forestal \cite{BAAPA2013}.

\section{Objetivos}
Atendiendo a la necesidad de metodolog\'ias alternativas para el monitoreo de perdida de carbono en el campo ambiental, los objetivos delineados son los siguientes.

\subsection{Objetivo General}

\begin{itemize}
\item Desarrollar una metodolog\'ia autom\'atica de an\'alisis de im\'agenes satelitales multitemporales para la generaci\'on de indicadores respecto a la perdida del contenido de carbono en zonas del Chaco Paraguayo.
\end{itemize}
\subsection{Objetivos Específicos}
Para el logro del objetivo general los siguientes objetivos espec\'ificos son propuestos:
\begin{itemize}


\item Realizar detecciones de cambio automatizada dentro del \'area de estudio a trav\'es de la Teledetecci\'on y un SIG.   
\item Desarrollar normalizaciones de im\'agenes para la comparación multi-temporal. 
\item Determinar una constante para la clasificaci\'on de vegetaci\'on en im\'agenes satelitales.
\item Determinar la relación entre la biomasa y el carbono a trav\'es de muestreos.
 
    
%\item Implementaci\'on de la metodolog\'ia como complemento de una herramienta SIG de c\'odigo abierto.


\end{itemize}



\section{Organización de la Tesis}

%La distribución de capítulos del presente trabajo final de grado se encuentra distribuido en 6 capítulos.
%en este capítulo se da una breve introducción al tema, se describe el problema de manera precisa para lograr su mejor entendimiento, también se citan los objetivos trazados tanto específicos como generales finalmente se habla de los antecedentes.
%en el capítulo 2  se realiza una breve introducción sobre la diabetes y los tipos existentes, 
%\begin{comment} luego hablaremos de la enfermedad de los ojos que se da en las personas con diabetes que es conocida como  retinopatía diabética, de la misma se menciona las causas, factores de riesgos y los síntomas. Además se menciona las etapas en cuales la enfermedad se va desarrollando, las anormalidades que se van dando dentro de los ojos, así como también los tratamientos usados para combatir esta enfermedad: fotocoagulación con láser, terapia médica intravítrea y tratamiento quirúrgico. 
%\end{comment} 
%en el capítulo 3  se presenta el marco teórico de las técnicas de  visión por computadora, %\begin{comment} desde los espacios de colores utilizados para representar las imágenes de fondo de ojo hasta los algoritmos se mencionan con detalle el funcionamiento del algoritmo de normalización utilizado,  en los procesos de segmentación, extracción de características y clasificación.
%\end{comment} 
%en el capítulo 4  se detallan  los algoritmos de detección y segmentación utilizados en el sistema de diagnóstico, además  se menciona las sub-segmentaciones utilizadas en estos procesos,
%Luego tenemos la extracción de características cuya importancia radica en el hecho de que reduce la cantidad de datos a procesar. Al final de la metodología tenemos al clasificador de máquina de vector de soporte el cual dará el diagnóstico final. 
%en el capítulo 5  se presenta las métricas  utilizadas para medir el desempeño, luego se evalúa los resultados obtenidos y se realiza  la comparación con respecto al estado del arte y por último en el capítulo 6 se presentan las conclusiones finales tras los experimentos y análisis de resultados del proyecto, por último los trabajos futuros que podrían dar continuidad al trabajo final de grado.


La distribución de capítulos del presente trabajo final de grado se encuentra organizado de la siguiente forma:
%en este capítulo se da una breve introducción al tema, se describe el problema de manera precisa para lograr su mejor entendimiento, también se citan los objetivos trazados tanto específicos como generales finalmente se habla de los antecedentes.
\begin{itemize}

\item En el cap\'itulo 2  se describen los conceptos generales relacionados al cambio clim\'atico y perdida de carbono.

\item En el cap\'itulo 3 se pretende dar un marco te\'orico acerca del procesamiento digital de im\'agenes satelitales.
%\end{comment} 
\item En el cap\'itulo 4  se detalla los algoritmos y procedimientos empleados en la metodolog\'ia de estimaci\'on de perdida de carbono.

\item En el cap\'itulo 5  se presenta las m\'etricas para medir la calidad de los resultados. Tambi\'en se evalu\'a los resultados en base a las m\'etricas previstas.

\item En el cap\'itulo 6 se presentan las conclusiones finales tras los experimentos y an\'alisis de resultados del proyecto, concluyendo con propuestas de trabajos futuros para dar continuidad al trabajo final de grado.

\end{itemize}

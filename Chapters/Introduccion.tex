\newpage{\ } 
\thispagestyle{empty} 

\chapter{Introducción}
\lhead{Capítulo 1. \emph{Introducción}} % This is for the header on each page - perhaps a shortened title

La captura de carbono es un servicio ambiental proporcionado por los bosques para mitigar las emisiones de di\'oxido de carbono (CO2) a la atm\'osfera \cite{peralta2013analisis}. La captura de carbono es determinante para disminuir el calentamiento global y estabilizar el cambio clim\'atico producidos por el incremento en la atm\'osfera de los llamados Gases de Efecto Invernadero (GEI) \cite{marquezestimacion}. El CO2 es el gas mas abundante, contribuyendo con un 76\% al GEI \cite{avila2001almacenamiento}, debido principalmente al cambio de paisajes de bosques tropicales maduros a paisajes agr\'icolas.\\~\\
Los bosques tropicales en condiciones naturales contienen m\'as carbono a\'ereo por unidad de superficie que cualquier otro tipo de cobertura terrestre \cite{gibbs2007monitoring}. Por esto, cuando los bosques se convierten a otros usos del suelo, ocurre una gran liberaci\'on neta de carbono a la atm\'osfera. El cambio en el uso del suelo es responsable del 15-20\% de las emisiones totales de gases de efecto invernadero \cite{peralta2013analisis}.\\~\\
Las transformaciones química de compuestos, que contienen carbono en los intercambios entre biosfera, atm\'osfera, hidrosfera y litosfera son conocidas como ciclo de carbono \cite{wikixxx}. La fotosíntesis de las plantas constituye un proceso fundamental en el ciclo, ya que permite separar  el CO2 en oxigeno que consumimos y carbono (C) en materia \'organica, actuando en forma de almacenes de C como biomasa en función a la composición flor\'istica, la edad y la densidad de cada estrato por comunidad vegetal por periodos prolongados \cite{acosta2003diseno}.\\~\\
El t\'ermino biomasa se refiere a toda la materia org\'anica que proviene de \'arboles, plantas y desechos de animales que pueden ser convertidos en energ\'ia. La biomasa forestal se define como la cantidad total de materia org\'anica a\'erea presente en los \'arboles incluyendo hojas, ramas, tronco principal y corteza \cite{garzuglia2003wood}.\\~\\
La teledetecci\'on o percepci\'on remota sin estar en un contacto f\'isico directo permite el uso de informaciones provenientes de sensores instalados en plataformas espaciales, que complementados con sistemas de informaci\'on geogr\'aficas (SIG) permiten an\'alisis mas continuos y din\'amico. Estos sensores remotos captan la energ\'ia reflejada o radiada por la superficie, ya sea emitida por el sol (sensores pasivos) o por el mismo sensor (sensores activos), para ser transformadas a valores digitales (VD) como im\'agenes satelitales, de manera secuencial para cada espacio de la tierra, a intervalos regulares de tiempo.\\~\\
Las coberturas vegetales, a trav\'es de la im\'agenes prove\'idas por los sensores remotos, es posible calcular \'indices que var\'ian dentro de m\'argenes, indicando el vigor de la vegetaci\'on o la densidad de la biomasa forestal. Estos indices junto con la comparaci\'on multitemporal hace posible identificar la evoluci\'on de coberturas vegetales en periodos de tiempos obteniendo resultados cualitativos y/o cuantitativos en espacio y tiempo \cite{martinez2013normalizacion}.\\~\\
En la actualidad existen diferentes m\'etodos para la detecci\'on de cambios de vegetaci\'on. Los m\'etodos requieren una supervisi\'on, un trabajo de campo y la utilizaci\'on de complejos sistemas de informaci\'on geogr\'afica. Estos sistemas de información geográfica son com\'unmente software de pago, elevando de esta manera el costo de dichos estudios. \\~\\
La falta de informaci\'on nos lleva a varios cuestionamientos referente a como estamos manejando nuestro medio ambiente y que efectos acarreara esos usos. El empleo de la teledetecci\'on y las im\'agenes satelitales multitemporales permiten realizar un an\'alisis a lo largo del tiempo de los cambios que el ambiente est\'a experimentando, mas aun en zonas como el Chaco Paraguayo, donde la informaci\'on ambiental es escaso por los altos costos y dificultades en el acceso al realizar controles en el terreno.\\~\\
Se propone dise\~{n}ar e implementar una metodolog\'ia autom\'atica que permita estimar la perdida de carbono a trav\'es de la biomasa, empleando procesamiento digital de im\'agenes satelitales, disponibles de forma libre, din\'amicos y no complejos.

\section{Justificación y Motivación}

REDD+ es una iniciativa que tiene como objetivo reducir la p\'erdida de bosques, teniendo como actividades principales \cite{peralta2013analisis}:
\begin{itemize}
	\item Evitar p\'erdidas como emisiones de gases de efecto invernadero (conservaci\'on, no deforestaci\'on, no degradaci\'on).
	\item Mantener el dep\'osito o stock de carbono (conservaci\'on, gesti\'on sostenible).
	\item Incrementar el dep\'osito por su efecto de retenci\'on o sumidero de carbono (conservaci\'on, restauraci\'on, gesti\'on sostenible).
\end{itemize}

El Paraguay se ha embarcado en el proceso de preparaci\'on para reducir la deforestaci\'on y degradaci\'on forestal, a fin de disminuir las emisiones de CO2, conservar los bosques y su biodiversidad. Este proceso deriva la necesidad de elaborar una estrategia nacional, con pol\'iticas socios ambientales y econ\'omicos viables.\\~\\
Para medir los beneficios de carbono de un proyecto REDD+ es necesario calcular la cantidad de carbono almacenado en el bosque en cuesti\'on y luego predecir la cantidad de carbono que se podr\'ia conservar si se detiene o reduce la deforestaci\'on y la degradaci\'on forestal \cite{nellemann2009carbono}.\\~\\
Las grandes extensiones de las \'areas de estudio, la dificultad de acceder a las mismas, el alto costo del establecimiento de las parcelas de inventario y su limitada utilidad hacen que la mayoría de las investigaciones para estimar y mapear la biomasa en bosques se centren en las t\'ecnicas de Sensores Remotos. \\~\\
La necesidad de crear metodolog\'ias que ayuden al monitoreo de forma din\'amica y barata, lleva al desarrollo de herramientas libres que permitan estimar focos de alerta para la toma de acciones y controles m\'as rigurosos a tiempo.


\section{Antecedentes}

Paraguay es un pa\'is que basa su econom\'ia en la agricultura y la ganader\'ia extensiva, actividades que han afectado al recurso forestal, dando como resultado extensas \'areas deforestadas y degradadas \cite{BAAPA2013}.\\~\\
	En el informe realizado por la ENPAB \cite{basualdo2003estrategia} se menciona que existe una fuerte presi\'on pol\'itica y social, proveniente de diversos grupos que buscan transformar las tierras del Chaco paraguayo en unidades econ\'omicas de producci\'on, cuyo enfoque gira en torno al crecimiento econ\'omico antes que al desarrollo sostenible. 
	En muchas zonas del chaco paraguayo, el modelo de desarrollo y uso de la tierra ha producido grandes extensiones de tierras altamente degradadas, arenales, desertificaci\'on y salinizaci\'on.\\~\\	
	A pesar que existen leyes de protecci\'on para evitar la deforestaci\'on y valorar los bosques, los mismos necesitan apoyo para su monitoreo y aplicaci\'on efectiva, debido a que los costos en tiempo y recursos son elevados.\\~\\
	Con el objetivo de implementar pol\'iticas de mitigaci\'on del cambio clim\'atico relativas a reducir las emisiones provenientes de la degradaci\'on y la deforestaci\'on (REDD+), los pa\'ises en desarrollo deben contar con estimaciones robustas s\'olidas en cuanto a las reservas de carbono forestal \cite{BAAPA2013}.

\section{Trabajos relacionados}\label{sec:antecedente}

El proyecto del Mapa global de carbono fue desarrollado por el Jet Propulsion Laboratory, California Institute of Technology en el a\~{n}o 2011 \cite{saatchi2011benchmark}, abarcando m\'as de 2.5 millones en hect\'areas de bosques, para tres continentes, trazando el stock total de carbono en la biomasa viva (por debajo y por encima). La incertidumbre media en las estimaciones realizadas para todos los continentes fue de $ \pm 30$\%.\\~\\
En el marco denominado \textit{Desarrollo del estudio de linea de base para el sitio piloto Bosque atl\'antico de Alto Paran\'a. (BAAPA)} \cite{BAAPA2013}, fue confeccionado un mapa de carbono en base al an\'alisis de regresi\'on entre el \'indice de vegetaci\'on diferencial normalizada (NDVI) y toneladas de carbono por hect\'area. El coeficiente de determinaci\'on hallado fue $ r^{2}=0.64 $.\\~\\
 Vanessa Almando Dur\'e \cite{kris2014estimacion} estima el carbono almacenado en el Parque Nacional Defensores del chaco seg\'un formaci\'on vegetal a trav\'es de im\'agenes satelitales . La metodolog\'ia utilizada corresponde a un an\'alisis de regresi\'on entre el NDVI y toneladas de carbono por hect\'area. La investigaci\'on evalu\'o tres tipos de comunidades vegetales obteniendo coeficientes de determinaci\'on similares a $ r_{1}^{2}=0.8$, $ r_{2}^{2}=0.7 $, $r_{3}^{2}=0.8 $.\\~\\
 Gustavo Miguel Huespe Duarte \cite{gustavo2012deteccion} detecta el cambios en la cobertura vegetal mediante indices de vegetaci\'on (NDVI), dentro y fuera de la reserva de la biosfera del Chaco en el periodo 1985-2011.\\~\\
Tyukavina et al. \cite{tyukavina2015aboveground} estima la p\'erdida de carbono sobre el suelo en los bosques tropicales naturales y gestionados para los a\~{n}os 2000-2012. La estimaci\'on fue hecha en todos los continentes, donde la incertidumbre en los resultados de Am\'erica Latina fueron de $ \pm 8 $\%.\\~\\
Maxime R\'ejou-M\'echain et al. \cite{rejou2015using} utiliza un modelo repetido para inferir variaciones espaciales y temporales de un bosque neotropical con alta biomasa mediante la adquisici\'on del small-footprint LiDAR. El modelo presenta un error cuadr\'atico medio (RMS) del $ 14\% $ para resoluciones de 1 hect\'area y de $ 23\% $ para una resoluci\'on de 0,25 hect\'areas en las variaciones de biomasa.\\~\\
Jean Pierre Ometto et al. \cite{ometto2015amazon} realiza un mapa densidad de biomasa forestal en el Amazonas. El m\'etodo estima las variaciones espaciales de la biomasa utilizando el an\'alisis de im\'agenes satelitales. El modelo presenta valores de incertidumbres en su estimaci\'on para el periodo 1990-1999 del $ \pm 15\% $ y 2000-2009 del $ \pm 14\% $.\\~\\
Michael W. Palace et al. \cite{palace2015estimating} realiza estimaciones de la estructura de un bosque tropical mediante mediciones de campo, un modelo sint\'etico y  discreto a partir de datos retornados por un sensor lidar. En estimaciones de campo realizadas para alturas dosel se obtuvo un $ r^{2}=0,17 $. El modelo sint\'etico desarrollado estima muchas propiedades estructurales de los bosque, tales como, medida de di\'ametro del tronco con $ r^{2}=0,51 $ y densidad de \'arboles con $ r^{2}=0,43 $.\\~\\
Nancy L. Harris et al. \cite{harris2012baseline} desarrolla un mapa de referencia para las emisiones de carbono producidas por deforestaciones en bosques tropicales. El mapa presenta un intervalo de predicci\'on de $ 0.57 $ a $ 1.22 $ petagramos de carbono por a\~{n}o (Pg C year$ ^{-1} $) .\\~\\
Un estudio realizado por University of Maryland Institute for Advanced Computer Studies denominado Forest Cover Change in Paraguay, nos muestra el cambio de vegetaci\'on estimado en todo el pa\'is entre los a\~{n}os 1990 y 2000 \cite{huang2009assessment}. Los resultados para todas las escenas (Fotograf\'ias \'areas y satelitales) obtuvieron precisiones globales mayor al 90\% y errores por comisi\'on y omisi\'on menores al 10\%.\\~\\
Xiao-Peng Song et al. \cite{song2014annual} realiza una detecci\'on anual de cobertura vegetal perdido, utilizando im\'agenes satelitales multitemporales que contienen porcentaje de cobertura vegetal. El m\'etodo empleado encuentra una relaci\'on entre im\'agenes satelitales de Vegetation Continuous Fields (VCF) y muestras de datos correspondientes a vegetaci\'on en im\'agenes satelitales Landsat. Los coeficientes de determinaci\'on van de $ 0.7 $ a $ 0.9 $ en los porcentajes de vegetaci\'on.\\~\\
Clovis Grinand et al. \cite{grinand2013estimating} estima la deforestaci\'on en los bosques húmedos y secos tropicales de Madagascar desde 2000 hasta 2010. La estimaci\'on es hecha utilizando im\'agenes satelitales Landsat multitemporales y la clasificaci\'on de bosques al azar. Los resultados obtiene un error de comisi\'on del 85\% para coberturas estables y 61\% para coberturas con cambios.\\~\\
Matthew L. Clark et al. \cite{clark2010scalable} propone un enfoque escalable para cartografiar anualmente la cubierta terrestre a 250 m utilizando datos MODIS de series de tiempo utilizando como caso de estudio la ecorregi\'on chaco seco de Am\'erica del Sur. El enfoque presenta una precisi\'on global del 79\% en la clasificaci\'on de tipos de coberturas.\\~\\
Dolors Armenteras et al. \cite{armenteras2013national} calcula determinantes nacionales y regionales de la deforestaci\'on tropical en Colombia. El estudio presenta los coeficientes de determinaci\'on para cada regi\'on del pa\'is y nacionalmente en base a la deforestaci\'on y los diferentes tipos de usos del suelo.\\~\\
Gu et. al \cite{gu2015downscaling} realiza un estudio para reducir la escala de 250 m que poseen las im\'agenes satelitales MODIS NDVI, sobre im\'agenes satelitales multitemporales Landsat utilizando enfoques de miner\'ias de datos. La temporada de evaluaci\'on fue la etapa de crecimiento vegetal. El coeficiente de determinaci\'on hallado a partir del MODIS NDVI y el calculado a trav\'es de im\'agenes landsat fue de $ r^{2}=0.97 $.\\~\\
En la siguiente tabla \ref{t:resumenrelacionado} se presenta un resumen de los trabajos relacionados.
% Please add the following required packages to your document preamble:
% \usepackage[table,xcdraw]{xcolor}
% If you use beamer only pass "xcolor=table" option, i.e. \documentclass[xcolor=table]{beamer}
	\begin{longtable}{|p{3cm}|p{3cm}|p{3cm}|p{3cm}|}



		 \hline
		 \endhead
		 \hline
		 % cell that spans multiple columns, justified right
		 \multicolumn{4}{|r|}{{Continua\ldots}} \\

		 \hline
		 \endfoot
		 
		 \hline 
		 		 		\caption{Resumen de trabajos relacionados.} \label{t:resumenrelacionado} \\
		 \endlastfoot
		 
\multicolumn{1}{|l|}{{\bf A\~{n}o}} & \multicolumn{1}{l|}{{\bf Autores}}                                                                                                                                            & \multicolumn{1}{l|}{{\bf Trabajo}} & \multicolumn{1}{l|}{{\bf Evaluaciones}} \\ \hline
		2011          & Saatchi et al.                                         & Benchmark map of forest carbon stocks in tropical regions across three continents                                                                                      & Incertidumbre media en las estimaciones del 30\%                                          \\ \hline
		2013          & ParLu, WWF Paraguay y la Facultad de Ciencias Agrarias & Desarrollo del estudio de linea de base para el sitio piloto Bosque atl\'antico de Alto Paran\'a. (BAAPA)                                                               & Coeficiente de determinaci\'on entre el NDVI y carbono r2=0.64                              \\ \hline
		2014           & Vanessa Almando Dur\'e                                 & Estimaci\'on de carbono almacenado en el Parque Nacional Defensores del Chaco seg\'un formaci\'on vegetal mediante im\'agenes satelitales, a\~{n}o 2014              & Coeficientes de determinac\'on $ r_{1}^{2}=0.8$, $ r_{2}^{2}=0.7 $, $r_{3}^{2}=0.8 $      \\ \hline
		2012          & Gustavo Miguel Huespe Duarte                           & Detecci\'on de cambios de la cobertura vegetal mediante indices de vegetaci\'on (NDVI), dentro y fuera de la Reserva de la biosfera del Chaco en el periodo 1985-2011  & Presenta conclusiones acerca del cambio detectado en diferentes regiones del caso de estudio.                                                                               \\ \hline
		2015          & Tyukavina et al.                                       & Aboveground carbon loss in natural and managed tropical forests from 2000 to 2012                                                                                      & Incertidumbre en los valores estimados de $ \pm 8 $\%                                     \\ \hline
		2015          & R\'ejou-M\'echain et al.                       & Using repeated small-footprint LiDAR acquisitions to infer spatial and temporal variations of a high-biomass Neotropical forest                                        & RMS 14\% y 23\% en las pruebas.                                                          \\ \hline
		2015          & Jean Pierre Ometto et al.                              & Amazon forest biomass density maps: tackling the uncertainty in carbon emission estimates                                                                              & Incertidumbre en los valores estimados de $ \pm 15\% $ y $ \pm 14\% $                     \\ \hline
		2015          & Michael W. Palace et al.                               & Estimating forest structure in a tropical forest using field measurements, a synthetic model and discrete return lidar data                                            & Coeficientes de determinaci\'on $ r^{2}=0,17 $, $ r^{2}=0,51 $, $ r^{2}=0,43 $            \\ \hline
		2012          & Nancy L. Harris et al.                                 & Baseline map of carbon emissions from deforestation in tropical regions                                                                                                & Intervalo de predicci\'on entre $ 0.57 $ a $ 1.22 $ (Pg C year$ ^{-1} $)                      \\ \hline
		2009          & Huang et al.                                           & Assessment of Paraguay's forest cover change using Landsat observations                                                                                                & Presiciones globales mayores al $90\%$ y errores por comisi\'on menores al $10\%$         \\ \hline
		2014          & Xiao-Peng Song et al.                                  & Annual detection of forest cover loss using time series satellite measurements of percent tree cover                                                                   & Coeficientes de determinaci\'on entre $ 0.7 - 0.9 $                                       \\ \hline
		2013          & Clovis Grinand et al.                                  & Estimating deforestation in tropical humid and dry forests in Madagascar from 2000 to 2010 using multi-date Landsat satellite images and the random forests classifier & Error de comisi\'on del 85\% para coberturas estables y 61\% para coberturas con cambios. \\ \hline
		2010          & Matthew L. Clark et al.                                & A scalable approach to mapping annual land cover at 250 m using MODIS time series data: A case study in the Dry Chaco ecoregion of South America                       & Presici\'on global del $79\%$                                                             \\ \hline
		2013          & Dolors Armenteras et al.                               & National and regional determinants of tropical deforestation in Colombia                                                                                               & Coeficientes de determinaci\'ons por regiones del pa\'is.                                 \\ \hline
		2015          & Gu et al.                                         & Downscaling 250-m MODIS Growing Season NDVI Based on Multiple-Date Landsat Images and Data Mining Approaches                                                           & Coeficiente de determinaci\'on $ r^{2}=0.97 $                                             \\ \hline		 

	\end{longtable}


En esta secci\'on pudimos observar que existen diferentes trabajos relacionados a la estimaci\'on de carbono y detecci\'on de cambio forestal, tanto a nivel nacional e internacional. Este trabajo pretende dise\~{n}ar una metodolog\'ia, interceptando los dos t\'opicos mencionados, que estime perdidas de carbono forestal de manera din\'amica. Aquellos procedimientos que agilizan la estimaciones ahorrando visitas al terreno y supervisiones humanas en los mismos son los que determinan el dinamismo.


\section{Objetivos}
Atendiendo a la necesidad de metodolog\'ias alternativas para el monitoreo de perdida de carbono en el campo ambiental, los objetivos delineados son los siguientes.

\subsection{Objetivo General}

\begin{itemize}
\item Desarrollar una metodolog\'ia autom\'atica de an\'alisis de im\'agenes satelitales multitemporales para la generaci\'on de indicadores respecto a la perdida del contenido de carbono en zonas del Chaco Paraguayo.
\item Dise\~{n}ar un m\'etodo de detecci\'on de cambio forestal automatizada entre secuencias multitemporales de im\'agenes satelitales. 
\end{itemize}
\subsection{Objetivos Espec\'ificos}
Para el logro del objetivo general los siguientes objetivos espec\'ificos son propuestos:
\begin{itemize}
\item Establecer normalizaciones de im\'agenes para la comparación multitemporal. 
\item Determinar una constante para la clasificaci\'on de vegetaci\'on en im\'agenes satelitales.
\item Determinar la relación entre el NDVI y el carbono a trav\'es de muestreos.
\item Evaluar la detecci\'on de cambio forestal con el estado del arte. 
 
    
%\item Implementaci\'on de la metodolog\'ia como complemento de una herramienta SIG de c\'odigo abierto.


\end{itemize}



\section{Organizaci\'on de la Tesis}

%La distribución de capítulos del presente trabajo final de grado se encuentra distribuido en 6 capítulos.
%en este capítulo se da una breve introducción al tema, se describe el problema de manera precisa para lograr su mejor entendimiento, también se citan los objetivos trazados tanto específicos como generales finalmente se habla de los antecedentes.
%en el capítulo 2  se realiza una breve introducción sobre la diabetes y los tipos existentes, 
%\begin{comment} luego hablaremos de la enfermedad de los ojos que se da en las personas con diabetes que es conocida como  retinopatía diabética, de la misma se menciona las causas, factores de riesgos y los síntomas. Además se menciona las etapas en cuales la enfermedad se va desarrollando, las anormalidades que se van dando dentro de los ojos, así como también los tratamientos usados para combatir esta enfermedad: fotocoagulación con láser, terapia médica intravítrea y tratamiento quirúrgico. 
%\end{comment} 
%en el capítulo 3  se presenta el marco teórico de las técnicas de  visión por computadora, %\begin{comment} desde los espacios de colores utilizados para representar las imágenes de fondo de ojo hasta los algoritmos se mencionan con detalle el funcionamiento del algoritmo de normalización utilizado,  en los procesos de segmentación, extracción de características y clasificación.
%\end{comment} 
%en el capítulo 4  se detallan  los algoritmos de detección y segmentación utilizados en el sistema de diagnóstico, además  se menciona las sub-segmentaciones utilizadas en estos procesos,
%Luego tenemos la extracción de características cuya importancia radica en el hecho de que reduce la cantidad de datos a procesar. Al final de la metodología tenemos al clasificador de máquina de vector de soporte el cual dará el diagnóstico final. 
%en el capítulo 5  se presenta las métricas  utilizadas para medir el desempeño, luego se evalúa los resultados obtenidos y se realiza  la comparación con respecto al estado del arte y por último en el capítulo 6 se presentan las conclusiones finales tras los experimentos y análisis de resultados del proyecto, por último los trabajos futuros que podrían dar continuidad al trabajo final de grado.


La distribución de capítulos del presente trabajo final de grado se encuentra organizado de la siguiente forma:
%en este capítulo se da una breve introducción al tema, se describe el problema de manera precisa para lograr su mejor entendimiento, también se citan los objetivos trazados tanto específicos como generales finalmente se habla de los antecedentes.
\begin{itemize}

\item En el cap\'itulo 2  se describen los conceptos generales relacionados al cambio clim\'atico y perdida de carbono.

\item En el cap\'itulo 3 se pretende dar un marco te\'orico acerca del procesamiento digital de im\'agenes satelitales.
%\end{comment} 
\item En el cap\'itulo 4  se detalla los algoritmos y procedimientos empleados en la metodolog\'ia de estimaci\'on de perdida de carbono.

\item En el cap\'itulo 5  se presenta las m\'etricas para medir la calidad de los resultados. Tambi\'en se evalu\'a los resultados en base a las m\'etricas previstas.

\item En el cap\'itulo 6 se presentan las conclusiones finales tras los experimentos y an\'alisis de resultados del proyecto, concluyendo con propuestas de trabajos futuros para dar continuidad al trabajo final de grado.

\end{itemize}

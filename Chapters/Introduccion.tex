\newpage{\ } 
\thispagestyle{empty} 

\chapter{Introducción}
\lhead{Capítulo 1. \emph{Introducción}} % This is for the header on each page - perhaps a shortened title


De entre los servicios ambientales que proporcionan los bosques, la captura de carbono ser\'a determinante para disminuir el calentamiento global y estabilizar el cambio clim\'atico producidos por el incremento en la atmósfera de los llamados Gases de Efecto Invernadero (GEI). El di\'oxido de carbono (CO2) es el gas mas abundante, contribuyendo con un 76\% al GEI \cite{avila2001almacenamiento} debido principalmente al cambio de paisajes de bosques tropicales maduros a paisajes agr\'icolas.\\~\\
Los bosques tropicales en condiciones naturales contienen m\'as carbono a\'ereo por unidad de superficie que cualquier otro tipo de cobertura terrestre. Por esto, cuando los bosques se convierten a otros usos del suelo, ocurre una gran liberaci\'on neta de carbono a la atm\'osfera. El cambio en el uso del suelo y la silvicultura son responsables del 15-20\% de las emisiones totales de gases de efecto invernadero\cite{peralta2013analisis}.\\~\\
El \textbf{ciclo de carbono} son las transformaciones qu\'imicas de compuestos que contienen carbono en los intercambios entre biosfera, atm\'osfera, hidrosfera y litosfera. La fotosíntesis de las plantas constituye un proceso fundamental en el ciclo ya que permite separar  el CO2 en oxigeno que consumimos y carbono (C) en materia \'organica, actuando en forma de almacenes de C como biomasa en función a la composición flor\'istica, la edad y la densidad de cada estrato por comunidad vegetal por periodos prolongados \cite{acosta2003diseno}.\\~\\
De manera general el t\'ermino biomasa se refiere a toda la materia org\'anica que proviene de \'arboles, plantas y desechos de animales que pueden ser convertidos en energ\'ia. En nuestro caso utilizaremos la definici\'on de biomasa forestal como la cantidad total de materia org\'anica a\'erea presente en los \'arboles incluyendo hojas, ramas, tronco principal y corteza\cite{garzuglia2003wood}.\\~\\
La teledetecci\'on o percepci\'on remota sin estar en un contacto f\'isico directo, nos permite adquirir im\'agenes de la superficie terrestre\cite{lillesand1994remote} empleando el uso de informaciones provenientes de sensores instalados en plataformas espaciales, complementados con sistemas de informaci\'on geogr\'aficas (SIG) para un an\'alisis mas continuo y din\'amico. Estos sensores remotos captan la energ\'ia reflejada o radiada por la superficie, ya sea emitida por el sol (sensores pasivos) o por el mismo sensor (sensores activos), para ser transformadas a valores digitales (VD) como imag\'enes satelitales, de manera secuencial para cada espacio de la tierra, a intervalos regulares de tiempo.\\~\\
Las coberturas vegetales poseen un comportamiento caracter\'istico en su radiaci\'on, permitiendo a trav\'es de la im\'agenes prove\'idos por los sensores remotos calcular \'indices que var\'ian dentro de margenes conocidos indicando el vigor de la vegetaci\'on o la densidad de la biomasa forestal. A esto, junto con la comparaci\'on multitemporal sera posible identificar la evoluci\'on de coberturas vegetales en periodos de tiempos obteniendo resultados cualitativos y/o cuantitativos en espacio y tiempo\cite{martinez2013normalizacion}.\\~\\
Existen muchos m\'etodos para la detecci\'on de cambios de vegetaci\'on pero en su mayor\'ia requieren una supervisi\'on y un trabajo de campo como tambi\'en la utilizaci\'on de complejos sistemas de informaci\'on geogr\'afica bajo licencia que elevan el costo de los estudios. En vista a esto, se propone dise\~{n}ar e implementar una metodolog\'ia autom\'atica que permita estimar la perdida de carbono a trav\'es de la biomasa de forma din\'amica, empleando procesamiento de im\'agenes satelitales disponibles de forma libre.\\~\\
Dentro de todo esto, la falta de una mayor informaci\'on nos lleva a varios cuestionamientos de como estamos manejando nuestro medio ambiente y de que efectos acarreara esos usos, por lo que el empleo de la teledetecci\'on y las im\'agenes satelitales multitemporales nos permitir\'an realizar un an\'alisis a lo largo del tiempo de los cambios que el ambiente est\'a experimentando, mas aun en zonas como el Chaco Paraguayo donde la informaci\'on referentes al ambiente son escasos a causa de los altos costos y las dificultades de acceso a la hora de realizar muestreos en el terreno.

\section{Justificación y Motivación}

REDD+ es una iniciativa que tiene como objetivo reducir la p\'erdida de bosques, las actividades REDD+ evitan p\'erdidas como emisiones de gases de efecto invernadero (conservaci\'on, no deforestaci\'on, no degradaci\'on), mantienen el dep\'osito o stock de carbono (conservaci\'on, gesti\'on sostenible), o incrementan el dep\'osito por su efecto de retenci\'on o sumidero de carbono (conservaci\'on, restauraci\'on, gesti\'on sostenible)\cite{peralta2013analisis}.\\~\\
El Paraguay se ha embarcado en el proceso de preparaci\'on para Reducir la Deforestaci\'on y Degradaci\'on forestal (REDD+) a fin de disminuir las emisiones de CO2, conservar los bosques y su biodiversidad, por tanto se busca elaborar una estrategia nacional, con pol\'iticas socios ambientales y econ\'omicos viables, as\'i como el desarrollo de capacidades.\\~\\
As\'i para medir los beneficios de carbono de un proyecto REDD+, es necesario calcular la cantidad de carbono almacenado en el bosque en cuesti\'on y luego predecir la cantidad de carbono que se podr\'ia conservar si se detiene o reduce la deforestaci\'on y la degradaci\'on forestal\cite{nellemann2009carbono}.\\~\\
La mayor\'ia de las investigaciones para estimar y mapear la biomasa en bosques se centran en las t\'ecnicas de Sensores Remotos; debido a las grandes extensiones de las \'areas de estudio, la dificultad de acceder a las mismas, el alto costo del establecimiento de las parcelas de inventario y su limitada utilidad debido a la variabilidad natural espacial de la biomasa forestal. Por ello la necesidad de crear metodolog\'ias que ayuden al monitoreo de forma din\'amica y barata nos lleva al desarrollo de herramientas libres que permitan estimar focos de alerta para la toma de acciones y controles m\'as rigurosos a tiempo.

\section{Antecedentes}
	Sassan Saatchi\cite{saatchi2011benchmark} ha mapeado el stock de carbono vivo en la biomasa para a\~{n}os proximos despu\'es de los 2000 utilizando una combinaci\'on de datos de 4079 en parcelas de inventario in situ y detecci\'on de luz v\'ia sat\'elite que van generando muestras de las estructuras de bosques de manera a estimar el almacenamiento de carbono, adem\'as de im\'agenes \'opticas y de microondas para extrapolar toda la superficie terrestre. De las parcelas de inventario in situ 493 fueron utilizados para verificar la consistencia de la estimaci\'on, donde 298 parcelas son de Latino Am\'erica presentando un error cuadr\'atico medio (RMSE) del 15\% en la predicci\'on.\\~\\
	Existen trabajos realizados por estudiantes de la Facultad de Ciencias Agrarias - UNA como proyecto de grado en zonas especificas como la reserva de la biosfera del Chaco, Parque Nacional San Rafael y el Parque Nacional Defensores del chaco, todos ellos en la regi\'on Occidental del Paraguay. Implementan una metodolog\'ia base hecha en el marco denominado \textit{Desarrollo del estudio de linea de base para el sitio piloto Bosque atl\'antico de Alto Paran\'a. (BAAPA)} realizado por el Paraguay Land Use (ParLu), el cual es una iniciativa de World Wildlife Fund (WWF)	Paraguay y WWF Alemania que apoya las iniciativas REDD+ en Paraguay, generando mapas de stock de carbono y los correspondientes mapas de cobertura y de Deforestaci\'on 2000\textendash2005 y 2005\textendash2011 a partir de muestreos de parcelas in situ y clasificaciones supervisadas con la ayuda de aplicaciones con licencias de car\'acter propietario, todo esto conjuntamente con la  Carrera de Ingenier\'ia Forestal de la Facultad de Ciencias Agrarias perteneciente a la Universidad Nacional de Asunci\'on .\\~\\
	Un estudio realizado por por University of Maryland Institute for Advanced Computer Studies denominado Forest Cover Change in Paraguay, nos muestra el cambio de vegetaci\'on estimado en todo el pa\'is utilizando un m\'etodo iterativo de etiquetado de cambio por clusterizaci\'on supervisada. El trabajo detecta cambios de los a\~{n}os 1992 al 2000, donde aparte de proveer un etiquetado de cambios de vegetaci\'on fue realizada con im\'agenes de acceso libre, generando informaci\'on m\'as precisa. Las validaciones fueron hechas con varias im\'agenes satelitales de alta precisi\'on, no libres, con una presici\'on global  en todas, mayor al 90\% para cambio/no cambio de \'areas forestales/no forestales\cite{huang2009assessment}.

\section{Planteamiento del problema}

Paraguay es un pa\'is que basa su econom\'ia en la agricultura y la ganader\'ia extensiva, actividades que han afectado al recurso forestal dando como resultado extensas \'areas deforestadas y degradadas.\\~\\
	En el informe realizado por la ENPAB \cite{basualdo2003estrategia} se menciona que existe una fuerte presi\'on pol\'itica y social, proveniente de diversos grupos que buscan transformar las tierras del Chaco paraguayo en unidades econ\'omicas de producci\'on, cuyo enfoque gira en torno al crecimiento econ\'omico antes que al desarrollo sostenible. 
	En muchas zonas del chaco paraguayo, el modelo de desarrollo y uso de la tierra ha producido grandes extensiones de tierras altamente degradadas, arenales, desertificaci\'on y salinizaci\'on.\\~\\	
	A pesar que existen leyes de protecci\'on para evitar la deforestaci\'on y valorar los bosques como la Ley de Deforestaci\'on Cero en la Regi\'on Oriental del Paraguay promulgada en el a\~{n}o 2004, y que ser\'a extendida hasta el 2018 y, la Ley de servicios ambientales 3001/06, entre otros instrumentos, los mismos necesitan apoyo para su monitoreo y aplicaci\'on efectiva, debido a que los costos en tiempo y dinero son elevados por la necesidad de realizar muestreos en el terreno y de adquirir licencias para las herramientas de monitoreo.\\~\\
	Con el objetivo de implementar Pol\'iticas de mitigaci\'on del Cambio Cl\'im\'atico relativas a reducir las emisiones provenientes de la degradaci\'on y la deforestaci\'on (REDD+), los pa\'ises en desarrollo deben contar con estimaciones robustas s\'olidas en cuanto a las reservas de carbono forestal\cite{BAAPA2013}.

\section{Objetivos}
Atendiendo a la necesidad de metodolog\'ias alternativas para el monitoreo de perdida de carbono en el campo ambiental, los objetivos delineados son los siguientes.

\subsection{Objetivos Generales}

\begin{itemize}
\item Desarrollar una metodolog\'ia autom\'atica de an\'alisis de im\'agenes satelitales multitemporales para la generaci\'on de indicadores respecto a la perdida del contenido de carbono en zonas del Chaco Paraguayo.
\end{itemize}
\subsection{Objetivos Específicos}
Para el logro de los objetivos generales los siguientes objetivos específicos son propuestos:
\begin{itemize}


\item Realizar detecciones de cambio automatizada dentro del \'area de estudio a trav\'es de la Teledetecci\'on y un SIG.   

\item Desarrollar ecuaciones que determinen la relación entre la biomasa y el carbono.
    
\item Implementaci\'on de la metodolog\'ia como complemento de una herramienta SIG de c\'odigo abierto.


\end{itemize}



\section{Organización de la Tesis}

%La distribución de capítulos del presente trabajo final de grado se encuentra distribuido en 6 capítulos.
%en este capítulo se da una breve introducción al tema, se describe el problema de manera precisa para lograr su mejor entendimiento, también se citan los objetivos trazados tanto específicos como generales finalmente se habla de los antecedentes.
%en el capítulo 2  se realiza una breve introducción sobre la diabetes y los tipos existentes, 
%\begin{comment} luego hablaremos de la enfermedad de los ojos que se da en las personas con diabetes que es conocida como  retinopatía diabética, de la misma se menciona las causas, factores de riesgos y los síntomas. Además se menciona las etapas en cuales la enfermedad se va desarrollando, las anormalidades que se van dando dentro de los ojos, así como también los tratamientos usados para combatir esta enfermedad: fotocoagulación con láser, terapia médica intravítrea y tratamiento quirúrgico. 
%\end{comment} 
%en el capítulo 3  se presenta el marco teórico de las técnicas de  visión por computadora, %\begin{comment} desde los espacios de colores utilizados para representar las imágenes de fondo de ojo hasta los algoritmos se mencionan con detalle el funcionamiento del algoritmo de normalización utilizado,  en los procesos de segmentación, extracción de características y clasificación.
%\end{comment} 
%en el capítulo 4  se detallan  los algoritmos de detección y segmentación utilizados en el sistema de diagnóstico, además  se menciona las sub-segmentaciones utilizadas en estos procesos,
%Luego tenemos la extracción de características cuya importancia radica en el hecho de que reduce la cantidad de datos a procesar. Al final de la metodología tenemos al clasificador de máquina de vector de soporte el cual dará el diagnóstico final. 
%en el capítulo 5  se presenta las métricas  utilizadas para medir el desempeño, luego se evalúa los resultados obtenidos y se realiza  la comparación con respecto al estado del arte y por último en el capítulo 6 se presentan las conclusiones finales tras los experimentos y análisis de resultados del proyecto, por último los trabajos futuros que podrían dar continuidad al trabajo final de grado.


La distribución de capítulos del presente trabajo final de grado se encuentra organizado en 6 capítulos.
%en este capítulo se da una breve introducción al tema, se describe el problema de manera precisa para lograr su mejor entendimiento, también se citan los objetivos trazados tanto específicos como generales finalmente se habla de los antecedentes.
\begin{itemize}

\item En el capítulo 2 .
%\begin{comment} luego hablaremos de la enfermedad de los ojos que se da en las personas con diabetes que es conocida como  retinopatía diabética, de la misma se menciona las causas, factores de riesgos y los síntomas. Además se menciona las etapas en cuales la enfermedad se va desarrollando, las anormalidades que se van dando dentro de los ojos, así como también los tratamientos usados para combatir esta enfermedad: fotocoagulación con láser, terapia médica intravítrea y tratamiento quirúrgico. 
%\end{comment} 
\item En el capítulo 3  . %\begin{comment} desde los espacios de colores utilizados para representar las imágenes de fondo de ojo hasta los algoritmos se mencionan con detalle el funcionamiento del algoritmo de normalización utilizado,  en los procesos de segmentación, extracción de características y clasificación.
%\end{comment} 
\item En el capítulo 4  .
%Luego tenemos la extracción de características cuya importancia radica en el hecho de que reduce la cantidad de datos a procesar. Al final de la metodología tenemos al clasificador de máquina de vector de soporte el cual dará el diagnóstico final. 
\item En el capítulo 5  .

\item En el capítulo 6 se presentan las conclusiones finales tras los experimentos y análisis de resultados del proyecto, por último los trabajos futuros que podrían dar continuidad al trabajo final de grado. 
\end{itemize}


\newpage{\ } 
\thispagestyle{empty} 

\chapter{Procesamiento de im\'agenes satelitales}
\lhead{Capítulo 3. \emph{Procesamiento de im\'agenes satelitales}} % This is for the header on each page - perhaps a shortened title
La teledetecci\'on presenta un principio base similar al de la visi\'on, permitiendo mediante una fuente de energ\'ia, un objetivo o escena y un sensor, generar im\'agenes digitales que posibilitan resaltar aquellos elementos dif\'iciles de percibir o ser distinguidos directamente a trav\'es de una imagen normal. El comportamiento caracter\'istico que poseen los recursos naturales a sensores remotos, nos posibilita el empleo amplio de t\'ecnicas de procesamiento de im\'agenes provechosos para el logro de los objetivos en la investigaci\'on \cite{deespectro}. \\~\\
Este capitulo consiste en brindar conceptos espec\'ificos utilizados por la metodolog\'ia, posibilitando comprender la influencia de cada factor en el empleo de im\'agenes satelitales para la estimaci\'on de p\'erdida del contenido de carbono forestal.

\section{Sensores Remotos}
Los sensores remotos nos permiten obtener informaci\'on de la superficie terrestre, soportados en diferentes plataformas (terrestre, a\'erea y sat\'elite), mediante la captura de energ\'ias reflejadas o radiadas proveniente del sol (sensores pasivos) o del mismo sensor (sensores activos) \cite{gustavo2012deteccion}. La energ\'ia capturada es transformada en productos, con diversos y diferentes especificaciones, siendo las fotografi\'as \'areas e im\'agenes de sat\'elites las m\'as conocidos.
\subsection{El espectro electromagn\'etico}
Las longitudes de ondas son continuas, pero de igual modo se establecen un serie de bandas donde las radiaciones manifiestan un comportamiento similar, organizandolas de este modo en un espectro electromagn\'etico \cite{remote2010abdulrahman}.
Las bandas m\'as empleadas son las siguientes \cite{salinero2002teledeteccion}:
	\begin{itemize}
		\item \textbf{Espectro visible:} (400 nm a 700 nm) se denomina as\'i por tratarse de la \'unica radiaci\'on electromagn\'etica que pueden percibir nuestros ojos, coincidiendo con las longitudes de onda en donde es m\'axima la radiaci\'on solar. Dentro de esta se distinguen tres bandas fundamentales: Azul (400 nm a 500 nm), verde (500 nm a 600 nm) y rojo (600 nm a 700 nm).
		\item \textbf{Infrarrojo pr\'oximo:} (700 nm a 1300 nm) se utiliza para discriminar masas vegetales y concentraciones de humedad.
		\item \textbf{Infrarrojo medio:} (1,3 um a 8 um) en esta franja se entremezclan los procesos de reflexi\'on de la luz solar y de emisi\'on de la superficie terrestre. El infrarrojo medio es muy utilizado para estimar el contenido de humedad en la vegetaci\'on y los focos de alta temperatura.
		\item \textbf{Infrarrojo lejano o térmico:} (8 um a 14 um) se detecta el calor de la mayor\'ia de las cubiertas terrestres.
		\item \textbf{Microondas:} (a partir de 1 um) de gran inter\'es por ser un tipo de energ\'ia transparente a la cubierta nubosa.
	\end{itemize}
	En la Figura \ref{fig:bandasIs} podemos observar como el sensor montado en una plataforma espacial capta la informaci\'on terrestre en diferentes bandas de acuerdo a la longitud de onda.
	\begin{figure}[H]
		\centering
		\includegraphics[width=0.7	\textwidth]{./Figures/cap3/bandas_imagen.png}
		\caption{Bandas capturadas por un sat\'elite \cite{teledet2015perce}.}
		\label{fig:bandasIs}
	\end{figure}

\subsection{Firmas espectrales}
Las firmas espectrales consisten en la representaci\'on de energ\'ia reflejada con relaci\'on a las longitudes de ondas, consideradas sin el efecto atmosf\'erico y medida en condiciones ideales del \'angulo incidente. Las firmas espectrales ayudan a identificar los objetos en la superficie terrestre debido a que cada uno presenta una respuesta espectral \'unica \cite{sivakumar2004satellite}.\\~\\
En la Figura \ref{fig:firmaEspectral} se observa como cada objeto difiere de los dem\'as en sus firmas espectrales.

\begin{figure}[H]
	\centering
	\includegraphics[width=0.9	\textwidth]{./Figures/cap3/firmaEspectral.jpg}
	\caption{Firmas espectrales de diferentes coberturas.}
	\label{fig:firmaEspectral}
\end{figure}

\subsection{Resoluciones de un sensor}

La resoluci\'on de un sensor se define como el menor cambio en la magnitud de entrada que puede ser apreciada en la magnitud de salida. El concepto de resoluci\'on implica al menos cuatro manifestaciones \cite{peralta2013analisis}: 
	\begin{itemize}
		
		\item \textbf{Resoluci\'on espacial:} es el tama\~{n}o que representa en el terreno una unidad de pixel de la imagen. Esta resoluci\'on tiene mucha importancia en la interpretaci\'on pues marca el nivel de detalle que ofrece. En la Figura \ref{fig:espatialRes} podemos observar que cuanto menor sea el tama\~{n}o del pixel, menor ser\'a tambi\'en la probabilidad de que corresponda a un compuesto de dos o m\'as \'areas fronterizas.
		\begin{figure}[H]
			\centering
			\includegraphics[width=0.4	\textwidth]{./Figures/cap3/resolucion_espacial_n.jpg}
			\caption{Resoluci\'on espacial \cite{chara2015sate}.}
			\label{fig:espatialRes}
		\end{figure}
			\item \textbf{Resoluci\'on espectral:} indica el n\'umero y anchura de las bandas espectrales que puede discriminar el sensor. Un sensor ser\'a tanto m\'as id\'oneo cuanto mayor n\'umero de bandas proporcione, ya que facilita la caracterizaci\'on espectral de las distintas cubiertas. En la Figura \ref{fig:espectralRes} se puede observar la comparaci\'on entre la resoluci\'on espectral de dos diferentes sensores espaciales.
				\begin{figure}[H]
					\centering
					\includegraphics[width=0.6	\textwidth]{./Figures/cap3/espectral_spot_landsat.png}
					\caption{Resoluci\'on espectral igual a 3 para el sensor SPOT y 7 en el sensor Landsat \cite{martinez2005percepcion}.}
					\label{fig:espectralRes}
				\end{figure}
		\item \textbf{Resoluci\'on radiom\'etrica:} es la sensibilidad del sensor para detectar variaciones en la cantidad de energ\'ia espectral recibida. La sensibilidad se expresa en bits e indica el n\'umero de los distintos niveles radiom\'etricos que puede detectar un sensor. En la Figura \ref{fig:radioRes} se puede observar diferentes resoluciones radiom\'etricas.
		\nomenclature[10]{$ r $}{Bits o Niveles radiom\'etricos de la imagen satelital.}	
						\begin{figure}[H]
							\centering
							\includegraphics[width=0.6	\textwidth]{./Figures/cap3/radiometrica_bits}
							\caption{Diferentes resoluci\'ones radiom\'etricas en im\'agenes satelitales .}
							\label{fig:radioRes}
						\end{figure}
		\item \textbf{Resoluci\'on temporal:} Este tipo de resoluci\'on se refiere al intervalo de tiempo entre muestras sucesivas de la misma zona de la cobertura terrestre. El ciclo de cobertura presentada por la Figura \ref{fig:temporaRes}, est\'a en funci\'on de las caracter\'isticas orbitales de la plataforma, su velocidad, el ancho de barrido del sensor y las caracter\'isticas de construcci\'on del sistema.
			\begin{figure}[H]
					\centering
					\includegraphics[width=0.6	\textwidth]{./Figures/cap3/resolucion_temporal_land.png}
					\caption{Resoluci\'on temporal de 16 d\'ias \cite{teledet2015Combi}.}
					\label{fig:temporaRes}
				\end{figure}
	\end{itemize}

\section{Im\'agenes satelitales}

Una imagen satelital es una funci\'on $ f:(x,y,i) \longrightarrow \{0,...,2^{r}\} $. Cada $ (x,y,i) $ indica la posición $ (x,y) $ en a banda $ i $, donde $ i \in \{1,...,k\} $, $ x \in \{0,...,m\} $ e $ y \in \{0,...,n\} $ para una matriz $ m \times n $, siendo $ k $ el numero de bandas y $ r $ la resoluci\'on radiom\'etrica en la imagen. Las im\'agenes satelitales son conocidos tambi\'en como raster \cite{vasquez2011mineria} y se puede representar de forma matricial. La Figura \ref{fig:imagenMultiespectral} nos muestra los ejes de coordenadas espaciales $ (x,y) $ para cada plano que representan las bandas, pudiendo acceder a valor de la intensidad o nivel digital mediante el nivel digital  $ f(x,y,i) $.
  \begin{figure}[H]
  	\centering
  	\includegraphics[width=0.8	\textwidth]{./Figures/cap3/imagen_satelital_k4.png}
  	\caption{Valor digital en una imagen satelital de 4 bandas $ (k=4) $ y resoluci\'on radiometrica $ r=8 $.}
  	\label{fig:imagenMultiespectral}
  \end{figure}
  	\nomenclature[12]{$i$}{ Banda de la imagen satelital.}
  	\nomenclature[13]{$f$}{ Imagen satelital.}
  	\nomenclature[13]{$(x,y)$}{ Coordenadas espaciales.}
  	\nomenclature[16]{$(x,y,i)$}{Coordenadas espaciales de la banda $ i $ en la imagen satelital.}
  	\nomenclature[17]{$k$}{ N\'umero de bandas que posee una imagen satelital.}
  	\nomenclature[18]{$f(x,y,i)$}{ Nivel digital, en la posici\'on $ (x,y) $ de la banda $ i $, de la imagen satelital.}



\subsection{Histogramas}
El histograma de una imagen satelital, con niveles digitales en el rango de $ [0,2^{r}] $, es una funci\'on discreta $ H(ND)=n_{ND} $, donde $ ND $ es el nivel digital y $ n_{ND} $ el n\'umero de pixeles en la imagen teniendo el nivel digital $ ND $ \cite{gonzalez2002woods}.\\~\\
  	\nomenclature[19]{$ND$}{ Nivel digital de una imagen satelital.}
  	\nomenclature[20]{$H(ND)$}{ Funci\'on discreta que determina la cantidad de apariciones de $ ND $, en la imagen satelital.}  	
  	\nomenclature[21]{$n_{ND}$}{ N\'umero de pixeles en la imagen satelital teniendo el nivel digital $ ND $.}  	  	
 	\begin{figure}[H]
 		\centering
 		\includegraphics[width=0.7	\textwidth]{./Figures/cap3/histograma_def.png}
 		\caption{Histograma de una imagen con niveles digitales de $ 0 $ a $ 255 $.}
 		\label{fig:histDef}
 	\end{figure}
El histograma de una una imagen satelital es una representaci\'on gr\'afica \'util de la informaci\'on contenida por las im\'agenes obtenidas a trav\'es de la percepci\'on remota. En la Figura \ref{fig:histDef} podemos observar que para cada nivel digital va asociado un n\'umero de apariciones en la imagen.\\~\\
 Los analistas a menudo despliegan el histograma en cada banda, ya que proporciona una apreciaci\'on de la calidad de los datos que presenta una imagen. Por ejemplo si el contraste es bajo o muy alto (histogramas estrechos y amplios); si son multimodales responden a distintos tipos de coberturas detectadas (agua, humedales, tipos de vegetaci\'on, etc.), si en  el histograma de la banda infrarroja cercana se encuentran picos desplazados hacia la derecha implicar\'ia que existe una alta probabilidad de aparici\'on vegetal en la imagen, entre otros an\'alisis.

\subsection{Combinaci\'on de bandas}
La visualizaci\'on de las im\'agenes de teledetecci\'on es mejor cuando se tiene una representaci\'on en colores, ya que el ojo humano percibe mejor las diferencias de color que los niveles de gris.\\~\\
Las combinaci\'on de tres bandas a color en las im\'agenes satelitales recibe el nombre de imagen de color compuesta \cite{com2015color}. Las im\'agenes de las distintas bandas se pueden combinar entre ellas para producir una imagen en color real o en falso color en funci\'on de las bandas escogidas. Esto se hace asignando a cada uno de los canales (RGB) de la pantalla de ordenador, una banda en particular.\\~\\
Las im\'agenes compuestas en color natural o real son combinaciones de las bandas 1 (azul) , 2 (verde) y 3 (rojo) que coinciden aproximadamente con la gama visual del ojo humano, por lo que se parecen bastante a lo que esperar\'iamos ver en una fotograf\'ia normal en color. Las im\'agenes de color real tienden a presentar un bajo contraste y un aspecto algo borroso. Ello es debido a que la luz azul es m\'as afectada que las dem\'as por la dispersión atmosf\'erica.\\~\\
Otras combinaciones de bandas distintas, generan im\'agenes en falso color. La naturaleza de los objetos que se quieren investigar, determina la selecci\'on de las tres bandas a combinar \cite{com2015color}. A continuaci\'on se describe algunas combinaciones posibles con im\'agenes Landsat para la identificaci\'on visual de aspectos terrestres \cite{lillesand2014remote}:
	\begin{itemize}
		\item \textbf{Bandas 3,2,1 (RGB):} Es una imagen de color natural. Refleja el \'area tal como la observa el ojo humano en una fotograf\'ia a\'erea a color.
		\item  \textbf{Bandas 7,4,2 (RGB):} Permite discriminar los tipos de rocas. Ayuda en la interpretaci\'on estructural de los complejos intrusivos asociados a los patrones volcano - tect\'onicos.
		\item  \textbf{Bandas 5,4,2 (RGB):} Es una imagen que no refleja los patrones en colores naturales (falso color), por lo tanto las carreteras pueden ser rojas, el agua amarilla y la vegetaci\'on azul.
		\item \textbf{Bandas 7,3,1 (RGB):} Ayuda a diferenciar tipos de rocas, definir anomal\'ias de color que generalmente son de color amarillo claro algo verdoso, la vegetaci\'on es verde oscuro a negro, los r\'ios son negros y con algunas coloraciones azules a celestes.		
	\end{itemize}
 La Figura \ref{fig:combinacionColor} muestra como es combinada las bandas (3,4,5) en los canales (R,G,B).
  \begin{figure}[H]
  	\centering
  	\includegraphics[width=0.8	\textwidth]{./Figures/cap3/combinacionColor.jpg}
  	\caption{Combinaci\'on de bandas espectrales a trav\'es de los canales RGB.}
  	\label{fig:combinacionColor}
  \end{figure}

\section{\'Algebra de mapas}
El \'algebra de mapas constituye un marco te\'orico en la mayor parte de las operaciones hechas con SIG a partir de raster. Pueden desarrollarse operaciones de muy diverso tipo que se clasifican \cite{tomlin1990map} en:
	\begin{itemize}
		\item \textbf{Operadores locales:} los operadores locales generan una nueva imagen a partir de una o m\'as im\'agenes previamente existentes. Cada pixel de la nueva imagen recibe un valor que es funci\'on de los valores de ese mismo pixel en las dem\'as im\'agenes.
		\begin{equation}
		\label{e:opLoc}
		f_{1,2,3}=\rho(f_{1},f_{2},f_{3})
		\end{equation}
		\nomenclature[22]{$\rho$}{Funci\'on que representa a una operaci\'on aritm\'etica, l\'ogico, entre otros.}
		Donde $ \rho $ representa a alguna funci\'on del tipo:
		\begin{itemize}
			\item Aritm\'etico (suma, resta, multiplicaci\'on, divisi\'on, raiz cuadrada, potencia, ...).
			\item L\'ogico (AND, OR, XOR, NOT).
			\item Relacional ($ >, >=, <, <=, ==, != $).
			\item Trigonom\'etrico (sen, cos, tan, ...).
			\item Condicional (si cumple la condición ejecuta la instrucción).
		\end{itemize}
		La Figura \ref{fig:oploccond} nos muestra el proceso de operadores locales l\'ogico y condicional. En el \'item (a), el operador l\'ogico binariza la imagen si $ DEM > 400 $, mientras que en el \'item (b) el operador condicional clasifica la imagen en base a un rango ($ 600 - 650 = 1; 650 - 700 = 2; 700 - 750 = 3; 750 - 800 = 4 $).
		  \begin{figure}[H]
		  	\centering
		  	\includegraphics[width=0.9	\textwidth]{./Figures/cap3/operadores.png}
		  	\caption{Operadores locales condicional y l\'ogico.}
		  	\label{fig:oploccond}
		  \end{figure}
		\item  \textbf{Operadores de vecindad:} adjudican a cada pixel un valor que es funci\'on de los valores de un conjunto de pixeles contiguas, en una o varias im\'agenes. El conjunto de pixeles contiguas al pixel, m\'as ella misma constituye una vecindad.
		\begin{itemize}
			\item Filtrado de im\'agenes: es el conjunto de t\'ecnicas que se aplican a las im\'agenes digitales con el objetivo de mejorar la calidad o facilitar la b\'usqueda de informaci\'on.
			\item Operadores estad\'isticos: calcula variables estad\'isticas (media, desviaci\'on t\'ipica, m\'inimo, m\'aximo, entre otros.) a partir de los valores de todas los pixeles que forman la vecindad y lo adjudican al pixel central en la imagen de salida. 
			\item Operadores direccionales: Permiten estimar un conjunto de par\'ametros relacionados con la ubicaci\'on de los diferentes valores dentro de la vecindad. Su utilidad primordial es el an\'alisis de Modelos Digitales de Terreno (pendiente, orientación, curvatura, entre otros.)
		\end{itemize}		
		\item  \textbf{Operadores de vecindad extendida:} son aquellos que afectan a zonas relativamente extensas, que cumplen determinado criterio pero cuya localizaci\'on precisa no se conoce previamente. Por tanto el operador (algoritmo) debe determinar previamente cual es el \'area que cumple dichas caracter\'isticas. En la Figura \ref{fig:oplvecext} podemos ver el resultado de haber aplicado un operador de vecindad extendida a partir de pixeles situados a distancias $ 25,50,75,100,125 $.
						  \begin{figure}[H]
						  	\centering
						  	\includegraphics[width=0.9	\textwidth]{./Figures/cap3/operadorExte.png}
						  	\caption{\'Areas situadas a una distancia inferior a los valores umbrales 25,50,75,100,125.}
						  	\label{fig:oplvecext}
						  \end{figure}		
		\item \textbf{Operadores de \'area o zonales:} son aquellos que calculan alg\'un par\'ametro (superficie, per\'imetro, \'indices de forma, distancias, estad\'isticos) para una zona previamente conocida. Los valores pueden tratarse de diferentes niveles de una variable cualitativa o digitalizada e introducida por el usuario. En la Figura \ref{fig:oplarea} se observa tres im\'agenes. La primera est\'a clasificada en base a alg\'un criterio (Variable cualitativa) y la otra con niveles digitales igual a la altitud (Variable cuantitativa), donde la imagen resultante corresponde a la altitud media para cada grupo.
								  \begin{figure}[H]
								  	\centering
								  	\includegraphics[width=0.9	\textwidth]{./Figures/cap3/operadorearea.png}
								  	\caption{Operador de \'area: Altitud media por \'areas.}
								  	\label{fig:oplarea}
								  \end{figure}		
		
		\end{itemize}
 
\section{\'Indices de vegetaci\'on}
Los \'indices de vegetaci\'on son transformaciones que implican efectuar una combinaci\'on matem\'atica, entre los niveles digitales almacenados en dos o m\'as bandas espectrales de la misma imagen, teniendo en cuenta el comportamiento radiom\'etrico de la vegetaci\'on vigorosa para la elecci\'on de bandas \cite{speranza2005potencialidad}. \\~\\
El estudio de las cubiertas vegetales mediante la teledetecci\'on se aborda tradicionalmente mediante la utilización de los denominados “índices de vegetaci\'on”, siendo el m\'as utilizado el \'Indice de vegetaci\'on diferencial normalizada (NDVI) \cite{sader2000estimacion}.

\subsection{\'Indice de vegetaci\'on diferencial normalizada}\label{subsec:ndvi}
Sea una funci\'on $ ndvi:(x,y) \longrightarrow [-1,1] $ que determina la imagen con los NDVI en cada coordenada espacial $ (x,y) $ definida por la siguiente expresi\'on:
\begin{equation}
\label{e:ndvi}
ndvi(x,y)=\dfrac{f(x,y,IRc)-f(x,y,R)}{f(x,y,IRc)+f(x,y,R)}
\end{equation}
\nomenclature[23]{$ndvi$}{Imagen del \'indice de vegetaci\'on diferencial normalizada (NDVI).}
\nomenclature[24]{$ndvi(x,y)$}{ NDVI de las coordenadas $ (x,y) $.}
\nomenclature[25]{$R$}{ Posici\'on de la banda roja en la imagen satelital.}
\nomenclature[26]{$IRc$}{ Posici\'on de la banda infrarroja cercana en la imagen satelital.}
Donde $ R \in \{1,...,k\}$ representa la banda roja del espectro visible y  $ IRc \in \{1,...,k\}$ a la banda infrarroja cercana del espectro infrarrojo.\\~\\
En la Figura \ref{fig:firmaVegetacion} podemos observar como las plantas muestran un fuerte pico de absorci\'on causados por los pigmentos fotosint\'eticos en longitudes de onda cercanas a los 700 micrones (banda roja), hecho que contrasta con una fuerte reflexi\'on de las longitudes de onda del infrarrojo cercano \cite{salinero2002teledeteccion}. Por su parte, los suelos desnudos se caracterizan por un incremento suavemente monot\'onico de la reflectancia, a medida que aumenta la longitud de onda \cite{salinero2002teledeteccion}. Estas caracter\'isticas relevantes nos permiten elaborar varios tipos de an\'alisis como extracci\'on de \'indices.
\begin{figure}[H]
	\centering
	\includegraphics[width=0.8 \textwidth]{./Figures/cap3/Firma_espectral_vegetacion_vigorosa.jpg}
	\caption{Firma espectral de la vegetaci\'on \cite{ndvi2015com}.}
	\label{fig:firmaVegetacion}
\end{figure}

\subsubsection{Caracter\'isticas del NDVI}\label{subsec:subndvi}
El NDVI es un \'indice usado para estimar la cantidad, calidad y desarrollo de la vegetaci\'on por medio de sensores remotos instalados com\'unmente desde la plataforma espaciales, es decir mide las condiciones de vigor vegetal de la planta, principalmente su contenido en clorofila \cite{salinero2002teledeteccion}. El objetivo del NDVI es la reducci\'on de m\'ultiples bandas a una sola, condensando la informaci\'on m\'as importante, en este caso la vegetaci\'on.\\~\\
La principal ventaja del NDVI es su f\'acil interpretaci\'on, ya que sus valores var\'ian entre -1 y +1, permitiendo conocer el estado de vigor vegetal en grandes superficies y detecta fen\'omenos de amplio rango \cite{salinero2002teledeteccion}.

\section{An\'alisis Multitemporal}
El an\'alisis multitemporal de im\'agenes satelitales consiste en el estudio de zonas determinadas mediante tomas hechas en diferentes tiempos. El factor temporal puede abordarse con un doble objetivo: por un lado reconstruir la variaci\'on estacional de la zona y por otra parte la detecci\'on de cambios. Este \'ultimo objetivo se enfoca en detectar cambios entre dos o m\'as
fechas alejadas en el tiempo, estudiando el dinamismo temporal de una determinada zona como por ejemplo: el crecimiento urbano, transformaciones agrícolas, entre otras \cite{salinero2002teledeteccion}.\\~\\
En el enfoque aplicado al estudio multitemporal resulta preciso abordar previamente una serie de tratamientos sobre las im\'agenes satelitales de cara a garantizar su comparabilidad, ya que existen factores naturales o las del sensor, que influyen desde la captura de informaci\'on hasta su transformaci\'on final a niveles digitales que afectar\'ia el an\'alisis.

\section{Correcciones a las im\'agenes satelitales}
Las correcciones satelitales son el producto de aplicar un operador a una imagen satelital para la obtenci\'on de otra. Las correcciones est\'an definida seg\'un la expresi\'on:
		\begin{equation}
			f^{'}=T[f]
		\end{equation} 
Donde $ f $ es una imagen satelital de entrada, $ f^{'} $ es la imagen corregida y $ T $ es un operador que realiza las correcciones a la imagen $ f $, debido a fallos en los sensores, alteraciones en el movimiento del sat\'elite o interferencias de la atm\'osfera \cite{teledUm}.
\nomenclature[27]{$ T[f] $}{ Operador que corrige la imagen satelital $ f $.}
\subsection{Correccci\'on geom\'etrica}\label{sec:corrGeometrica}
Una imagen de sat\'elite, al igual que las fotograf\'ias a\'ereas, no proporciona informaci\'on georreferenciada; cada pixel se ubica en un sistema de coordenadas arbitrario de tipo fila-columna como los que manejan los programas de tratamiento digital de im\'agenes \cite{deniseCultivos}.\\~\\
El proceso consiste en dar a cada pixel su localizaci\'on en un sistema de coordenadas estandard (UTM, lambert, coordenadas geogr\'aficas) para poder combinar la imagen de sat\'elite con otro tipo de capas en un entorno SIG. El proceso obtiene una nueva capa en la que cada columna corresponde con un valor de longitud y cada fila con un valor de latitud. En caso de que la imagen no hubiese sufrido ningún tipo de distorsi\'on, el procedimiento ser\'ia bastante sencillo, sin embargo una imagen puede sufrir diversos tipos de distorsiones.\\~\\
Es necesario localizar puntos comunes de la imagen con puntos de control, como tarea inicial para la correcci\'on geom\'etrica, de manera a poder realizar una interpolaci\'on espacial y de los valores radiom\'etricos \cite{deniseCultivos}.

\subsubsection{Interpolaci\'on espacial}
La interpolaci\'on espacial consiste en determinar la relaci\'on geom\'etrica entre las coordenadas del pixel de la imagen a corregir y sus coordenadas geogr\'afica correspondientes. Utilizando los puntos comunes localizados, se plantea una ecuaci\'on de transformaci\'on mediante la cual se obtiene la posici\'on de los pixeles en la imagen de salida, ilustrada en la Figura \ref{fig:intEspacial}. Este proceso tambi\'en es conocido como Georreferenciaci\'on.  
    \begin{figure}[H]
    	\centering
    	\includegraphics[width=0.6	\textwidth]{./Figures/cap3/inter_spacial.png}
    	\caption{Localizaci\'on de puntos comunes y puntos de referencia.}
    	\label{fig:intEspacial}
    \end{figure}
    
\paragraph{Transformación usando ecuaciones polinomicas. }\mbox{}\\\mbox{}\\
El m\'etodo mas utilizado para la transformaci\'on es el de ecuaciones polin\'omicas.	 La transformaci\'on puede expresarse de la siguiente manera:
	\begin{equation}
	x^{'}_{i} = \sum_{j=0}^{l} \sum_{e=0}^{l-j} a_{ij}x^{j}_{i}y^{e}_{i}
	\end{equation} 
		\begin{equation}
		y^{'}_{i} = \sum_{j=0}^{l} \sum_{e=0}^{l-j} b_{ij}x^{j}_{i}y^{e}_{i}
		\end{equation} 


\nomenclature[28]{$ l $}{ Grado del polinomio de ajuste.}
Donde $ x^{'}_{i} $ e $ y^{'}_{i} $ indica la coordenada en la imagen corregida para la banda $ i $. El superindice $ l $ indica el grado del polinomio de ajuste, $ a_{i} $ y $ b_{i} $ los coeficientes del polinomio. Siendo la ecuaci\'on lineal las mas simple:
	\begin{equation}
	x^{'}_{i} = a_{0}+a_{1}x_{i}+a_{2}y_{i}
	\end{equation} 
		\begin{equation}
		y^{'}_{i} = b_{0}+b_{1}x_{i}+b_{2}y_{i}
		\end{equation} 
		\nomenclature[29]{$ (x^{'},y^{'}) $}{Coordenadas de la imagen transformada.}
En distorsiones moderadas o en un \'area reducida, se utilizan transformaciones de primer orden, pudiendo corregir efectos de translaci\'on en $ x^{'}_{i} $ e $ y^{'}_{i} $, cambios de escala y rotaci\'on.
En distorsiones m\'as importantes o en \'areas extensas, es necesario una transfomaci\'on de segundo orden. Este tipo de transformaci\'on agregan a diferencia del primer orden, correcciones a deformaciones locales.
En la Figura \ref{fig:intPolEcua} podemos observar las transformaciones con polinomios de primer y segundo orden.
    \begin{figure}[H]
    	\centering
    	\includegraphics[width=0.9	\textwidth]{./Figures/cap3/ecuacPolinomica.png}
    	\caption{Interpolaci\'on espacial con polinomios de primer y segundo orden.}
    	\label{fig:intPolEcua}
    \end{figure}
  \paragraph{Calidad de la interpolaci\'on espacial. }\mbox{}\\\mbox{}\\
 La calidad en la interpolaci\'on espacial y los puntos de control seleccionados $ \eta $ se calcula utilizando el promedio de los errores cuadráticos medios (RMS), que consiste en la diferencia entre la coordenada transformada deseada para un punto de control y la coordenada real obtenida como salida.

 \begin{equation}
 RMS_{i} = \sqrt{\dfrac{\sum_{j=1}^{\eta} ((x_{i,j}^{'}-x_{i,j})^{2}+(y_{i,j}^{'}-y_{i,j})^{2})}{\eta}}
 \end{equation} 
 		\nomenclature[30]{$ RMS_{i} $}{Error cuadr\'atico de la banda $ i $.}
 		\nomenclature[31]{$ \eta $}{N\'umero de puntos de control.}

 	El valor de $ RMS $ elegido por referencia para corregir una imagen debe ser aproximadamente $ 0.5 $, y en lo posible nunca superar la unidad \cite{guide1999erdas}. \\~\\
 	La Figura \ref{fig:rms} nos muestra la manera de como es calculado el RMS para un punto de control determinado.
 
 
     \begin{figure}[H]
     	\centering
     	\includegraphics[width=0.9	\textwidth]{./Figures/cap4/rms.png}
     	\caption{Error RMS de un punto de control $ (x,y) $ y su transformaci\'on $ (x^{'},y^{'}) $.}
     	\label{fig:rms}
     \end{figure}



\subsubsection{Interpolaci\'on de los valores radiom\'etricos}
La interpolaci\'on de los valores radiom\'etricos es el traslado del nivel digital perteneciente a la imagen original a la imagen corregida espacialmente. La imagen original debe corresponderse con las coordenadas de la imagen corregida. La interpolaci\'on puede ser abordada por tres m\'etodos diferentes:
	\begin{itemize}
		\item \textbf{Vecino m\'as pr\'oximo:} situ\'a en cada pixel de la imagen corregida el nivel digital $ (ND) $ del pixel m\'as cercano en la imagen original. Constituye la soluci\'on m\'as r\'apida y la que supone menor transformaci\'on en los niveles digitales originales. Su principal inconveniente es que produce una distorsi\'on en rasgos lineales en la imagen (fracturas, carreteras, caminos), que pueden aparecer en la corregida como lineales quebradas. \\~\\
En la Figura \ref{fig:vecinoMasCercano2} se observa el como los pixeles de la imagen transformada son trasladados a la imagen corregida a lado del vecino m\'as pr\'oximo.
				    \begin{figure}[H]
				    	\centering
				    	\includegraphics[width=0.6	\textwidth]{./Figures/cap3/vecinoMasCercano2.png}
				    	\caption{Interpolaci\'on Vecino m\'as Cercano.}
				    	\label{fig:vecinoMasCercano2}
				    \end{figure}
		
		\item \textbf{Interpolaci\'on bilineal:} Considera el valor de los 4 pixeles mas cercanos en la imagen de entrada para asignar el nuevo valor de la imagen de salida. Las ventajas son que no existe el efecto de escalones en los bordes pudiendo aparecer en el vecino superior izquierdo y ademas cuenta con mejor exactitud espacial. El m\'etodo es utilizado a menudo cuando se cambia el tama\~{n}o de las celdas en los datos. La desventaja es que como los pixeles son promediados, algunos extremos de los valores de los datos pueden perderse. En la Figura \ref{fig:bilineal2} podemos observar los 4 pixeles cercanos tomados para la interpolaci\'on.
		\begin{figure}[H]
			\centering
			\includegraphics[width=0.6	\textwidth]{./Figures/cap3/bilineal.png}
			\caption{Interpolaci\'on Bilineal.}
			\label{fig:bilineal2}
		\end{figure}
		
		    		\item \textbf{Convoluci\'on c\'ubica:} es similar a la interporlaci\'on bilineal pero considera niveles digitales de los 16 pixeles m\'as pr\'oximos. El efecto visutal es mejor, pero supone un volumen de c\'alculo mucho m\'as elevado. La Figura \ref{fig:convCubica2} nos muestra los 16 pixeles tomados en el m\'etodo. 
		    				    \begin{figure}[H]
		    				    	\centering
		    				    	\includegraphics[width=0.3	\textwidth]{./Figures/cap3/convolucionCubica_matriz.png}
		    				    	\caption{Convoluci\'on c\'ubica.}
		    				    	\label{fig:convCubica2}
		    				    \end{figure}
	\end{itemize}
\subsection{Correcci\'on radiom\'etrica}
La correci\'on radiom\'etrica se encarga de minimizar los desajustes producidos en el registro del valor digital en los pixeles de la imagen, de hecho en algunos casos las estaciones receptoras llevan a cabo alg\'un tipo de correcci\'on en el momento de recepci\'on de la imagen. La corrección radiom\'etrica implica por una parte la restauraci\'on de lineas o p\'ixeles perdidos y por otra la correcci\'on del bandeado en la imagen \cite{teledUm}.
    \begin{figure}[H]
    	\centering
    	\includegraphics[width=0.9	\textwidth]{./Figures/cap3/correcError.png}
    	\caption{Fallos del sensor en la captura de la imagen.}
    	\label{fig:correcError}
    \end{figure}
La Figura \ref{fig:correcError} nos muestran tres tipos de errores radiom\'etricos frecuentes, donde la lineas claras representan pixeles que no representan su nivel digital correcto a causa de la descalibraci\'on del sensor. Las lineas y pixeles negros son pixeles nulos que no pudieron ser convertidos a su nivel digital por fallos en el detector o transmisiones, como tambi\'en por conversiones de la informaci\'on anal\'ogica a digital.

\subsubsection{Pixeles o lineas perdidas}\label{subsec:pixelesP}
Sea $ f^{'} $ una imagen satelital con los pixeles corregidos, mediante estimaciones de la media entre los pixeles adyacentes. Donde cada nivel digital es calculado de la siguiente manera: 

		\begin{equation}\label{ec:pixelPerdido}
		f^{'}(x,y,i) =  ceiling(1/2 \times [f(x-1,y,i) + f(x+1,y,i)])
		\end{equation} 
Donde $ ceiling(:) $ representa la funci\'on techo (redondeo para arriba). No es recomendable utilizar los pixeles adyacentes de la misma linea (eje $ y $) por que han sido captados por el mismo detector o banda que ha dado el fallo, por tanto son poco fiables.

				\nomenclature[42]{$ \mu $}{Valor de la media.}
				\nomenclature[43]{$ \sigma $}{Desviaci\'on t\'ipica.}
				\nomenclature[44]{$ i_{*} $}{Banda de la imagen satelital.}
				\nomenclature[45]{$ \sigma_{i} $}{Desviaci\'on t\'ipica de la banda $ i $ de una imagen satelital.}
				\nomenclature[46]{$ \sigma_{i_{*}} $}{Desviaci\'on t\'ipica de la banda $ i_{*} $ de una imagen satelital.}
La bandas de una imagen son de detectores diferentes y est\'an altamente correlacionadas, por lo que se hace una modificaci\'on a la ecuaci\'on \ref{ec:pixelPerdido} teniendo en cuenta la banda $ i_{*} $ de la imagen satelital, donde $ i_{*} \in [1,k] $, pudiendo utilizarse el valor del pixel faltante en una banda diferente para mejorar la estimaci\'on:
		\begin{equation}
		f^{'}(x,y,i) = ceiling( dif_{i}+ \dfrac{\sigma_{i}}{\sigma_{i_{*}}} \times [f(x,y,i_{*})-dif_{i_{*}}])
		\end{equation} 
\nomenclature[47]{$ f^{'}(x,y,i) $}{Imagen satelital corregida, con coordenadas espaciales $ (x,y) $ en la banda $ i $.}
Donde $ \sigma_{i} $ representa la desviaci\'on t\'ipica de la banda $ i $, $ \sigma_{i_{*}} $ representa la desviaci\'on t\'ipica de la banda $ i_{*} $ y la variable $ dif_{e} $, donde $ e \in \{ i,i_{*}\} $, esta representada por la siguiente expresi\'on:
		\begin{equation}
		 dif_{e}  = 1/2 \times [f(x-1,y,e) + f(x+1,y,e)]
		\end{equation} 
En caso de que la imagen abarque un territorio amplio y cambiante resulta recomendable calcular las desviaciones t\'ipicas ($ \sigma_{i} $ y $ \sigma_{i_{*}} $) en un entorno cercano al pixel perdido.\\~\\
Para detectar lineas perdidas se compara la media de los $ ND $ de una linea con las medias de las lineas anterior y posterior, para detectar pixeles perdidos se compara el valor de un pixel con los de los 8 pixeles.
\subsubsection{Bandeado}\label{subsec:bandeado}
El fen\'omeno del bandeado se debe a una mala calibraci\'on entre detectores y resulta especialmente visible en las zonas de baja radiancia (zonas marinas por ejemplo). El resultado es la aparici\'on peri\'odica de una banda m\'as clara u oscura que las dem\'as.
Para corregir el bandeado se asume que, en caso de no haber error, los histogramas obtenidos por cada uno de los detectores ser\'ian similares entre s\'i y similares al histograma global de la imagen que se toma como referencia.\\~\\
En primer lugar se calculan los coeficientes $ m $ y $ s $ para una correcci\'on lineal de cada uno de las bandas.
		\begin{equation}
		m =\dfrac{\sigma_{i}}{\sigma_{i_{*}}}
		\end{equation} 	
				\begin{equation}
				s=\mu_{i} - m \times \mu_{i_{*}}
				\end{equation} 	
				
Siendo $ \mu_{i} $ y $ \mu_{i_{*}} $ las medias para la banda $ i $ e $ i_{*} $. Donde cada nivel digital de la imagen satelital $ f^{'} $ es calcula de la siguiente manera:
				\begin{equation}
				f^{'}(x,y,i) = m \times f(x,y,i) + s
				\end{equation} 				

En Figura \ref{fig:bandeado} podemos observar el histogramas de una banda corregida en funci\'on a las dem\'as bandas pertenecientes a una imagen satelital. 
    \begin{figure}[H]
    	\centering
    	\includegraphics[width=0.9	\textwidth]{./Figures/cap3/bandeo_k.png}
    	\caption{Proceso de correcci\'on del bandeo.}
    	\label{fig:bandeado}
    \end{figure}

\section{Proceso de detecci\'on de cambios}
	En los m\'etodos comunes de detecci\'on de cambios se asigna un valor correspondiente al grado de cambio sobre cada pixel, independientemente del resto de la imagen satelital. En estos m\'etodos se considera el pixel como unidad b\'asica (\'algebra de mapas) para aplicar las correspondientes operaciones matem\'aticas asociadas a cada algoritmo.\\~\\
Los m\'etodos de comparaci\'on, generan una imagen (\'indice de cambios) que representa el grado de cambio entre dos situaciones temporales; los pixeles de la imagen resultante, contienen una variable continua de tipo cuantitativo (Niveles digitales), por lo que se requieren t\'ecnicas que los conviertan en variables cualitativa (Categor\'ias) \cite{martinez2013normalizacion}.
\subsection{Comparaci\'on multitemporal}\label{subsec:compMult}
La comparaci\'on parte de un par de im\'agenes semejantes que abarcan la misma zona de estudio, siguiendo una secuencia multitemporal. Una secuencia multitemporal esta definido por $ \{f_{t}\}_{t \in \mathbb{N}} $, donde cada imagen satelital representa a una en diferentes tiempos. Las operaciones m\'as utilizadas son \cite{chuvieco1998factor}: 
	\begin{itemize}
		\item \textbf{Diferencia de im\'agenes:} es el m\'etodo m\'as simple, f\'acil de interpretar y directo, ya que consiste en una diferencia algebraica entre los niveles digitales ($ ND $ ) inicial y final para la obtenci\'on de un \'indice ($ I_{dif} $). Normalmente es realizada combinada  con extracciones de \'indices espectrales.
								\begin{equation}
								I_{dif} = f_{t_{*}}-f_{t}
								\end{equation} 	
		Donde $ t_{*} \neq t $.
\nomenclature[49]{$ I_{dif} $}{Indice de cambio por diferencia de im\'agenes.}
\nomenclature[50]{$ I_{ratio} $}{Indice de cambio por el m\'etodo de ratio.}
				\item \textbf{Ratio:} se obtiene aplicando la operación de cociente, entre los niveles digitales ($ ND $ ) inicial y final para la obtenci\'on de un \'indice ($ I_{ratio} $). Podr\'ia  generar mejores resultados pero no se ajusta a una distribución normal.
										\begin{equation}
										I_{ratio} = \dfrac{f_{t_{*}}}{f_{t}}
										\end{equation} 	

		\end{itemize}
Estas dos operaciones generan una imagen con \'indices de cambios $ I_{c} $ a partir de cada conjunto de datos multitemporal, dando lugar a tantos mapas de cambios como bandas/capas se consideren.
\subsection{Criterios de decisi\'on}
El resultado de los c\'alculos es una imagen en donde el valor de salida indica el grado de cambio, desde la mayor p\'erdida a la mayor ganancia, en una escala gradual. Si se pretende generar una imagen binaria (cambio/estable), es preciso se\~{n}alar un umbral que delimite ambas categor\'ias en las im\'agenes. Ah\'i se plantea un problema de dif\'icil soluci\'on ya que no existen criterios de aplicación general.\\~\\
Si el cambio abarca un sector importante de la imagen, el histograma de la imagen de cambios ($ I_{c} $) debiese mostrar un perfil bimodal, lo que permitir\'ia establecer umbrales naturales de cambio, aunque esta situaci\'on no es muy habitual, ya que los cambios en la naturaleza no suelen producirse de modo abrupto \cite{martinez2013normalizacion}.\\~\\
Si es necesario establecer un umbral para separar las \'areas de cambio, puede optarse por se\~{n}alar alg\'un criterio estad\'istico, como la media y la desviaci\'on t\'ipica de una serie de p\'ixeles elegidos aleatoriamente. En ocasiones se ha propuesto utilizar unas \'areas de entrenamiento para calcular que rango de desviaci\'on se pod\'ia considerar l\'imite para p\'ixeles estables, aplicando luego ese valor al conjunto de la imagen \cite{tung1988determination}.

\subsubsection{Discriminaci\'on de las zonas de cambio}\label{sec:discriminacion}
La comparaciones multitemporales generan indices que corresponden a variables cuantitativas, por lo que la aplicaci\'on de m\'etodos que discriminen las zonas, en tipos de cambios, permitir\'an un an\'alisis mas especifico sobre la imagen satelital.\\~\\ 
Sea $ B:I_{c} \longrightarrow \{0,1\}$ 	una funci\'on que binariza una imagen en base a un Umbral $ U $:
\begin{equation}\label{ec:umbralizacion}
B(I_{c}) = \begin{cases}
1 & \text{si se cumple que } I_{c} \geq U \\
0 & \text{en cualquier otro caso}
\end{cases}
\end{equation}

\nomenclature[51]{$ I_{c} $}{Indice de cambio entre dos im\'agenes.}
\nomenclature[52]{$ B $}{Imagen binaria en el proceso de detecci\'on de cambio.}
\nomenclature[53]{$ B(I_{c}) $}{Nivel digital para el $ I_{c} $.}
\nomenclature[54]{$ U $}{Umbral que binariza una imagen en base a un $ I_{c} $.}
La ecuaci\'on \ref{ec:umbralizacion} genera una m\'ascara binaria de cambios (0, No Cambio; 1, Cambio) aplicando un umbral ($ U $) especifico sobre la imagen resultante del proceso de comparaci\'on multitemporal \cite{singh1989review}. Son f\'acilmente implementables en procesos de car\'acter autom\'atico/semiautom\'atico. Partiendo de la hip\'otesis de que el porcentaje de cambios es muy reducido, los valores correspondientes se encuentran situados en los extremos del histograma de frecuencias \cite{estornell2004analisis}. Es preciso se\~{n}alar un umbral que delimite ambas categorías (cambio/no cambio) a partir del \'indice de cambios \cite{radke2005image} para generar una m\'ascara.\\~\\
El m\'etodo de discriminaci\'on basado en los par\'ametros estad\'isticos del \'indice de cambio entre la secuencia temporal de im\'agenes tiene la siguiente expresi\'on \cite{rodriguez2010analisis}:
\begin{equation}
U=\mu_{I_{c}} \pm n \times \sigma _{I_{c}}
\end{equation}
\nomenclature[55]{$ n $}{Coeficiente de fiabilidad de los datos.}
\nomenclature[56]{$ \sigma_{I_{c}} $}{Desviaci\'on t\'ipica de la imagen de \'Indices de cambio.}
\nomenclature[56]{$ \mu_{I_{c}} $}{Media de la imagen de \'Indices de cambio.}
Donde el valor de umbral entre cambio/no cambio $ (U) $ se estima en funci\'on a la media ($ \mu_{I_{c}}) $ y desviaci\'on ($ \sigma_{I_{c}} $) de la imagen de Indice de cambio ($ I_{c} $), junto con un coeficiente de tolerancia $ n $ asignado en base a la fiabilidad de los datos. Los resultados se clasifican en funci\'on de $ n $; alta probabilidad de cambio $ (n \geq 2) $ y
zonas de media probabilidad de cambio $ (1 < n < 2) $ \cite{estornell2004analisis}.

\subsection{Filtrado}
Los filtros constituyen unos de los principales m\'etodos del procesamiento digital de im\'agenes . Pueden usarse para distintos fines, pero siempre, el resultado sobre cada pixel depende de los pixeles en su entorno. Tiene como objetivos: 
	\begin{itemize}
		\item \textbf{Suavizar la imagen:} reducir las variaciones de intensidad entre p\'ixeles vecinos.
		\item \textbf{Eliminar ruido:}  modificar aquellos p\'ixeles cuyo nivel de intensidad es muy diferente al de sus vecinos.
		\item \textbf{Realzar la imagen:} aumentar las variaciones de intensidad, all\'i donde se producen.
		\item \textbf{Detectar bordes::} detectar aquellos p\'ixeles donde se produce un cambio brusco en la funci\'on intensidad.	
	\end{itemize}

\section{Resumen}

La utilizaci\'on de im\'agenes satelitales implica un pre-procesamiento adicional, diferente a las que se le realiza a im\'agenes normales. Estos pre-procesamientos van ligados a resoluciones, firmas espectrales y tipos de im\'agenes satelitales propias del sensor espacial que captura la informaci\'on. En este capitulo se describen conocimientos previos para aplicar an\'alisis multitemporales y detecci\'on de cambio en im\'agenes de sat\'elite que componen piezas fundamentales en la metodolog\'ia propuesta.
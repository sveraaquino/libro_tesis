 %%%%%%%%%%%%%%%%%%%%%%%%%%%%%%%%%%%%%%%%%
% Thesis 
% LaTeX Template
% Version 1.3 (21/12/12)
%
% This template has been downloaded from:
% http://www.latextemplates.com
%
% Original authors:
% Steven Gunn 
% http://users.ecs.soton.ac.uk/srg/softwaretools/document/templates/
% and
% Sunil Patel
% http://www.sunilpatel.co.uk/thesis-template/
%
% License:
% CC BY-NC-SA 3.0 (http://creativecommons.org/licenses/by-nc-sa/3.0/)
%
% Note:
% Make sure to edit document variables in the Thesis.cls file
%
%%%%%%%%%%%%%%%%%%%%%%%%%%%%%%%%%%%%%%%%%

%----------------------------------------------------------------------------------------
%	PACKAGES AND OTHER DOCUMENT CONFIGURATIONS
%----------------------------------------------------------------------------------------

%\documentclass[draft]{article} %no mostrar las imagenes%
\documentclass[11pt, a4paper, oneside]{Thesis} % Paper size, default font size and one-sided paper
\usepackage{multirow}
\usepackage{url}
\usepackage[spanish,es-tabla]{babel}
\providecommand{\abs}[1]{\lvert#1\rvert}
\usepackage[spanish]{babel}
\selectlanguage{spanish} 
\usepackage[spanish,onelanguage,ruled,linesnumbered]{algorithm2e}
\SetKwFor{For}{para}{hacer}{fin}
\usepackage[table,xcdraw]{xcolor}

\usepackage{titlesec}






\graphicspath{{./Pictures/}} % Specifies the directory where pictures are stored
\usepackage{listings}
\usepackage{graphicx}
\usepackage{float}

\usepackage{subfigure}
\usepackage{enumerate}
\usepackage{pdfpages}
\usepackage{verbatim} 
\usepackage{enumerate}

\usepackage{color}
\newcommand{\hilight}[1]{\colorbox{yellow}{#1}}
\usepackage{amsmath}
\usepackage[intoc]{nomencl}

\makenomenclature
\renewcommand{\nomname}{\'Indice de s\'imbolos}
\usepackage{program}
\setcounter{secnumdepth}{4}




%\renewcommand{\nomname}{Nomenclatura}
%------Bibliografía------------


%\usepackage{apacite} %Uno de los esfectos es elimina los números de la bibliografía.
\usepackage[square, numbers, comma, sort]{natbib} % Use the natbib reference package - read up on this to edit the reference style; if you want text (e.g. Smith et al., 2012) for the in-text references (instead of numbers), remove 'numbers' 
%\usepackage[square, numbers, comma, sort]{biblatex}
\usepackage[utf8]{inputenc}
\hypersetup{urlcolor=blue, colorlinks=true} % Colors hyperlinks in blue - change to black if annoying
\title{\ttitle} % Defines the thesis title - don't touch this
\usepackage{etoolbox}
\makeatletter
\patchcmd{\@makecaption}
  {\scshape}
  {}
  {}
  {}
\makeatletter
\patchcmd{\@makecaption}
  {\\}
  {.\ }
  {}
  {}
\makeatother
\def\tablename{Tabla}


\begin{document}
%\renewcommand\tablename{Tabla}
%\renewcommand{\figurename}{Figura}
\frontmatter % Use roman page numbering style (i, ii, iii, iv...) for the pre-content pages
\setstretch{1.3} % Line spacing of 1.3

% Define the page headers using the FancyHdr package and set up for one-sided printing
\fancyhead{} % Clears all page headers and footers
\rhead{\thepage} % Sets the right side header to show the page number
\lhead{} % Clears the left side page header

\pagestyle{fancy} % Finally, use the "fancy" page style to implement the FancyHdr headers

\newcommand{\HRule}{\rule{\linewidth}{0.5mm}} % New command to make the lines in the title page

% PDF meta-data
\hypersetup{pdftitle={\ttitle}}
\hypersetup{pdfsubject=\subjectname}
\hypersetup{pdfauthor=\authornames}
\hypersetup{pdfkeywords=\keywordnames}
%----------------------------------------------------------------------------------------
%	TITLE PAGE
%----------------------------------------------------------------------------------------

\begin{titlepage}
\begin{center}
%\includegraphics{Logo} % University/department logo - uncomment to place it
%\textsc{\LARGE Universidad Nacional de Asunci\'on}\\[1.5cm] % University name
%\textsc{\Large Facultad Politécnica}\\[0.5cm] 
%\textsc{\Large Facultad Politécnica}\\[0.5cm] 
\includegraphics[scale=1]{./Figures/logo.png} % University/department logo - uncomment to place it
%\textbf{\rmfamily \textsc{\LARGE Universidad Nacional de Asunci\'on} \\ \textsc{\LARGE Facultad Politécnica}}\\[1.5cm] % University name

\large \textit{INGENIERÍA EN INFORMÁTICA}\\[0.3cm]

\HRule \\[0.4cm] % Línea Horizontal
{\huge \bfseries Metodolog\'ia autom\'atica para estimar p\'erdida de
	carbono a trav\'es de procesamiento de im\'agenes
	satelitales. Caso de uso Chaco Paraguayo}\\[0.4cm] % Thesis title
\HRule \\[1.5cm] % Línea Horizontal
 
\large \textit{PROYECTO FINAL DE GRADO}\\[0.3cm]
 

 \vfill
 
\vfill

\vfill

\begin{minipage}{0.4\textwidth}
\begin{center} \large
\emph{Autor:}\\
{Santiago Smael Vera Aquino}
\end{center}
\end{minipage} 
%Using a minipage:
 \hfill\begin{minipage}{0.4\textwidth}
\begin{center} \large
\emph{Tutor:}\\ 
{Dr. Horacio Legal Ayala} \\

\end{center}
\end{minipage}\\[3cm]


 
\textsc{\LARGE San Lorenzo - Paraguay}\\[1.5cm]
\textsc{\LARGE Octubre - 2015}\\[1.5cm]
 
%{\large \today}\\[4cm] % Date
\vfill
\end{center}
\end{titlepage}





%----------------------------------------------------------------------------------------
%	DEDICATORIA
%----------------------------------------------------------------------------------------
\clearpage % Start a new page
\setstretch{1.3} % Return the line spacing back to 1.3
\pagestyle{empty} % Page style needs to be empty for this page
\dedicatory{

A mis familiares, profesores, compañeros y amigos por su apoyo, aliento y comprensión incondicional. 

} 
\addtocontents{toc}{\vspace{2em}} % Add a gap in the Contents, for aesthetics


%----------------------------------------------------------------------------------------
%	AGRADECIMIENTO
%----------------------------------------------------------------------------------------
\clearpage % Start a new page
\setstretch{1.3} % Reset the line-spacing to 1.3 for body text (if it has changed)
\acknowledgements{\addtocontents{toc}{\vspace{1em}} % Add a gap in the Contents, for aesthetics
A todos los que conocen


}


%----------------------------------------------------------------------------------------
%	RESUMEN
%----------------------------------------------------------------------------------------
\clearpage % Start a new page
\addtotoc{Resumen} % Add the "Abstract" page entry to the Contents
\resumen{\addtocontents{toc}{\vspace{1em}} % Add a gap in the Contents, for aesthetics



El cambio clim\'atico es un problema de carácter mundial, que engloba distintos factores ligadas a las actividades humanas. Los bosques constituyen un medio principal de conservaci\'on de carbono, donde la de-forestaci\'on y degradaci\'on contribuyen al GEI(Gases de efecto invernadero) como di\'oxido de carbono (CO2).En los \'ultimos 50 a\~{n}os la explotaci\'on Forestal y el cambio en el uso de la tierra, produjo la p\'erdida del 90\% de los bosques en la región Oriental del Paraguay. La degradaci\'on de los suelos, genera una progresiva desertificaci\'on, que atenta contra la biodiversidad eliminado sumideros de carbono. La regi\'on Occidental o Chaco Paraguayo es una region en la cual aplicar mecanismos de control en cuanto al uso ambiental implica un costo elevado. Por ello, mediante esta investigaci\'on se pretende elaborar una metodología practica y din\'amica que permita identificar focos de perdidas de carbono empleando el procesamiento de im\'agenes satelitales. Esto permitir\'a una vez identificado las posibles alertas, elaborar planes o estrategias para controles mas rigurosos y espec\'ificos que direccionen las pol\'iticas de acci\'on, con objeto de prevenci\'on.


%----------------------------------------------------------------------------------------
%	ABSTRACT
%----------------------------------------------------------------------------------------


\clearpage % Start a new page
\abstract{\addtocontents{toc}{\vspace{1em}} % Add a gap in the Contents, for aesthetics

xxxxx



%----------------------------------------------------------------------------------------
%	LIST OF CONTENTS/FIGURES/TABLES PAGES
%----------------------------------------------------------------------------------------
\clearpage % Start a new page
\pagestyle{fancy}
\lhead{\emph{Contenido}} % Set the left side page header to "Contents"


%--------------INICIO CONFIGURACION DE INDICE-----------------------

\setcounter{tocdepth}{5} %El numero es el nivel que mostrará en el índice
\tableofcontents % Write out the Table of Contents

%--------------FIN DE CONFIGURACION DE INDICE-----------------------

\lhead{\emph{Índice de figuras}} % Set the left side page header to "List of Figures"
\listoffigures % Write out the List of Figures

\clearpage % Start a new page
%\setstretch{1.5} % Set the line spacing to 1.5, this makes the following tables easier to read
\lhead{\emph{Lista de símbolos}} % Set the left side page header to "Abbreviations"

\printnomenclature

%\lhead{\emph{Lista de Tablas}} % Set the left side page header to "List of Tables"
\listoftables % Write out the List of Tables




%----------------------------------------------------------------------------------------
%	ABBREVIATIONS
%----------------------------------------------------------------------------------------
\clearpage % Start a new page
\setstretch{1.5} % Set the line spacing to 1.5, this makes the following tables easier to read
\lhead{\emph{Índice de abreviaciones}} % Set the left side page header to "Abbreviations"
\listofsymbols{ll} % Include a list of Abbreviations (a table of two columns)
{
\textbf{GEI} & \textit{Gases de Efecto Invernadero.}\\
\textbf{CO2} & \textit{Di\'oxido de carbono.}\\
\textbf{C} & \textit{Carbono.}\\
\textbf{SIG} & \textit{Sistemas de Informaci\'on Geogr\'aficas.}\\
\textbf{REDD+} & \textit{Reducción de GEI por la Deforestación y Degradación de bosques.}\\
\textbf{RMSE} & \textit{Error cuadr\'atico medio.}\\
\textbf{ParLu} & \textit{Paraguay Land Use.}\\
\textbf{WWF} & \textit{World Wildlife Fund.}\\
\textbf{ENPAB} & \textit{Estrategia nacional y plan de acción para la conservacion de la Biodiversidad.}\\
\textbf{VD} & \textit{Valor Digital.}\\
\textbf{FMAM} & \textit{Fondo para el Medio Ambiente Mundial.}\\
\textbf{PDD} & \textit{Programa de Peque\~{n}as Donaciones.}\\
\textbf{LiDAR} & \textit{Detecci\'on \'area de luz y medidas de rango.}\\
\textbf{NDVI} & \textit{\'Indice de vegetaci\'on diferencial normalizada.}\\
\textbf{UTM} & \textit{Universal Transverse Mercator.}\\
\textbf{RMS} & \textit{Root Mean Squared Error.}\\
\textbf{NASA} & \textit{National Aeronautics and Space Administration.}\\
\textbf{MSS} & \textit{Multi-spectral Scanner.}\\
\textbf{TM} & \textit{Thematic Mapper.}\\
\textbf{ETM+} & \textit{Enhanced Thematic Mapper Plus.}\\
\textbf{OLI} & \textit{Operational Land Imager.}\\
\textbf{TIRS} & \textit{Thermal Infrared Sensor.}\\
\textbf{VCF} & \textit{Vegetation Continuous Fields.}\\
\textbf{MODIS} & \textit{MODerate-resolution Imaging Spectroradiometer.}\\
\textbf{PFCP} & \textit{Paraguay Forest Change Product.}\\
\textbf{USGS} & \textit{United States Geological Survey.}\\
\textbf{L1T} & \textit{Level 1 Terrain Corrected.}\\
\textbf{GPL} & \textit{General Public License.}\\
\textbf{GA} & \textit{Global Acurrancy.}\\
\textbf{WRS-2} & \textit{Landsat Worldwide Reference System-2.}\\
\textbf{PFCP} & \textit{Forest Change Produc.}\\
\textbf{Km} & \textit{Kil\'ometros.}\\
\textbf{Has} & \textit{Hect\'areas.}\\
\textbf{GCP} & \textit{Global Control Points .}\\


 



 






}






%----------------------------------------------------------------------------------------
% CAPÍTULOS DE LA TESIS
%----------------------------------------------------------------------------------------
\mainmatter % Begin numeric (1,2,3...) page numbering
\pagestyle{fancy} % Return the page headers back to the "fancy" style
\newpage{\ } 
\thispagestyle{empty} 

\chapter{Introducción}
\lhead{Capítulo 1. \emph{Introducción}} % This is for the header on each page - perhaps a shortened title

La captura de carbono es un servicio ambiental proporcionado por los bosques los bosques. La captura ser\'a determinante para disminuir el calentamiento global y estabilizar el cambio clim\'atico producidos por el incremento en la atm\'osfera de los llamados Gases de Efecto Invernadero (GEI) \cite{marquezestimacion}. El di\'oxido de carbono (CO2) es el gas mas abundante, contribuyendo con un 76\% al GEI \cite{avila2001almacenamiento}, debido principalmente al cambio de paisajes de bosques tropicales maduros a paisajes agr\'icolas.\\~\\
Los bosques tropicales en condiciones naturales contienen m\'as carbono a\'ereo por unidad de superficie que cualquier otro tipo de cobertura terrestre \cite{gibbs2007monitoring}. Por esto, cuando los bosques se convierten a otros usos del suelo, ocurre una gran liberaci\'on neta de carbono a la atm\'osfera. El cambio en el uso del suelo es responsable del 15-20\% de las emisiones totales de gases de efecto invernadero \cite{peralta2013analisis}.\\~\\
Las transformaciones química de compuestos, que contienen carbono en los intercambios entre biosfera, atm\'osfera, hidrosfera y litosfera son conocidas como ciclo de carbono \cite{wikixxx}. La fotosíntesis de las plantas constituye un proceso fundamental en el ciclo, ya que permite separar  el CO2 en oxigeno que consumimos y carbono (C) en materia \'organica, actuando en forma de almacenes de C como biomasa en función a la composición flor\'istica, la edad y la densidad de cada estrato por comunidad vegetal por periodos prolongados \cite{acosta2003diseno}.\\~\\
El t\'ermino biomasa se refiere a toda la materia org\'anica que proviene de \'arboles, plantas y desechos de animales que pueden ser convertidos en energ\'ia. La biomasa forestal se define como la cantidad total de materia org\'anica a\'erea presente en los \'arboles incluyendo hojas, ramas, tronco principal y corteza \cite{garzuglia2003wood}.\\~\\
La teledetecci\'on o percepci\'on remota sin estar en un contacto f\'isico directo, permite el uso de informaciones provenientes de sensores instalados en plataformas espaciales, que complementados con sistemas de informaci\'on geogr\'aficas (SIG) permiten an\'alisis mas continuos y din\'amico. Estos sensores remotos captan la energ\'ia reflejada o radiada por la superficie, ya sea emitida por el sol (sensores pasivos) o por el mismo sensor (sensores activos), para ser transformadas a valores digitales (VD) como im\'agenes satelitales, de manera secuencial para cada espacio de la tierra, a intervalos regulares de tiempo.\\~\\
Las coberturas vegetales, a trav\'es de la im\'agenes prove\'idas por los sensores remotos, es posible calcular \'indices que var\'ian dentro de margenes, indicando el vigor de la vegetaci\'on o la densidad de la biomasa forestal. Estos indices junto con la comparaci\'on multitemporal hace posible identificar la evoluci\'on de coberturas vegetales en periodos de tiempos obteniendo resultados cualitativos y/o cuantitativos en espacio y tiempo \cite{martinez2013normalizacion}.\\~\\
En la actualidad existen diferentes m\'etodos para la detecci\'on de cambios de vegetaci\'on. Los m\'etodos requieren una supervisi\'on, un trabajo de campo y la utilizaci\'on de complejos sistemas de informaci\'on geogr\'afica. Estos sistemas de información geográfica son com\'unmente software de pago, elevando de esta manera el costo de dichos estudios. \\~\\
La falta de informaci\'on nos lleva a varios cuestionamientos referente a como estamos manejando nuestro medio ambiente y que efectos acarreara esos usos. El empleo de la teledetecci\'on y las im\'agenes satelitales multitemporales permiten realizar un an\'alisis a lo largo del tiempo de los cambios que el ambiente est\'a experimentando, mas aun en zonas como el Chaco Paraguayo, donde la informaci\'on ambiental es escaso por los altos costos y dificultades en el acceso al realizar controles en el terreno.\\~\\
Se propone dise\~{n}ar e implementar una metodolog\'ia autom\'atica que permita estimar la perdida de carbono a trav\'es de la biomasa, empleando procesamiento digital de im\'agenes satelitales, disponibles de forma libre, din\'amicos y no complejos.

\section{Justificación y Motivación}

REDD+ es una iniciativa que tiene como objetivo reducir la p\'erdida de bosques, teniendo como actividades principales \cite{peralta2013analisis}:
\begin{itemize}
	\item Evitar p\'erdidas como emisiones de gases de efecto invernadero (conservaci\'on, no deforestaci\'on, no degradaci\'on).
	\item Mantiener el dep\'osito o stock de carbono (conservaci\'on, gesti\'on sostenible).
	\item Incrementar el dep\'osito por su efecto de retenci\'on o sumidero de carbono (conservaci\'on, restauraci\'on, gesti\'on sostenible).
\end{itemize}

El Paraguay se ha embarcado en el proceso de preparaci\'on para reducir la deforestaci\'on y degradaci\'on forestal, a fin de disminuir las emisiones de CO2, conservar los bosques y su biodiversidad. Este proceso deriva la necesidad de elaborar una estrategia nacional, con pol\'iticas socios ambientales y econ\'omicos viables.\\~\\
Para medir los beneficios de carbono de un proyecto REDD+ es necesario calcular la cantidad de carbono almacenado en el bosque en cuesti\'on y luego predecir la cantidad de carbono que se podr\'ia conservar si se detiene o reduce la deforestaci\'on y la degradaci\'on forestal \cite{nellemann2009carbono}.\\~\\
Las grandes extensiones de las \'areas de estudio, la dificultad de acceder a las mismas, el alto costo del establecimiento de las parcelas de inventario y su limitada utilidad hacen que la mayoría de las investigaciones para estimar y mapear la biomasa en bosques se centren en las t\'ecnicas de Sensores Remotos. \\~\\
La necesidad de crear metodolog\'ias que ayuden al monitoreo de forma din\'amica y barata, lleva al desarrollo de herramientas libres que permitan estimar focos de alerta para la toma de acciones y controles m\'as rigurosos a tiempo.

\section{Antecedentes}\label{sec:antecedente}

El proyecto del Mapa global de carbono fue desarrollado por el Jet Propulsion Laboratory, California Institute of Technology en el a\~{n}o 2011 \cite{saatchi2011benchmark}, abarcando m\'as de 2.5 millones en hect\'areas de bosques, para tres continentes, trazando el stock total de carbono en la biomasa viva (por debajo y por encima).\\~\\
El estudio utilizo una combinaci\'on de datos con 4079 parcelas de inventario in situ e im\'agenes satelitales provenientes de sensores remotos LIDAR (Light Detection and Ranging o Laser Imaging Detection and Ranging). La combinaci\'on genera muestras de las estructura boscosa para estimar el almacenamiento de carbono y as\'i poder ser extrapolados en toda las superficie terrestre a trav\'es de im\'agenes \'opticas y microondas (resoluci\'on espacial de 1km).\\~\\
En la actualidad existen varios proyectos finales de grado, realizados por estudiantes de la Facultad de Ciencias Agrarias - UNA, que desarrollan metodologias de deteccion de cambio forestal y estimaciones de carbono. Los trabajos relacionados a estimaciones de carbono, implementan una metodolog\'ia base hecha en el marco denominado \textit{Desarrollo del estudio de linea de base para el sitio piloto Bosque atl\'antico de Alto Paran\'a. (BAAPA)} \cite{BAAPA2013} realizado por el Paraguay Land Use (ParLu). El ParLu es una iniciativa de World Wildlife Fund (WWF) Paraguay y WWF Alemania que apoya a las iniciativas REDD+ en Paraguay, enfoc\'andose principalmente a nivel local en comunidades del Bosque Atl\'antico y el Pantanal.\\~\\
Los productos generados por BAAPA, consisten en mapas de stock de carbono con sus correspondientes mapas de cobertura y deforestaci\'on 2000\textendash2005 y 2005\textendash2011. Estos productos fueron realizados a partir de muestreos en parcelas in situ y algoritmos de clasificaci\'on supervisadas, proporcionada por software SIG de pago. El estudio tambi\'en fue hecho conjuntamente con la  Carrera de Ingenier\'ia Forestal de la Facultad de Ciencias Agrarias perteneciente a la Universidad Nacional de Asunci\'on. Algunos proyectos finales de grado son citados a continuaci\'on:
\begin{itemize}
	\item Detecci\'on de cambios de la cobertura vegetal mediante indices de vegetaci\'on (NDVI), dentro y fuera de la Reserva de la biosfera del Chaco en el periodo 1985-2011 \cite{gustavo2012deteccion}.
	\item An\'alisis del cambio de cobertura de la tierra y estimaci\'on de carbono en el \'area para Parque Nacional San Rafael, a\~{n}o 2008/2013 \cite{peralta2013analisis}.
	\item Estimaci\'on de carbono almacenado en el Parque Nacional Defensores del Chaco seg\'un formaci\'on vegetal mediante im\'agenes satelitales, a\~{n}o 2014 \cite{kris2014estimacion}.
\end{itemize}

	Un estudio realizado por University of Maryland Institute for Advanced Computer Studies denominado Forest Cover Change in Paraguay, nos muestra el cambio de vegetaci\'on estimado en todo el pa\'is utilizando un m\'etodo iterativo de etiquetado de cambio por clusterizaci\'on supervisada \cite{huang2009assessment}. Este trabajo detecta cambios en los a\~{n}os 1990 al 2000, donde aparte de proveer un etiquetado de cambios de vegetaci\'on fue realizada con im\'agenes de acceso libre. Las validaciones fueron hechas con varias im\'agenes satelitales de alta resoluci\'on espacial(entre 4 y 0.5 metros), no libres, arrojando para todas las escenas precisiones globales mayor al 90\% y errores por comisi\'on y omisi\'on menores al 10\%.

\section{Planteamiento del problema}

Paraguay es un pa\'is que basa su econom\'ia en la agricultura y la ganader\'ia extensiva, actividades que han afectado al recurso forestal, dando como resultado extensas \'areas deforestadas y degradadas \cite{BAAPA2013}.\\~\\
	En el informe realizado por la ENPAB \cite{basualdo2003estrategia} se menciona que existe una fuerte presi\'on pol\'itica y social, proveniente de diversos grupos que buscan transformar las tierras del Chaco paraguayo en unidades econ\'omicas de producci\'on, cuyo enfoque gira en torno al crecimiento econ\'omico antes que al desarrollo sostenible. 
	En muchas zonas del chaco paraguayo, el modelo de desarrollo y uso de la tierra ha producido grandes extensiones de tierras altamente degradadas, arenales, desertificaci\'on y salinizaci\'on.\\~\\	
	A pesar que existen leyes de protecci\'on para evitar la deforestaci\'on y valorar los bosques, los mismos necesitan apoyo para su monitoreo y aplicaci\'on efectiva, debido a que los costos en tiempo y recursos son elevados.\\~\\
	Con el objetivo de implementar pol\'iticas de mitigaci\'on del cambio clim\'atico relativas a reducir las emisiones provenientes de la degradaci\'on y la deforestaci\'on (REDD+), los pa\'ises en desarrollo deben contar con estimaciones robustas s\'olidas en cuanto a las reservas de carbono forestal\cite{BAAPA2013}.

\section{Objetivos}
Atendiendo a la necesidad de metodolog\'ias alternativas para el monitoreo de perdida de carbono en el campo ambiental, los objetivos delineados son los siguientes.

\subsection{Objetivo General}

\begin{itemize}
\item Desarrollar una metodolog\'ia autom\'atica de an\'alisis de im\'agenes satelitales multitemporales para la generaci\'on de indicadores respecto a la perdida del contenido de carbono en zonas del Chaco Paraguayo.
\end{itemize}
\subsection{Objetivos Específicos}
Para el logro de los objetivos generales los siguientes objetivos específicos son propuestos:
\begin{itemize}


\item Realizar detecciones de cambio automatizada dentro del \'area de estudio a trav\'es de la Teledetecci\'on y un SIG.   

\item Desarrollar normalizaciones de im\'agenes para la comparación multi-temporal. 
\item Determinar la relación entre la biomasa y el carbono a trav\'es de muestreos.
    
%\item Implementaci\'on de la metodolog\'ia como complemento de una herramienta SIG de c\'odigo abierto.


\end{itemize}



\section{Organización de la Tesis}

%La distribución de capítulos del presente trabajo final de grado se encuentra distribuido en 6 capítulos.
%en este capítulo se da una breve introducción al tema, se describe el problema de manera precisa para lograr su mejor entendimiento, también se citan los objetivos trazados tanto específicos como generales finalmente se habla de los antecedentes.
%en el capítulo 2  se realiza una breve introducción sobre la diabetes y los tipos existentes, 
%\begin{comment} luego hablaremos de la enfermedad de los ojos que se da en las personas con diabetes que es conocida como  retinopatía diabética, de la misma se menciona las causas, factores de riesgos y los síntomas. Además se menciona las etapas en cuales la enfermedad se va desarrollando, las anormalidades que se van dando dentro de los ojos, así como también los tratamientos usados para combatir esta enfermedad: fotocoagulación con láser, terapia médica intravítrea y tratamiento quirúrgico. 
%\end{comment} 
%en el capítulo 3  se presenta el marco teórico de las técnicas de  visión por computadora, %\begin{comment} desde los espacios de colores utilizados para representar las imágenes de fondo de ojo hasta los algoritmos se mencionan con detalle el funcionamiento del algoritmo de normalización utilizado,  en los procesos de segmentación, extracción de características y clasificación.
%\end{comment} 
%en el capítulo 4  se detallan  los algoritmos de detección y segmentación utilizados en el sistema de diagnóstico, además  se menciona las sub-segmentaciones utilizadas en estos procesos,
%Luego tenemos la extracción de características cuya importancia radica en el hecho de que reduce la cantidad de datos a procesar. Al final de la metodología tenemos al clasificador de máquina de vector de soporte el cual dará el diagnóstico final. 
%en el capítulo 5  se presenta las métricas  utilizadas para medir el desempeño, luego se evalúa los resultados obtenidos y se realiza  la comparación con respecto al estado del arte y por último en el capítulo 6 se presentan las conclusiones finales tras los experimentos y análisis de resultados del proyecto, por último los trabajos futuros que podrían dar continuidad al trabajo final de grado.


La distribución de capítulos del presente trabajo final de grado se encuentra organizado en 6 capítulos.
%en este capítulo se da una breve introducción al tema, se describe el problema de manera precisa para lograr su mejor entendimiento, también se citan los objetivos trazados tanto específicos como generales finalmente se habla de los antecedentes.
\begin{itemize}

\item En el cap\'itulo 2  se describir\'an los conceptos generales relacionados al cambio clim\'atico y perdida de carbono.

\item En el cap\'itulo 3 se pretende dar un marco te\'orico acerca del procesamiento digital de im\'agenes satelitales.
%\end{comment} 
\item En el cap\'itulo 4  se detalla los algoritmos y procedimientos empleados en la metodolog\'ia de estimaci\'on de perdida de carbono.

\item En el cap\'itulo 5  se presenta las m\'etricas para medir la calidad de los resultados. Tambi\'en se evalu\'a los resultados en base a las m\'etricas previstas.

\item En el cap\'itulo 6 se presentan las conclusiones finales tras los experimentos y an\'alisis de resultados del proyecto, concluyendo con propuestas de trabajos futuros para dar continuidad al trabajo final de grado.

\end{itemize}

\newpage{\ } 
\thispagestyle{empty} 

\chapter{Cambio Clim\'atico}
\lhead{Cap\'itulo 2. \emph{Cambio Clim\'atico}} % This is for the header on each page - perhaps a shortened title


El cambio clim\'atico es definido como cualquier variaci\'on del clima a lo largo del tiempo, ya sea por variabilidad natural o como resultado de las actividades humanas que altera la composici\'on de la atm\'osfera y que se suma a la variabilidad clim\'atica natural observada en periodos de tiempos comparables \cite{robert2002captura}.\\~\\
En la Figura \ref{fig:cambioClimatico} podemos observar como el planeta tierra esta cubierta por una capa de gases que deja penetrar energ\'ia solar que calienta la superficie terrestre. Algunos de los gases en la atm\'osfera, llamados gases de efecto invernadero (GEI), impiden el escape de este calor hacia el espacio . El escape de calor  mantiene a la tierra a una temperatura promedio arriba del punto de congelaci\'on del agua y permite la vida. A pesar de esto, las actividades humanas est\'an produciendo un exceso de gases que est\'an potencialmente calentando el clima de la tierra \cite{almando2014estimacion}.
    \begin{figure}[!hbtp]
    	\centering
    	\includegraphics[width=0.5	\textwidth]{./Figures/cap2/calentamientoGlobal.jpg}
    	\caption{Calentamiento Global \cite{calent2015global}.}
    	\label{fig:cambioClimatico}
    \end{figure}


\section{Ciclo de carbono} 
Las plantas absorben el di\'oxido de carbono existente en el aire o el agua, acumul\'andolos en los tejidos vegetales en forma de materia org\'anica, mediante la fotos\'intesis \cite{natur2015PW}. Posteriormente, los animales herb\'ivoros se alimentan de estos vegetales para transferir esa energ\'ia a los dem\'as niveles (carn\'ivoros que se alimentan de los herb\'ivoros).
La energ\'ia transferida sigue varios caminos: por un lado es devuelto a la atm\'osfera como di\'oxido de carbono mediante la respiraci\'on; por otro lado se deriva hacia el medio acu\'atico, donde puede quedar como sedimentos org\'anicos, o combinarse con las aguas para producir carbonatos y bicarbonatos (suponen el 71\% de los recursos de carbono de la Tierra). La acumulaci\'on de carbono en zonas h\'umedas genera turba (carb\'on ligero y esponjoso), resultado de una descomposici\'on incompleta, lo que da lugar a la formaci\'on de dep\'ositos de combustibles f\'osiles como petr\'oleo, carb\'on y gas natural.\\~\\
El ciclo del carbono queda completado gracias a los organismos des\-componedores, los cuales llevan a cabo el proceso de mineralizar y descomponer los restos org\'anicos, cad\'averes, excrementos, entre otros. Adem\'as de la actividad que llevan a cabo el reino vegetal y animal en el ciclo, tambi\'en liberan carbono la putrefacci\'on y la combusti\'on \cite{natur2015PW}. La Figura \ref{fig:ciclocarbono} nos presenta el ciclo completo del carbono.
    \begin{figure}[!hbtp]
    	\centering
    	\includegraphics[width=1.0	\textwidth]{./Figures/captura_de_carbono.jpg}
    	\caption{Ciclo de carbono \cite{ciclot2015carbo}.}
    	\label{fig:ciclocarbono}
    \end{figure}


\subsection{Secuestro de carbono}
El CO2 y otros gases invernaderos act\'uan atrapando la energ\'ia cal\'orica (radiaci\'on solar de onda corta) reflejada de la superficie de la tierra y las nubes \cite{encaptura}. Este calor retenido puede conducir al calentamiento global en el planeta. Los niveles del di\'oxido de carbono atmosf\'erico pueden reducirse en la misma medida que los niveles de carbono org\'anico del suelo aumentan a trav\'es del secuestro de carbono. Si el carbono org\'anico del suelo no es alterado, puede permanecer en el suelo por muchos a\~{n}os como materia org\'anica estable. Este carbono es entonces secuestrado o removido de la atm\'osfera para ser reciclado. De esta forma se pueden reducir los niveles de CO2, disminuyendo las probabilidades de calentamiento global \cite{castillo2003manejo}.
\subsection{P\'erdida de Carbono}
P\'erdida de carbono se refiere a aquella porci\'on de carbono que no pudo ser almacenada o capturada en el intercambio normal que ocurre entre la superficie terrestre y la atm\'osfera en el ciclo de carbono \cite{marquezestimacion}, contribuyendo al calentamiento global mediante la emisi\'on de di\'oxido de carbono que compone el grupo de gases de efectos invernaderos.
\subsection{Secuestro de carbono en Paraguay}
El Fondo para el Medio Ambiente Mundial (FMAM) y el Programa de Peque\~{n}as Donaciones (PDD), en nuestro pa\'is, nos dice que el uso de hidrocarburos para generar energ\'ia el\'ectrica, el uso de biomasa como fuente de energ\'ia, las emisiones industriales, la deforestaci\'on, los incendios forestal, la actividad pecuaria, el manejo y disposici\'on de residuos y la actividad del transporte son los que presentan mayores emisiones de carbono \cite{cecilia2010Proyecto}, en consecuencia contribuyen al cambio clim\'atico.
\subsection{Gran Chaco Americano}
En el territorio del Gran Chaco Americano, se detecta una tendencia de importante aumento de las tasas de deforestaci\'on diaria por encima de las 1.400 hect\'areas, siendo el promedio del per\'iodo 15 de junio al 10 de julio de 2.011, de 1.042 hect\'areas por d\'ia, y del per\'iodo 10 de julio al 13 de agosto de 2.011 de 1.408 hect\'areas por d\'ia en toda la regi\'on, dando un total de 47.856 hect\'areas de \'areas boscosas que registraron cambio a uso agropecuario, en 34 d\'ias. Entre los pa\'ises que componen el Gran Chaco Americano,  Paraguay  registr\'o el mayor porcentaje de la deforestaci\'on (86\%), seguido por Argentina (13\%) y Bolivia (1\%). En Brasil, no se detectaron caso de deforestaci\'on para la regi\'on. En el caso espec\'ifico de Paraguay, la tasa de deforestaci\'on diaria ha aumentado, pasando de 998 hect\'areas por d\'ia a 1.210 hect\'areas por d\'ia \cite{fao2003revista}, perdi\'endose por consiguiente en gran medida sumideros de carbono, lo cual va aportando al desequilibrio del ciclo. En la Tabla \ref{tab:chacoamericano} podemos observar los principales problemas ambientales que afronta el Gran Chaco Americano, en cada pa\'is que lo compone.
\begin{table}[!hbtp]
	\centering
	\caption{Problem\'atica que afrontan los pa\'ises del gran chaco americano \cite{gustavo2012deteccion}.}
	\label{tab:chacoamericano}
	\begin{tabular}{|p{4cm}|p{4cm}|p{4cm}|}
		\hline
		{\bf Argentina} & {\bf Bolivia} & {\bf Paraguay} \\ \hline
		Deforestaci\'on de los bosques nativos. & Deforestaci\'on de los bosques nativos. & Deforestaci\'on de los bosques nativos. \\ \hline
		Excesiva dependencia dela producci\'on ganadera y explotaci\'on forestal. & Sobrepastoreo. & Sobrepastoreo. \\ \hline
		Sobrepastoreo. & Incendios de bosques y pastizales. & Incendios de bosques y pastizales. \\ \hline
		Incendios de bosques y pastizales. & P\'erdida de biodiversidad. & Manejo no sustentable de los recursos h\'idricos. \\ \hline
		Perdida de labiodiversidad. & Cambio clim\'atico. & P\'erdida de biodiversidad. \\ \hline
		Cambio clim\'atico. &  & Cambio clim\'atico. \\ \hline
	\end{tabular}
\end{table}


\section{Biomasa}
La biomasa es aquel material org\'anico biodegradable y no fosilizado originado de plantas, animales y microorganismos; incluye productos, subproductos, residuos y desechos de la agricultura, forester\'ia e industrias afines, as\'i como las fracciones org\'anicas y no fosilizadas de los desechos industriales y municipales. La biomasa tambi\'en incluye los gases y l\'iquidos recuperados de la descomposici\'on de materiales org\'anicos biodegradables y no fosilizados \cite{salinas2008guia}.
La biomasa es considerada como la masa total de organismos vivos en una zona o volumen determinado (a menudo tambi\'en se incluyen los restos de plantas que han muerto recientemente). La cantidad de biomasa se expresa mediante su peso en seco o su contenido de energ\'ia de carbono o de nitr\'ogeno \cite{garciduenas1987produccion}.
\subsection{Biomasa Forestal}
La biomasa forestal se define como el peso (o estimaci\'on equivalente) de materia org\'anica que
se encuentra en un determinado ecosistema forestal por encima y por debajo del suelo \cite{schlegel2000manual}, normalmente es
cuantificada en toneladas por hect\'area de peso verde o seco. La biomasa forestal es frecuentemente separada en
componentes, donde los m\'as t\'ipicos corresponden a la masa del fuste, ramas, hojas, corteza,
ra\'ices, hojarasca y madera muerta. \\~\\
En t\'erminos de p\'erdida y secuestro, representa la cantidad potencial de carbono que puede ser liberada a la atm\'osfera, debida a la deforestaci\'on, o la conservada en superficies terrestres cuando los bosques son correctamente gestionados \cite{lu2005exploring}.

\section{Medici\'on de balances de carbono}
La din\'amica del balance de carbono en un ecosistema forestal es muy compleja de medir, ya que es necesario determinar la captura de carbono por crecimiento de biomasa en los \'arboles y otros componentes en la vegetaci\'on como las p\'erdidas ocasionadas por disturbios, sean naturales o por actividades humanas; descomposici\'on de madera muerta; y la transferencia entre los compartimentos vivos, muertos y el suelo \cite{angelsen2008moving}.\\~\\
Existen metodolog\'ias que permiten medir y monitorear cambios en reservorios promedios de carbono por unidad de \'area. A continuaci\'on se citan algunos de ellos:

\begin{itemize}
	
	\item \textbf{Inventarios forestales:} se establecen relaciones alom\'etricas con mediciones de terreno en funci\'on al di\'ametro o volumen de arboles con las reservas de carbono forestal. La desventaja que presenta es su lentitud al realizar en \'areas grandes y costo elevado que presenta \cite{asner2005selective}. Definiendo alom\'etria como los cambios de dimensi\'on relativa de las partes corporales correlacionados con los cambios en el tama\~{n}o total. 
	\item \textbf{Sensores remotos:} existen diferentes tipos de sensores que permiten monitorear cambios en reservorios de carbono vegetal con mayor dinamismo y a gran escala \cite{libro2012Tsuyuki}. Podemos citar:
	\begin{enumerate}
	\item \textbf{Sensores remotos \'opticos (pasivos):} capturan luz solar o artificial reflejada desde el objeto, detectando la intensidad de luz visible e infrarroja en una o mas longitudes de ondas.
	\item  \textbf{Sensores remotos activos:} este sensor se encuentra montado en un sat\'elite, el cual emite pulsos de microondas oblicuamente detectando y registrando la intensidad, fase y tiempo de los impulsos reflejados desde la superficie terrestre.
	\item  \textbf{Sensores remotos l\'aser como LiDAR (detecci\'on \'area de luz y medidas de rango):} mide la distancia entre el sensor y el objeto usando el tiempo que tarda el pulso en viajar y la intensidad del pulso reflejado del objeto.
	\end{enumerate}
	La Figura \ref{fig:sensores} nos muestra como la informaci\'on es capturada, por medio de los 3 tipos de sensores descriptos.  
	    \begin{figure}[!hbtp]
	    	\centering
	    	\includegraphics[width=0.9	\textwidth]{./Figures/sensores.png}
	    	\caption{Tipos de sensores.}
	    	\label{fig:sensores}
	    \end{figure}
\end{itemize}

\section{Teledetecci\'on en el medio ambiente}
El t\'ermino teledetecci\'on esta definida como la ciencia y arte de obtener informaci\'on referente a la superficie terrestre sin entrar en contacto con ella. Esto se realiza detectando y grabando la energ\'ia emitida o reflejada para su procesamiento, an\'alisis y aplicaci\'on de esa informaci\'on \cite{salinero2002teledeteccion}.\\~\\
En los \'ultimos a\~{n}os se han desarrollado bastantes aplicaciones en casi todas las \'areas que involucra la tierra, debido a las grandes posibilidades y ventajas que presenta con la localizaci\'on de espacios geogr\'aficos, observaci\'on de fen\'omenos temporales e integraci\'on de resultados a los sistemas de informaci\'on geogr\'afica, reduciendo los costos en dinero y tiempo empleados en estudios sobre el terreno \cite{baker2006mapping}. La aplicaci\'on de la teledetecci\'on en los recursos naturales se fundamenta en que los elementos del mismo tienen un respuesta espectral propia a los sensores remotos. Por ello, la teledetecci\'on espacial es empleada como complemento y no como sustituto a estudios ambientales por permitir realizarlos a escalas espaciales y temporales distintas a las que se acceden desde experimentos controlados, lo cuales son tambi\'en necesarios e imprescindibles pero a veces insuficientes \cite{perez2011aplicaciones}.\\~\\
En la Figura \ref{fig:tele} podemos observar el proceso completo de la teledetecci\'on. El sensor remoto montado en el satelite captura la informaci\'on terrestre por medio de la energ\'ia solar. En una estaci\'on de recepci\'on, la informaci\'on capturada es transformada a im\'agenes para poder ser procesadas por el hombre en alg\'un tipo de an\'alisis. 

	\begin{figure}[H]
		\centering
		\includegraphics[width=0.7	\textwidth]{./Figures/cap3/teledeteccion.png}
		\caption{Teledetecci\'on \cite{teledet2015perce}.}
		\label{fig:tele}
	\end{figure}

\section{Resumen}

La p\'erdida de los bosques provocan da\~{n}os con consecuencias exponenciales al medio ambiente, ya que representan un factor fundamental en la estabilidad clim\'atica de la tierra. Las tierra esta cubierta por gases que dejan penetrar la energ\'ia solar, manteniendo temperaturas \'optimas para la vida. Algunos de los gases impiden el escape del calor hacia el espacio, a estos gases se los llaman gases de Efecto Invernadero (GEI). Las actividades humanas producen en exceso los GEI, principalmente con di\'oxido de carbono (CO2) a trav\'es de la deforestaci\'on y degradaci\'on en los bosques. La fotos\'intesis compone un elemento fundamental en el proceso natural denominado ciclo del carbono, mitigando el CO2 de la atm\'osfera con transformaciones del gas a materia org\'anica en las plantas (biomasa), a esto se lo conoce como secuestro de carbono. El chaco paraguayo presenta una tendencia importante en el aumento de las tasas de deforestaci\'on, siendo entre los pa\'ises que componen el Gran chaco americano el que presenta mayor porcentaje (86\%). 
En el 2011, el chaco paraguayo registro un promedio diario de 1402 hect\'areas de bosques deforestados a causa de actividades humanas ligadas a la agricultura, silvicultura y ganader\'ia, por ello la medici\'on del balance de carbono en nuestro pa\'is resulta importante para controles del manejo del medio ambiente. Los inventarios forestales establecen relaciones entre reservas de carbono forestal y variables alom\'etricas de los arboles  para medir el contenido de carbono, presentando como principal dificultad la lentitud en el estudio de \'areas extensas. En cambio, el empleo de procesamiento digital de im\'agenes satelitales junto con la teledetecci\'on nos brindan un dinamismo en el monitoreo a gran escala.

\newpage{\ } 
\thispagestyle{empty} 

\chapter{Procesamiento de im\'agenes satelitales}
\lhead{Capítulo 3. \emph{Procesamiento de im\'agenes satelitales}} % This is for the header on each page - perhaps a shortened title
La teledetecci\'on presenta un principio base similar al de la visi\'on, permitiendo mediante una fuente de energ\'ia, un objetivo o escena y un sensor, generar im\'agenes digitales que posibilitan resaltar aquellos elementos dif\'iciles de percibir o ser distinguidos directamente a trav\'es de una imagen normal. El comportamiento caracter\'istico que poseen los recursos naturales a sensores remotos, nos posibilita el empleo amplio de t\'ecnicas de procesamiento de im\'agenes provechosos para el logro de los objetivos en la investigaci\'on \cite{deespectro}. \\~\\
Este capitulo consiste en brindar conceptos espec\'ificos utilizados por la metodolog\'ia, posibilitando comprender la influencia de cada factor en el empleo de im\'agenes satelitales para la estimaci\'on de p\'erdida del contenido de carbono forestal.

\section{Sensores Remotos}
Los sensores remotos nos permiten obtener informaci\'on de la superficie terrestre, soportados en diferentes plataformas (terrestre, a\'erea y sat\'elite), mediante la captura de energ\'ias reflejadas o radiadas proveniente del sol (sensores pasivos) o del mismo sensor (sensores activos) \cite{gustavo2012deteccion}. La energ\'ia capturada es transformada en productos, con diversos y diferentes especificaciones, siendo las fotografi\'as \'areas e im\'agenes de sat\'elites las m\'as conocidos.
\subsection{El espectro electromagn\'etico}
Las longitudes de ondas son continuas, pero de igual modo se establecen un serie de bandas donde las radiaciones manifiestan un comportamiento similar, organizandolas de este modo en un espectro electromagn\'etico \cite{remote2010abdulrahman}.
Las bandas m\'as empleadas son las siguientes \cite{salinero2002teledeteccion}:
	\begin{itemize}
		\item \textbf{Espectro visible:} (400 nm a 700 nm) se denomina as\'i por tratarse de la \'unica radiaci\'on electromagn\'etica que pueden percibir nuestros ojos, coincidiendo con las longitudes de onda en donde es m\'axima la radiaci\'on solar. Dentro de esta se distinguen tres bandas fundamentales: Azul (400 nm a 500 nm), verde (500 nm a 600 nm) y rojo (600 nm a 700 nm).
		\item \textbf{Infrarrojo pr\'oximo:} (700 nm a 1300 nm) se utiliza para discriminar masas vegetales y concentraciones de humedad.
		\item \textbf{Infrarrojo medio:} (1,3 um a 8 um) en esta franja se entremezclan los procesos de reflexi\'on de la luz solar y de emisi\'on de la superficie terrestre. El infrarrojo medio es muy utilizado para estimar el contenido de humedad en la vegetaci\'on y los focos de alta temperatura.
		\item \textbf{Infrarrojo lejano o térmico:} (8 um a 14 um) se detecta el calor de la mayor\'ia de las cubiertas terrestres.
		\item \textbf{Microondas:} (a partir de 1 um) de gran inter\'es por ser un tipo de energ\'ia transparente a la cubierta nubosa.
	\end{itemize}
	En la Figura \ref{fig:bandasIs} podemos observar como el sensor montado en una plataforma espacial capta la informaci\'on terrestre en diferentes bandas de acuerdo a la longitud de onda.
	\begin{figure}[H]
		\centering
		\includegraphics[width=0.7	\textwidth]{./Figures/cap3/bandas_imagen.png}
		\caption{Bandas capturadas por un sat\'elite \cite{teledet2015perce}.}
		\label{fig:bandasIs}
	\end{figure}

\subsection{Firmas espectrales}
Las firmas espectrales consisten en la representaci\'on de energ\'ia reflejada con relaci\'on a las longitudes de ondas, consideradas sin el efecto atmosf\'erico y medida en condiciones ideales del \'angulo incidente. Las firmas espectrales ayudan a identificar los objetos en la superficie terrestre debido a que cada uno presenta una respuesta espectral \'unica \cite{sivakumar2004satellite}.\\~\\
En la Figura \ref{fig:firmaEspectral} se observa como cada objeto difiere de los dem\'as en sus firmas espectrales.

\begin{figure}[H]
	\centering
	\includegraphics[width=0.9	\textwidth]{./Figures/cap3/firmaEspectral.jpg}
	\caption{Firmas espectrales de diferentes coberturas.}
	\label{fig:firmaEspectral}
\end{figure}

\subsection{Resoluciones de un sensor}

La resoluci\'on de un sensor se define como el menor cambio en la magnitud de entrada que puede ser apreciada en la magnitud de salida. El concepto de resoluci\'on implica al menos cuatro manifestaciones \cite{peralta2013analisis}: 
	\begin{itemize}
		
		\item \textbf{Resoluci\'on espacial:} es el tama\~{n}o que representa en el terreno una unidad de pixel de la imagen. Esta resoluci\'on tiene mucha importancia en la interpretaci\'on pues marca el nivel de detalle que ofrece. En la Figura \ref{fig:espatialRes} podemos observar que cuanto menor sea el tama\~{n}o del pixel, menor ser\'a tambi\'en la probabilidad de que corresponda a un compuesto de dos o m\'as \'areas fronterizas.
		\begin{figure}[H]
			\centering
			\includegraphics[width=0.4	\textwidth]{./Figures/cap3/resolucion_espacial_n.jpg}
			\caption{Resoluci\'on espacial \cite{chara2015sate}.}
			\label{fig:espatialRes}
		\end{figure}
			\item \textbf{Resoluci\'on espectral:} indica el n\'umero y anchura de las bandas espectrales que puede discriminar el sensor. Un sensor ser\'a tanto m\'as id\'oneo cuanto mayor n\'umero de bandas proporcione, ya que facilita la caracterizaci\'on espectral de las distintas cubiertas. En la Figura \ref{fig:espectralRes} se puede observar la comparaci\'on entre la resoluci\'on espectral de dos diferentes sensores espaciales.
				\begin{figure}[H]
					\centering
					\includegraphics[width=0.6	\textwidth]{./Figures/cap3/espectral_spot_landsat.png}
					\caption{Resoluci\'on espectral igual a 3 para el sensor SPOT y 7 en el sensor Landsat \cite{martinez2005percepcion}.}
					\label{fig:espectralRes}
				\end{figure}
		\item \textbf{Resoluci\'on radiom\'etrica:} es la sensibilidad del sensor para detectar variaciones en la cantidad de energ\'ia espectral recibida. La sensibilidad se expresa en bits e indica el n\'umero de los distintos niveles radiom\'etricos que puede detectar un sensor. En la Figura \ref{fig:radioRes} se puede observar diferentes resoluciones radiom\'etricas.
		\nomenclature[10]{$ r $}{Bits o Niveles radiom\'etricos de la imagen satelital.}	
						\begin{figure}[H]
							\centering
							\includegraphics[width=0.6	\textwidth]{./Figures/cap3/radiometrica_bits}
							\caption{Diferentes resoluci\'ones radiom\'etricas en im\'agenes satelitales .}
							\label{fig:radioRes}
						\end{figure}
		\item \textbf{Resoluci\'on temporal:} Este tipo de resoluci\'on se refiere al intervalo de tiempo entre muestras sucesivas de la misma zona de la cobertura terrestre. El ciclo de cobertura presentada por la Figura \ref{fig:temporaRes}, est\'a en funci\'on de las caracter\'isticas orbitales de la plataforma, su velocidad, el ancho de barrido del sensor y las caracter\'isticas de construcci\'on del sistema.
			\begin{figure}[H]
					\centering
					\includegraphics[width=0.6	\textwidth]{./Figures/cap3/resolucion_temporal_land.png}
					\caption{Resoluci\'on temporal de 16 d\'ias \cite{teledet2015Combi}.}
					\label{fig:temporaRes}
				\end{figure}
	\end{itemize}

\section{Im\'agenes satelitales}

Una imagen satelital es una funci\'on $ f:(x,y,i) \longrightarrow \{0,...,2^{r}\} $. Cada $ (x,y,i) $ indica la posición $ (x,y) $ en a banda $ i $, donde $ i \in \{1,...,k\} $, $ x \in \{0,...,m\} $ e $ y \in \{0,...,n\} $ para una matriz $ m \times n $, siendo $ k $ el numero de bandas y $ r $ la resoluci\'on radiom\'etrica en la imagen. Las im\'agenes satelitales son conocidos tambi\'en como raster \cite{vasquez2011mineria} y se puede representar de forma matricial. La Figura \ref{fig:imagenMultiespectral} nos muestra los ejes de coordenadas espaciales $ (x,y) $ para cada plano que representan las bandas, pudiendo acceder a valor de la intensidad o nivel digital mediante el nivel digital  $ f(x,y,i) $.
  \begin{figure}[H]
  	\centering
  	\includegraphics[width=0.8	\textwidth]{./Figures/cap3/imagen_satelital_k4.png}
  	\caption{Valor digital en una imagen satelital de 4 bandas $ (k=4) $ y resoluci\'on radiometrica $ r=8 $.}
  	\label{fig:imagenMultiespectral}
  \end{figure}
  	\nomenclature[12]{$i$}{ Banda de la imagen satelital.}
  	\nomenclature[13]{$f$}{ Imagen satelital.}
  	\nomenclature[13]{$(x,y)$}{ Coordenadas espaciales.}
  	\nomenclature[16]{$(x,y,i)$}{Coordenadas espaciales de la banda $ i $ en la imagen satelital.}
  	\nomenclature[17]{$k$}{ N\'umero de bandas que posee una imagen satelital.}
  	\nomenclature[18]{$f(x,y,i)$}{ Nivel digital, en la posici\'on $ (x,y) $ de la banda $ i $, de la imagen satelital.}



\subsection{Histogramas}
El histograma de una imagen satelital, con niveles digitales en el rango de $ [0,2^{r}] $, es una funci\'on discreta $ H(ND)=n_{ND} $, donde $ ND $ es el nivel digital y $ n_{ND} $ el n\'umero de pixeles en la imagen teniendo el nivel digital $ ND $ \cite{gonzalez2002woods}.\\~\\
  	\nomenclature[19]{$ND$}{ Nivel digital de una imagen satelital.}
  	\nomenclature[20]{$H(ND)$}{ Funci\'on discreta que determina la cantidad de apariciones de $ ND $, en la imagen satelital.}  	
  	\nomenclature[21]{$n_{ND}$}{ N\'umero de pixeles en la imagen satelital teniendo el nivel digital $ ND $.}  	  	
 	\begin{figure}[H]
 		\centering
 		\includegraphics[width=0.7	\textwidth]{./Figures/cap3/histograma_def.png}
 		\caption{Histograma de una imagen con niveles digitales de $ 0 $ a $ 255 $.}
 		\label{fig:histDef}
 	\end{figure}
El histograma de una una imagen satelital es una representaci\'on gr\'afica \'util de la informaci\'on contenida por las im\'agenes obtenidas a trav\'es de la percepci\'on remota. En la Figura \ref{fig:histDef} podemos observar que para cada nivel digital va asociado un n\'umero de apariciones en la imagen.\\~\\
 Los analistas a menudo despliegan el histograma en cada banda, ya que proporciona una apreciaci\'on de la calidad de los datos que presenta una imagen. Por ejemplo si el contraste es bajo o muy alto (histogramas estrechos y amplios); si son multimodales responden a distintos tipos de coberturas detectadas (agua, humedales, tipos de vegetaci\'on, etc.), si en  el histograma de la banda infrarroja cercana se encuentran picos desplazados hacia la derecha implicar\'ia que existe una alta probabilidad de aparici\'on vegetal en la imagen, entre otros an\'alisis.

\subsection{Combinaci\'on de bandas}
La visualizaci\'on de las im\'agenes de teledetecci\'on es mejor cuando se tiene una representaci\'on en colores, ya que el ojo humano percibe mejor las diferencias de color que los niveles de gris.\\~\\
Las combinaci\'on de tres bandas a color en las im\'agenes satelitales recibe el nombre de imagen de color compuesta \cite{com2015color}. Las im\'agenes de las distintas bandas se pueden combinar entre ellas para producir una imagen en color real o en falso color en funci\'on de las bandas escogidas. Esto se hace asignando a cada uno de los canales (RGB) de la pantalla de ordenador, una banda en particular.\\~\\
Las im\'agenes compuestas en color natural o real son combinaciones de las bandas 1 (azul) , 2 (verde) y 3 (rojo) que coinciden aproximadamente con la gama visual del ojo humano, por lo que se parecen bastante a lo que esperar\'iamos ver en una fotograf\'ia normal en color. Las im\'agenes de color real tienden a presentar un bajo contraste y un aspecto algo borroso. Ello es debido a que la luz azul es m\'as afectada que las dem\'as por la dispersión atmosf\'erica.\\~\\
Otras combinaciones de bandas distintas, generan im\'agenes en falso color. La naturaleza de los objetos que se quieren investigar, determina la selecci\'on de las tres bandas a combinar \cite{com2015color}. A continuaci\'on se describe algunas combinaciones posibles con im\'agenes Landsat para la identificaci\'on visual de aspectos terrestres \cite{lillesand2014remote}:
	\begin{itemize}
		\item \textbf{Bandas 3,2,1 (RGB):} Es una imagen de color natural. Refleja el \'area tal como la observa el ojo humano en una fotograf\'ia a\'erea a color.
		\item  \textbf{Bandas 7,4,2 (RGB):} Permite discriminar los tipos de rocas. Ayuda en la interpretaci\'on estructural de los complejos intrusivos asociados a los patrones volcano - tect\'onicos.
		\item  \textbf{Bandas 5,4,2 (RGB):} Es una imagen que no refleja los patrones en colores naturales (falso color), por lo tanto las carreteras pueden ser rojas, el agua amarilla y la vegetaci\'on azul.
		\item \textbf{Bandas 7,3,1 (RGB):} Ayuda a diferenciar tipos de rocas, definir anomal\'ias de color que generalmente son de color amarillo claro algo verdoso, la vegetaci\'on es verde oscuro a negro, los r\'ios son negros y con algunas coloraciones azules a celestes.		
	\end{itemize}
 La Figura \ref{fig:combinacionColor} muestra como es combinada las bandas (3,4,5) en los canales (R,G,B).
  \begin{figure}[H]
  	\centering
  	\includegraphics[width=0.8	\textwidth]{./Figures/cap3/combinacionColor.jpg}
  	\caption{Combinaci\'on de bandas espectrales a trav\'es de los canales RGB.}
  	\label{fig:combinacionColor}
  \end{figure}

\section{\'Algebra de mapas}
El \'algebra de mapas constituye un marco te\'orico en la mayor parte de las operaciones hechas con SIG a partir de raster. Pueden desarrollarse operaciones de muy diverso tipo que se clasifican \cite{tomlin1990map} en:
	\begin{itemize}
		\item \textbf{Operadores locales:} los operadores locales generan una nueva imagen a partir de una o m\'as im\'agenes previamente existentes. Cada pixel de la nueva imagen recibe un valor que es funci\'on de los valores de ese mismo pixel en las dem\'as im\'agenes.
		\begin{equation}
		\label{e:opLoc}
		f_{1,2,3}=\rho(f_{1},f_{2},f_{3})
		\end{equation}
		\nomenclature[22]{$\rho$}{Funci\'on que representa a una operaci\'on aritm\'etica, l\'ogico, entre otros.}
		Donde $ \rho $ representa a alguna funci\'on del tipo:
		\begin{itemize}
			\item Aritm\'etico (suma, resta, multiplicaci\'on, divisi\'on, raiz cuadrada, potencia, ...).
			\item L\'ogico (AND, OR, XOR, NOT).
			\item Relacional ($ >, >=, <, <=, ==, != $).
			\item Trigonom\'etrico (sen, cos, tan, ...).
			\item Condicional (si cumple la condición ejecuta la instrucción).
		\end{itemize}
		La Figura \ref{fig:oploccond} nos muestra el proceso de operadores locales l\'ogico y condicional. En el \'item (a), el operador l\'ogico binariza la imagen si $ DEM > 400 $, mientras que en el \'item (b) el operador condicional clasifica la imagen en base a un rango ($ 600 - 650 = 1; 650 - 700 = 2; 700 - 750 = 3; 750 - 800 = 4 $).
		  \begin{figure}[H]
		  	\centering
		  	\includegraphics[width=0.9	\textwidth]{./Figures/cap3/operadores.png}
		  	\caption{Operadores locales condicional y l\'ogico.}
		  	\label{fig:oploccond}
		  \end{figure}
		\item  \textbf{Operadores de vecindad:} adjudican a cada pixel un valor que es funci\'on de los valores de un conjunto de pixeles contiguas, en una o varias im\'agenes. El conjunto de pixeles contiguas al pixel, m\'as ella misma constituye una vecindad.
		\begin{itemize}
			\item Filtrado de im\'agenes: es el conjunto de t\'ecnicas que se aplican a las im\'agenes digitales con el objetivo de mejorar la calidad o facilitar la b\'usqueda de informaci\'on.
			\item Operadores estad\'isticos: calcula variables estad\'isticas (media, desviaci\'on t\'ipica, m\'inimo, m\'aximo, entre otros.) a partir de los valores de todas los pixeles que forman la vecindad y lo adjudican al pixel central en la imagen de salida. 
			\item Operadores direccionales: Permiten estimar un conjunto de par\'ametros relacionados con la ubicaci\'on de los diferentes valores dentro de la vecindad. Su utilidad primordial es el an\'alisis de Modelos Digitales de Terreno (pendiente, orientación, curvatura, entre otros.)
		\end{itemize}		
		\item  \textbf{Operadores de vecindad extendida:} son aquellos que afectan a zonas relativamente extensas, que cumplen determinado criterio pero cuya localizaci\'on precisa no se conoce previamente. Por tanto el operador (algoritmo) debe determinar previamente cual es el \'area que cumple dichas caracter\'isticas. En la Figura \ref{fig:oplvecext} podemos ver el resultado de haber aplicado un operador de vecindad extendida a partir de pixeles situados a distancias $ 25,50,75,100,125 $.
						  \begin{figure}[H]
						  	\centering
						  	\includegraphics[width=0.9	\textwidth]{./Figures/cap3/operadorExte.png}
						  	\caption{\'Areas situadas a una distancia inferior a los valores umbrales 25,50,75,100,125.}
						  	\label{fig:oplvecext}
						  \end{figure}		
		\item \textbf{Operadores de \'area o zonales:} son aquellos que calculan alg\'un par\'ametro (superficie, per\'imetro, \'indices de forma, distancias, estad\'isticos) para una zona previamente conocida. Los valores pueden tratarse de diferentes niveles de una variable cualitativa o digitalizada e introducida por el usuario. En la Figura \ref{fig:oplarea} se observa tres im\'agenes. La primera est\'a clasificada en base a alg\'un criterio (Variable cualitativa) y la otra con niveles digitales igual a la altitud (Variable cuantitativa), donde la imagen resultante corresponde a la altitud media para cada grupo.
								  \begin{figure}[H]
								  	\centering
								  	\includegraphics[width=0.9	\textwidth]{./Figures/cap3/operadorearea.png}
								  	\caption{Operador de \'area: Altitud media por \'areas.}
								  	\label{fig:oplarea}
								  \end{figure}		
		
		\end{itemize}
 
\section{\'Indices de vegetaci\'on}
Los \'indices de vegetaci\'on son transformaciones que implican efectuar una combinaci\'on matem\'atica, entre los niveles digitales almacenados en dos o m\'as bandas espectrales de la misma imagen, teniendo en cuenta el comportamiento radiom\'etrico de la vegetaci\'on vigorosa para la elecci\'on de bandas \cite{speranza2005potencialidad}. \\~\\
El estudio de las cubiertas vegetales mediante la teledetecci\'on se aborda tradicionalmente mediante la utilización de los denominados “índices de vegetaci\'on”, siendo el m\'as utilizado el \'Indice de vegetaci\'on diferencial normalizada (NDVI) \cite{sader2000estimacion}.

\subsection{\'Indice de vegetaci\'on diferencial normalizada}\label{subsec:ndvi}
Sea una funci\'on $ ndvi:(x,y) \longrightarrow [-1,1] $ que determina la imagen con los NDVI en cada coordenada espacial $ (x,y) $ definida por la siguiente expresi\'on:
\begin{equation}
\label{e:ndvi}
ndvi(x,y)=\dfrac{f(x,y,IRc)-f(x,y,R)}{f(x,y,IRc)+f(x,y,R)}
\end{equation}
\nomenclature[23]{$ndvi$}{Imagen del \'indice de vegetaci\'on diferencial normalizada (NDVI).}
\nomenclature[24]{$ndvi(x,y)$}{ NDVI de las coordenadas $ (x,y) $.}
\nomenclature[25]{$R$}{ Posici\'on de la banda roja en la imagen satelital.}
\nomenclature[26]{$IRc$}{ Posici\'on de la banda infrarroja cercana en la imagen satelital.}
Donde $ R \in \{1,...,k\}$ representa la banda roja del espectro visible y  $ IRc \in \{1,...,k\}$ a la banda infrarroja cercana del espectro infrarrojo.\\~\\
En la Figura \ref{fig:firmaVegetacion} podemos observar como las plantas muestran un fuerte pico de absorci\'on causados por los pigmentos fotosint\'eticos en longitudes de onda cercanas a los 700 micrones (banda roja), hecho que contrasta con una fuerte reflexi\'on de las longitudes de onda del infrarrojo cercano \cite{salinero2002teledeteccion}. Por su parte, los suelos desnudos se caracterizan por un incremento suavemente monot\'onico de la reflectancia, a medida que aumenta la longitud de onda \cite{salinero2002teledeteccion}. Estas caracter\'isticas relevantes nos permiten elaborar varios tipos de an\'alisis como extracci\'on de \'indices.
\begin{figure}[H]
	\centering
	\includegraphics[width=0.8 \textwidth]{./Figures/cap3/Firma_espectral_vegetacion_vigorosa.jpg}
	\caption{Firma espectral de la vegetaci\'on \cite{ndvi2015com}.}
	\label{fig:firmaVegetacion}
\end{figure}

\subsubsection{Caracter\'isticas del NDVI}\label{subsec:subndvi}
El NDVI es un \'indice usado para estimar la cantidad, calidad y desarrollo de la vegetaci\'on por medio de sensores remotos instalados com\'unmente desde la plataforma espaciales, es decir mide las condiciones de vigor vegetal de la planta, principalmente su contenido en clorofila \cite{salinero2002teledeteccion}. El objetivo del NDVI es la reducci\'on de m\'ultiples bandas a una sola, condensando la informaci\'on m\'as importante, en este caso la vegetaci\'on.\\~\\
La principal ventaja del NDVI es su f\'acil interpretaci\'on, ya que sus valores var\'ian entre -1 y +1, permitiendo conocer el estado de vigor vegetal en grandes superficies y detecta fen\'omenos de amplio rango \cite{salinero2002teledeteccion}.

\section{An\'alisis Multitemporal}
El an\'alisis multitemporal de im\'agenes satelitales consiste en el estudio de zonas determinadas mediante tomas hechas en diferentes tiempos. El factor temporal puede abordarse con un doble objetivo: por un lado reconstruir la variaci\'on estacional de la zona y por otra parte la detecci\'on de cambios. Este \'ultimo objetivo se enfoca en detectar cambios entre dos o m\'as
fechas alejadas en el tiempo, estudiando el dinamismo temporal de una determinada zona como por ejemplo: el crecimiento urbano, transformaciones agrícolas, entre otras \cite{salinero2002teledeteccion}.\\~\\
En el enfoque aplicado al estudio multitemporal resulta preciso abordar previamente una serie de tratamientos sobre las im\'agenes satelitales de cara a garantizar su comparabilidad, ya que existen factores naturales o las del sensor, que influyen desde la captura de informaci\'on hasta su transformaci\'on final a niveles digitales que afectar\'ia el an\'alisis.

\section{Correcciones a las im\'agenes satelitales}
Las correcciones satelitales son el producto de aplicar un operador a una imagen satelital para la obtenci\'on de otra. Las correcciones est\'an definida seg\'un la expresi\'on:
		\begin{equation}
			f^{'}=T[f]
		\end{equation} 
Donde $ f $ es una imagen satelital de entrada, $ f^{'} $ es la imagen corregida y $ T $ es un operador que realiza las correcciones a la imagen $ f $, debido a fallos en los sensores, alteraciones en el movimiento del sat\'elite o interferencias de la atm\'osfera \cite{teledUm}.
\nomenclature[27]{$ T[f] $}{ Operador que corrige la imagen satelital $ f $.}
\subsection{Correccci\'on geom\'etrica}\label{sec:corrGeometrica}
Una imagen de sat\'elite, al igual que las fotograf\'ias a\'ereas, no proporciona informaci\'on georreferenciada; cada pixel se ubica en un sistema de coordenadas arbitrario de tipo fila-columna como los que manejan los programas de tratamiento digital de im\'agenes \cite{deniseCultivos}.\\~\\
El proceso consiste en dar a cada pixel su localizaci\'on en un sistema de coordenadas estandard (UTM, lambert, coordenadas geogr\'aficas) para poder combinar la imagen de sat\'elite con otro tipo de capas en un entorno SIG. El proceso obtiene una nueva capa en la que cada columna corresponde con un valor de longitud y cada fila con un valor de latitud. En caso de que la imagen no hubiese sufrido ningún tipo de distorsi\'on, el procedimiento ser\'ia bastante sencillo, sin embargo una imagen puede sufrir diversos tipos de distorsiones.\\~\\
Es necesario localizar puntos comunes de la imagen con puntos de control, como tarea inicial para la correcci\'on geom\'etrica, de manera a poder realizar una interpolaci\'on espacial y de los valores radiom\'etricos \cite{deniseCultivos}.

\subsubsection{Interpolaci\'on espacial}
La interpolaci\'on espacial consiste en determinar la relaci\'on geom\'etrica entre las coordenadas del pixel de la imagen a corregir y sus coordenadas geogr\'afica correspondientes. Utilizando los puntos comunes localizados, se plantea una ecuaci\'on de transformaci\'on mediante la cual se obtiene la posici\'on de los pixeles en la imagen de salida, ilustrada en la Figura \ref{fig:intEspacial}. Este proceso tambi\'en es conocido como Georreferenciaci\'on.  
    \begin{figure}[H]
    	\centering
    	\includegraphics[width=0.6	\textwidth]{./Figures/cap3/inter_spacial.png}
    	\caption{Localizaci\'on de puntos comunes y puntos de referencia.}
    	\label{fig:intEspacial}
    \end{figure}
    
\paragraph{Transformación usando ecuaciones polinomicas. }\mbox{}\\\mbox{}\\
El m\'etodo mas utilizado para la transformaci\'on es el de ecuaciones polin\'omicas.	 La transformaci\'on puede expresarse de la siguiente manera:
	\begin{equation}
	x^{'}_{i} = \sum_{j=0}^{l} \sum_{e=0}^{l-j} a_{ij}x^{j}_{i}y^{e}_{i}
	\end{equation} 
		\begin{equation}
		y^{'}_{i} = \sum_{j=0}^{l} \sum_{e=0}^{l-j} b_{ij}x^{j}_{i}y^{e}_{i}
		\end{equation} 


\nomenclature[28]{$ l $}{ Grado del polinomio de ajuste.}
Donde $ x^{'}_{i} $ e $ y^{'}_{i} $ indica la coordenada en la imagen corregida para la banda $ i $. El superindice $ l $ indica el grado del polinomio de ajuste, $ a_{i} $ y $ b_{i} $ los coeficientes del polinomio. Siendo la ecuaci\'on lineal las mas simple:
	\begin{equation}
	x^{'}_{i} = a_{0}+a_{1}x_{i}+a_{2}y_{i}
	\end{equation} 
		\begin{equation}
		y^{'}_{i} = b_{0}+b_{1}x_{i}+b_{2}y_{i}
		\end{equation} 
		\nomenclature[29]{$ (x^{'},y^{'}) $}{Coordenadas de la imagen transformada.}
En distorsiones moderadas o en un \'area reducida, se utilizan transformaciones de primer orden, pudiendo corregir efectos de translaci\'on en $ x^{'}_{i} $ e $ y^{'}_{i} $, cambios de escala y rotaci\'on.
En distorsiones m\'as importantes o en \'areas extensas, es necesario una transfomaci\'on de segundo orden. Este tipo de transformaci\'on agregan a diferencia del primer orden, correcciones a deformaciones locales.
En la Figura \ref{fig:intPolEcua} podemos observar las transformaciones con polinomios de primer y segundo orden.
    \begin{figure}[H]
    	\centering
    	\includegraphics[width=0.9	\textwidth]{./Figures/cap3/ecuacPolinomica.png}
    	\caption{Interpolaci\'on espacial con polinomios de primer y segundo orden.}
    	\label{fig:intPolEcua}
    \end{figure}
  \paragraph{Calidad de la interpolaci\'on espacial. }\mbox{}\\\mbox{}\\
 La calidad en la interpolaci\'on espacial y los puntos de control seleccionados $ \eta $ se calcula utilizando el promedio de los errores cuadráticos medios (RMS), que consiste en la diferencia entre la coordenada transformada deseada para un punto de control y la coordenada real obtenida como salida.

 \begin{equation}
 RMS_{i} = \sqrt{\dfrac{\sum_{j=1}^{\eta} ((x_{i,j}^{'}-x_{i,j})^{2}+(y_{i,j}^{'}-y_{i,j})^{2})}{\eta}}
 \end{equation} 
 		\nomenclature[30]{$ RMS_{i} $}{Error cuadr\'atico de la banda $ i $.}
 		\nomenclature[31]{$ \eta $}{N\'umero de puntos de control.}

 	El valor de $ RMS $ elegido por referencia para corregir una imagen debe ser aproximadamente $ 0.5 $, y en lo posible nunca superar la unidad \cite{guide1999erdas}. \\~\\
 	La Figura \ref{fig:rms} nos muestra la manera de como es calculado el RMS para un punto de control determinado.
 
 
     \begin{figure}[H]
     	\centering
     	\includegraphics[width=0.9	\textwidth]{./Figures/cap4/rms.png}
     	\caption{Error RMS de un punto de control $ (x,y) $ y su transformaci\'on $ (x^{'},y^{'}) $.}
     	\label{fig:rms}
     \end{figure}



\subsubsection{Interpolaci\'on de los valores radiom\'etricos}
La interpolaci\'on de los valores radiom\'etricos es el traslado del nivel digital perteneciente a la imagen original a la imagen corregida espacialmente. La imagen original debe corresponderse con las coordenadas de la imagen corregida. La interpolaci\'on puede ser abordada por tres m\'etodos diferentes:
	\begin{itemize}
		\item \textbf{Vecino m\'as pr\'oximo:} situ\'a en cada pixel de la imagen corregida el nivel digital $ (ND) $ del pixel m\'as cercano en la imagen original. Constituye la soluci\'on m\'as r\'apida y la que supone menor transformaci\'on en los niveles digitales originales. Su principal inconveniente es que produce una distorsi\'on en rasgos lineales en la imagen (fracturas, carreteras, caminos), que pueden aparecer en la corregida como lineales quebradas. \\~\\
En la Figura \ref{fig:vecinoMasCercano2} se observa el como los pixeles de la imagen transformada son trasladados a la imagen corregida a lado del vecino m\'as pr\'oximo.
				    \begin{figure}[H]
				    	\centering
				    	\includegraphics[width=0.6	\textwidth]{./Figures/cap3/vecinoMasCercano2.png}
				    	\caption{Interpolaci\'on Vecino m\'as Cercano.}
				    	\label{fig:vecinoMasCercano2}
				    \end{figure}
		
		\item \textbf{Interpolaci\'on bilineal:} Considera el valor de los 4 pixeles mas cercanos en la imagen de entrada para asignar el nuevo valor de la imagen de salida. Las ventajas son que no existe el efecto de escalones en los bordes pudiendo aparecer en el vecino superior izquierdo y ademas cuenta con mejor exactitud espacial. El m\'etodo es utilizado a menudo cuando se cambia el tama\~{n}o de las celdas en los datos. La desventaja es que como los pixeles son promediados, algunos extremos de los valores de los datos pueden perderse. En la Figura \ref{fig:bilineal2} podemos observar los 4 pixeles cercanos tomados para la interpolaci\'on.
		\begin{figure}[H]
			\centering
			\includegraphics[width=0.6	\textwidth]{./Figures/cap3/bilineal.png}
			\caption{Interpolaci\'on Bilineal.}
			\label{fig:bilineal2}
		\end{figure}
		
		    		\item \textbf{Convoluci\'on c\'ubica:} es similar a la interporlaci\'on bilineal pero considera niveles digitales de los 16 pixeles m\'as pr\'oximos. El efecto visutal es mejor, pero supone un volumen de c\'alculo mucho m\'as elevado. La Figura \ref{fig:convCubica2} nos muestra los 16 pixeles tomados en el m\'etodo. 
		    				    \begin{figure}[H]
		    				    	\centering
		    				    	\includegraphics[width=0.3	\textwidth]{./Figures/cap3/convolucionCubica_matriz.png}
		    				    	\caption{Convoluci\'on c\'ubica.}
		    				    	\label{fig:convCubica2}
		    				    \end{figure}
	\end{itemize}
\subsection{Correcci\'on radiom\'etrica}
La correci\'on radiom\'etrica se encarga de minimizar los desajustes producidos en el registro del valor digital en los pixeles de la imagen, de hecho en algunos casos las estaciones receptoras llevan a cabo alg\'un tipo de correcci\'on en el momento de recepci\'on de la imagen. La corrección radiom\'etrica implica por una parte la restauraci\'on de lineas o p\'ixeles perdidos y por otra la correcci\'on del bandeado en la imagen \cite{teledUm}.
    \begin{figure}[H]
    	\centering
    	\includegraphics[width=0.9	\textwidth]{./Figures/cap3/correcError.png}
    	\caption{Fallos del sensor en la captura de la imagen.}
    	\label{fig:correcError}
    \end{figure}
La Figura \ref{fig:correcError} nos muestran tres tipos de errores radiom\'etricos frecuentes, donde la lineas claras representan pixeles que no representan su nivel digital correcto a causa de la descalibraci\'on del sensor. Las lineas y pixeles negros son pixeles nulos que no pudieron ser convertidos a su nivel digital por fallos en el detector o transmisiones, como tambi\'en por conversiones de la informaci\'on anal\'ogica a digital.

\subsubsection{Pixeles o lineas perdidas}\label{subsec:pixelesP}
Sea $ f^{'} $ una imagen satelital con los pixeles corregidos, mediante estimaciones de la media entre los pixeles adyacentes. Donde cada nivel digital es calculado de la siguiente manera: 

		\begin{equation}\label{ec:pixelPerdido}
		f^{'}(x,y,i) =  ceiling(1/2 \times [f(x-1,y,i) + f(x+1,y,i)])
		\end{equation} 
Donde $ ceiling(:) $ representa la funci\'on techo (redondeo para arriba). No es recomendable utilizar los pixeles adyacentes de la misma linea (eje $ y $) por que han sido captados por el mismo detector o banda que ha dado el fallo, por tanto son poco fiables.

				\nomenclature[42]{$ \mu $}{Valor de la media.}
				\nomenclature[43]{$ \sigma $}{Desviaci\'on t\'ipica.}
				\nomenclature[44]{$ i_{*} $}{Banda de la imagen satelital.}
				\nomenclature[45]{$ \sigma_{i} $}{Desviaci\'on t\'ipica de la banda $ i $ de una imagen satelital.}
				\nomenclature[46]{$ \sigma_{i_{*}} $}{Desviaci\'on t\'ipica de la banda $ i_{*} $ de una imagen satelital.}
La bandas de una imagen son de detectores diferentes y est\'an altamente correlacionadas, por lo que se hace una modificaci\'on a la ecuaci\'on \ref{ec:pixelPerdido} teniendo en cuenta la banda $ i_{*} $ de la imagen satelital, donde $ i_{*} \in [1,k] $, pudiendo utilizarse el valor del pixel faltante en una banda diferente para mejorar la estimaci\'on:
		\begin{equation}
		f^{'}(x,y,i) = ceiling( dif_{i}+ \dfrac{\sigma_{i}}{\sigma_{i_{*}}} \times [f(x,y,i_{*})-dif_{i_{*}}])
		\end{equation} 
\nomenclature[47]{$ f^{'}(x,y,i) $}{Imagen satelital corregida, con coordenadas espaciales $ (x,y) $ en la banda $ i $.}
Donde $ \sigma_{i} $ representa la desviaci\'on t\'ipica de la banda $ i $, $ \sigma_{i_{*}} $ representa la desviaci\'on t\'ipica de la banda $ i_{*} $ y la variable $ dif_{e} $, donde $ e \in \{ i,i_{*}\} $, esta representada por la siguiente expresi\'on:
		\begin{equation}
		 dif_{e}  = 1/2 \times [f(x-1,y,e) + f(x+1,y,e)]
		\end{equation} 
En caso de que la imagen abarque un territorio amplio y cambiante resulta recomendable calcular las desviaciones t\'ipicas ($ \sigma_{i} $ y $ \sigma_{i_{*}} $) en un entorno cercano al pixel perdido.\\~\\
Para detectar lineas perdidas se compara la media de los $ ND $ de una linea con las medias de las lineas anterior y posterior, para detectar pixeles perdidos se compara el valor de un pixel con los de los 8 pixeles.
\subsubsection{Bandeado}\label{subsec:bandeado}
El fen\'omeno del bandeado se debe a una mala calibraci\'on entre detectores y resulta especialmente visible en las zonas de baja radiancia (zonas marinas por ejemplo). El resultado es la aparici\'on peri\'odica de una banda m\'as clara u oscura que las dem\'as.
Para corregir el bandeado se asume que, en caso de no haber error, los histogramas obtenidos por cada uno de los detectores ser\'ian similares entre s\'i y similares al histograma global de la imagen que se toma como referencia.\\~\\
En primer lugar se calculan los coeficientes $ m $ y $ s $ para una correcci\'on lineal de cada uno de las bandas.
		\begin{equation}
		m =\dfrac{\sigma_{i}}{\sigma_{i_{*}}}
		\end{equation} 	
				\begin{equation}
				s=\mu_{i} - m \times \mu_{i_{*}}
				\end{equation} 	
				
Siendo $ \mu_{i} $ y $ \mu_{i_{*}} $ las medias para la banda $ i $ e $ i_{*} $. Donde cada nivel digital de la imagen satelital $ f^{'} $ es calcula de la siguiente manera:
				\begin{equation}
				f^{'}(x,y,i) = m \times f(x,y,i) + s
				\end{equation} 				

En Figura \ref{fig:bandeado} podemos observar el histogramas de una banda corregida en funci\'on a las dem\'as bandas pertenecientes a una imagen satelital. 
    \begin{figure}[H]
    	\centering
    	\includegraphics[width=0.9	\textwidth]{./Figures/cap3/bandeo_k.png}
    	\caption{Proceso de correcci\'on del bandeo.}
    	\label{fig:bandeado}
    \end{figure}

\section{Proceso de detecci\'on de cambios}
	En los m\'etodos comunes de detecci\'on de cambios se asigna un valor correspondiente al grado de cambio sobre cada pixel, independientemente del resto de la imagen satelital. En estos m\'etodos se considera el pixel como unidad b\'asica (\'algebra de mapas) para aplicar las correspondientes operaciones matem\'aticas asociadas a cada algoritmo.\\~\\
Los m\'etodos de comparaci\'on, generan una imagen (\'indice de cambios) que representa el grado de cambio entre dos situaciones temporales; los pixeles de la imagen resultante, contienen una variable continua de tipo cuantitativo (Niveles digitales), por lo que se requieren t\'ecnicas que los conviertan en variables cualitativa (Categor\'ias) \cite{martinez2013normalizacion}.
\subsection{Comparaci\'on multitemporal}\label{subsec:compMult}
La comparaci\'on parte de un par de im\'agenes semejantes que abarcan la misma zona de estudio, siguiendo una secuencia multitemporal. Una secuencia multitemporal esta definido por $ \{f_{t}\}_{t \in \mathbb{N}} $, donde cada imagen satelital representa a una en diferentes tiempos. Las operaciones m\'as utilizadas son \cite{chuvieco1998factor}: 
	\begin{itemize}
		\item \textbf{Diferencia de im\'agenes:} es el m\'etodo m\'as simple, f\'acil de interpretar y directo, ya que consiste en una diferencia algebraica entre los niveles digitales ($ ND $ ) inicial y final para la obtenci\'on de un \'indice ($ I_{dif} $). Normalmente es realizada combinada  con extracciones de \'indices espectrales.
								\begin{equation}
								I_{dif} = f_{t_{*}}-f_{t}
								\end{equation} 	
		Donde $ t_{*} \neq t $.
\nomenclature[49]{$ I_{dif} $}{Indice de cambio por diferencia de im\'agenes.}
\nomenclature[50]{$ I_{ratio} $}{Indice de cambio por el m\'etodo de ratio.}
				\item \textbf{Ratio:} se obtiene aplicando la operación de cociente, entre los niveles digitales ($ ND $ ) inicial y final para la obtenci\'on de un \'indice ($ I_{ratio} $). Podr\'ia  generar mejores resultados pero no se ajusta a una distribución normal.
										\begin{equation}
										I_{ratio} = \dfrac{f_{t_{*}}}{f_{t}}
										\end{equation} 	

		\end{itemize}
Estas dos operaciones generan una imagen con \'indices de cambios $ I_{c} $ a partir de cada conjunto de datos multitemporal, dando lugar a tantos mapas de cambios como bandas/capas se consideren.
\subsection{Criterios de decisi\'on}
El resultado de los c\'alculos es una imagen en donde el valor de salida indica el grado de cambio, desde la mayor p\'erdida a la mayor ganancia, en una escala gradual. Si se pretende generar una imagen binaria (cambio/estable), es preciso se\~{n}alar un umbral que delimite ambas categor\'ias en las im\'agenes. Ah\'i se plantea un problema de dif\'icil soluci\'on ya que no existen criterios de aplicación general.\\~\\
Si el cambio abarca un sector importante de la imagen, el histograma de la imagen de cambios ($ I_{c} $) debiese mostrar un perfil bimodal, lo que permitir\'ia establecer umbrales naturales de cambio, aunque esta situaci\'on no es muy habitual, ya que los cambios en la naturaleza no suelen producirse de modo abrupto \cite{martinez2013normalizacion}.\\~\\
Si es necesario establecer un umbral para separar las \'areas de cambio, puede optarse por se\~{n}alar alg\'un criterio estad\'istico, como la media y la desviaci\'on t\'ipica de una serie de p\'ixeles elegidos aleatoriamente. En ocasiones se ha propuesto utilizar unas \'areas de entrenamiento para calcular que rango de desviaci\'on se pod\'ia considerar l\'imite para p\'ixeles estables, aplicando luego ese valor al conjunto de la imagen \cite{tung1988determination}.

\subsubsection{Discriminaci\'on de las zonas de cambio}\label{sec:discriminacion}
La comparaciones multitemporales generan indices que corresponden a variables cuantitativas, por lo que la aplicaci\'on de m\'etodos que discriminen las zonas, en tipos de cambios, permitir\'an un an\'alisis mas especifico sobre la imagen satelital.\\~\\ 
Sea $ B:I_{c} \longrightarrow \{0,1\}$ 	una funci\'on que binariza una imagen en base a un Umbral $ U $:
\begin{equation}\label{ec:umbralizacion}
B(I_{c}) = \begin{cases}
1 & \text{si se cumple que } I_{c} \geq U \\
0 & \text{en cualquier otro caso}
\end{cases}
\end{equation}

\nomenclature[51]{$ I_{c} $}{Indice de cambio entre dos im\'agenes.}
\nomenclature[52]{$ B $}{Imagen binaria en el proceso de detecci\'on de cambio.}
\nomenclature[53]{$ B(I_{c}) $}{Nivel digital para el $ I_{c} $.}
\nomenclature[54]{$ U $}{Umbral que binariza una imagen en base a un $ I_{c} $.}
La ecuaci\'on \ref{ec:umbralizacion} genera una m\'ascara binaria de cambios (0, No Cambio; 1, Cambio) aplicando un umbral ($ U $) especifico sobre la imagen resultante del proceso de comparaci\'on multitemporal \cite{singh1989review}. Son f\'acilmente implementables en procesos de car\'acter autom\'atico/semiautom\'atico. Partiendo de la hip\'otesis de que el porcentaje de cambios es muy reducido, los valores correspondientes se encuentran situados en los extremos del histograma de frecuencias \cite{estornell2004analisis}. Es preciso se\~{n}alar un umbral que delimite ambas categorías (cambio/no cambio) a partir del \'indice de cambios \cite{radke2005image} para generar una m\'ascara.\\~\\
El m\'etodo de discriminaci\'on basado en los par\'ametros estad\'isticos del \'indice de cambio entre la secuencia temporal de im\'agenes tiene la siguiente expresi\'on \cite{rodriguez2010analisis}:
\begin{equation}
U=\mu_{I_{c}} \pm n \times \sigma _{I_{c}}
\end{equation}
\nomenclature[55]{$ n $}{Coeficiente de fiabilidad de los datos.}
\nomenclature[56]{$ \sigma_{I_{c}} $}{Desviaci\'on t\'ipica de la imagen de \'Indices de cambio.}
\nomenclature[56]{$ \mu_{I_{c}} $}{Media de la imagen de \'Indices de cambio.}
Donde el valor de umbral entre cambio/no cambio $ (U) $ se estima en funci\'on a la media ($ \mu_{I_{c}}) $ y desviaci\'on ($ \sigma_{I_{c}} $) de la imagen de Indice de cambio ($ I_{c} $), junto con un coeficiente de tolerancia $ n $ asignado en base a la fiabilidad de los datos. Los resultados se clasifican en funci\'on de $ n $; alta probabilidad de cambio $ (n \geq 2) $ y
zonas de media probabilidad de cambio $ (1 < n < 2) $ \cite{estornell2004analisis}.

\subsection{Filtrado}
Los filtros constituyen unos de los principales m\'etodos del procesamiento digital de im\'agenes . Pueden usarse para distintos fines, pero siempre, el resultado sobre cada pixel depende de los pixeles en su entorno. Tiene como objetivos: 
	\begin{itemize}
		\item \textbf{Suavizar la imagen:} reducir las variaciones de intensidad entre p\'ixeles vecinos.
		\item \textbf{Eliminar ruido:}  modificar aquellos p\'ixeles cuyo nivel de intensidad es muy diferente al de sus vecinos.
		\item \textbf{Realzar la imagen:} aumentar las variaciones de intensidad, all\'i donde se producen.
		\item \textbf{Detectar bordes::} detectar aquellos p\'ixeles donde se produce un cambio brusco en la funci\'on intensidad.	
	\end{itemize}

\section{Resumen}

La utilizaci\'on de im\'agenes satelitales implica un pre-procesamiento adicional, diferente a las que se le realiza a im\'agenes normales. Estos pre-procesamientos van ligados a resoluciones, firmas espectrales y tipos de im\'agenes satelitales propias del sensor espacial que captura la informaci\'on. En este capitulo se describen conocimientos previos para aplicar an\'alisis multitemporales y detecci\'on de cambio en im\'agenes de sat\'elite que componen piezas fundamentales en la metodolog\'ia propuesta.
\newpage{\ } 
\thispagestyle{empty} 

\chapter{Metodología propuesta}
\lhead{Capítulo 4. \emph{Metodología propuesta}} % This is for the header on each page - perhaps a shortened title
En este capítulo  se describe  el funcionamiento de la metodología propuesta, presentada en la FIGURA \ref{fig:metodo}, compuesta por 3 módulos:
En el primer módulo se realiza la detección y segmentación de los vasos sanguíneos,  exudados duros y microaneurismas. En el siguiente módulo, se extraen las características de los vasos sanguíneos, exudados duros y microaneurismas de manera que puedan ser utilizadas posteriormente. En el tercer módulo se realiza la clasificación en base a la características extraídas haciendo uso del clasificador support vector machine (SVM). 
\begin{figure}[H]
	\centering
		\includegraphics[width=0.6	\textwidth]{./Figures/cap4/metodologiaa.png}
	\caption{Metodología Propuesta.}
	\label{fig:metodo}
\end{figure}


\section{Módulo 1: Detección y Segmentación}
Como se había mencionado en el capítulo anterior la segmentación es el proceso de asignar una etiqueta  a cada pixel en una imagen, tal que los píxeles con las mismas etiquetas compartan ciertas características visuales. El objetivo de la segmentación es simplificar y cambiar la representación de una imagen en algo que sea más significativo y fácil de analizar \cite{seg1,seg2}.
  
La segmentación de imágenes es usada par localizar estructuras y patologías de los ojos tales como vasos sanguíneos, exudados duros y microaneurismas.

\subsection{Detección y segmentación de vasos sanguíneos}


%\subsubsection{Canal verde de la imagen}

Como primer paso en la detección y segmentación de vasos sanguíneos se obtiene del canal verde de la imagen de retina $f_{in}$ debido a que la vasos contienen características que aparecen más contrastadas en este canal (FIGURA \ref{fig:vaso_1}). Sobre el canal verde se aplica el estiramiento de contraste utilizando la técnica de ecualización adaptativa del histograma de contraste limitado (CLAHE) \cite{zuiderveld1994contrast} para suavizar el fondo de la imagen (FIGURA \ref{fig:vaso_2}).
\begin{figure}[H]
\centering
\subfigure[Imagen de retina.]{\includegraphics[width=50mm]{./Figures/cap4/vasos/vaso1.jpg}}
\subfigure[Canal verde de la Imagen.]{\includegraphics[width=50mm]{./Figures/cap4/micro0.jpg}}
\caption{Canal verde de la imagen.} \label{fig:vaso_1}
\end{figure}

%\subsubsection{CLAHE}

\begin{figure}[H]
\centering
\subfigure[Canal verde de la Imagen.]{\includegraphics[width=50mm]{./Figures/cap4/micro0.jpg}}
\subfigure[Imagen ecualizada.]{\includegraphics[width=50mm]{./Figures/cap4/vasos/vaso2.jpg}}
\caption{Ecualización del histograma.} \label{fig:vaso_2}
\end{figure}

%\subsubsection{Normalización de la intensidad}
Luego se normaliza la intensidad de tal manera que la misma se expanda a través del rango de intensidad obteniendo la imagen normalizada (FIGURA \ref{fig:vaso_3}). Sobre esta imagen se aplica el filtro de la mediana para obtener una imagen de fondo (FIGURA \ref{fig:vaso_4}).
 
\begin{figure}[H]
\centering
\subfigure[Imagen ecualizada.]{\includegraphics[width=50mm]{./Figures/cap4/vasos/vaso2.jpg}}
\subfigure[Imagen normalizada.]{\includegraphics[width=50mm]{./Figures/cap4/vasos/vaso3.jpg}}
\caption{Normalización de la intensidad.} \label{fig:vaso_3}
\end{figure}


%\subsubsection{Filtro de la mediana}

\begin{figure}[H]
\centering
\subfigure[Imagen normalizada.]{\includegraphics[width=50mm]{./Figures/cap4/vasos/vaso3.jpg}}
\subfigure[Imagen filtrada.]{\includegraphics[width=50mm]{./Figures/cap4/vasos/vaso4.jpg}}
\caption{Filtro de la mediana.} \label{fig:vaso_4}
\end{figure}

%\subsubsection{Resta de imágenes}

%\subsubsection{Umbralización}
 Se resta de la imágen de fondo la imagen normalizada dando como resultado una imagen con vasos resaltados (FIGURA \ref{fig:vaso_5}). Esta imagen es umbralizada y se obtienen los vasos sanguíneos en una imagen binaria (FIGURA \ref{fig:vaso_6}).
\begin{figure}[H]
\centering
\subfigure[Imagen filtrada.]{\includegraphics[width=50mm]{./Figures/cap4/vasos/vaso4.jpg}}
\subfigure[Imagen resultante.]{\includegraphics[width=50mm]{./Figures/cap4/vasos/vaso5.jpg}}
\caption{Resta de imágenes.} \label{fig:vaso_5}
\end{figure}


 
\begin{figure}[H]
\centering
\subfigure[Imagen resultante.]{\includegraphics[width=50mm]{./Figures/cap4/vasos/vaso5.jpg}}
\subfigure[Imagen umbralizada.]{\includegraphics[width=50mm]{./Figures/cap4/vasos/vaso6.jpg}}
\caption{Umbralización.} \label{fig:vaso_6}
\end{figure}

%\subsubsection{Cierre de una imagen}
Luego, se aplica un cierre con elemento estructurante con forma de línea con un tamaño de 7 píxeles de manera a acentuar los vasos (FIGURA \ref{fig:vaso_7}). Para remover el ruido contenido en la imagen binaria se elimina los componentes conectados pequeños resultando en la imagen final $f_{vs}$ (FIGURA \ref{fig:vaso_8}). 

\begin{figure}[H]
\centering
\subfigure[Imagen umbralizada.]{\includegraphics[width=50mm]{./Figures/cap4/vasos/vaso6.jpg}}
\subfigure[Cierre de  imagen.]{\includegraphics[width=50mm]{./Figures/cap4/vasos/vaso7.jpg}}
\caption{Cierre de una imagen.} \label{fig:vaso_7}
\end{figure}

%\subsubsection{Eliminar componentes conectados pequeños}

\begin{figure}[H]
\centering
\subfigure[Cierre de imagen]{\includegraphics[width=50mm]{./Figures/cap4/vasos/vaso8.jpg}}
\subfigure[Imagen final de vasos sanguíneos.]{\includegraphics[width=50mm]{./Figures/cap4/vasos/vaso9.jpg}}
\caption{Imagen final $f_{vs}$.} \label{fig:vaso_8}
\end{figure}

La secuencia de pasos para la detección y segmentación de vasos sanguíneos puede verse en la FIGURA \ref{fig:bloquesVS}.
%y la secuencia de imágenes generadas son desplegadas en la FIGURA \ref{fig:secVena}.




\begin{figure}[H]
	\centering
		\includegraphics[width=0.7	\textwidth]{./Figures/cap4/dia_vena1.png}
	\caption{Diagrama de bloques de detección y segmentación de vasos sanguíneos.}
	\label{fig:bloquesVS}
\end{figure}

%\begin{figure}[H]
%	\centering
%		\includegraphics[width=0.7	\textwidth]{./Figures/cap4/sec_vena.png}
%	\caption{Secuencia de imágenes de detección y segmentación de vasos sanguíneos (a) Imagen de retina, (b) Canal verde, (c) Imagen ecualizada, (d) Imagen con intensidad Ajustada, (e) Imagen generada por el filtro de la mediana, (f) Resta de imágenes, (g) Imagen umbralizada, (h) Cierre de la imagen, (i) Imagen sin componentes conectados, (j) Imagen de  vasos sanguíneos segmentada $f_{vs}$.}
%	\label{fig:secVena}
%	\end{figure}
	
%	\subsection{Secuencias de imágenes genradas por el algoritmo de detección de vasos sanguíneos}
	


\subsection{Detección y segmentación de exudados duros} 
La detección y segmentación de los exudados duros necesitan previamente la segmentación y detección del disco óptico y el borde circular. 

\subsubsection{Detección y segmentación de Disco Óptico}
Uno de los mayores problemas a la hora de detectar exudados duros es la similaridad de coloración que los mismos poseen con el disco óptico \cite{el2013automatic}. Para resolver este problema se realiza la detección del disco óptico. 



La imagen original$f_{in}$ en RGB es pasada a escala de grises usando el canal verde de la imagen (FIGURA \ref{fig:disco_1}). Se realiza el estiramiento de contraste utilizando la técnica de ecualización adaptativa del histograma de contraste limitado (CLAHE) para suavizar el fondo de la imagen (FIGURA \ref{fig:disco_2}).

\begin{figure}[H]
\centering
\subfigure[Imagen de retina.]{\includegraphics[width=50mm]{./Figures/cap4/vasos/vaso1.jpg}}
\subfigure[Canal verde de la Imagen.]{\includegraphics[width=50mm]{./Figures/cap4/disco/disco1.png}}
\caption{Canal verde de la imagen.} \label{fig:disco_1}
\end{figure}

\begin{figure}[H]
\centering
\subfigure[Canal verde de la imagen.]{\includegraphics[width=50mm]{./Figures/cap4/disco/disco1.png}}
\subfigure[Imagen ecualizada.]{\includegraphics[width=50mm]{./Figures/cap4/disco/disco2.png}}
\caption{Ecualización del histograma.} \label{fig:disco_2}
\end{figure}

Luego  se realiza un ajuste de los valores de intensidad para reducir el ruido de la imagen, de manera a mejorar la imagen ( FIGURA \ref{fig:disco_3}). Como siguiente paso, se procede a la umbralización de la imagen mejorada, obteniendo de esta manera una imagen umbralizada ( FIGURA \ref{fig:disco_4}).

\begin{figure}[H]
\centering
\subfigure[Imagen ecualizada.]{\includegraphics[width=50mm]{./Figures/cap4/disco/disco2.png}}
\subfigure[Imagen con intensidad ajustada.]{\includegraphics[width=50mm]{./Figures/cap4/disco/disco3.png}}
\caption{Ajuste de intensidad.} \label{fig:disco_3}
\end{figure}

\begin{figure}[H]
\centering
\subfigure[Imagen con intensidad ajustada.]{\includegraphics[width=50mm]{./Figures/cap4/disco/disco3.png}}
\subfigure[Imagen umbralizada.]{\includegraphics[width=50mm]{./Figures/cap4/disco/disco4.png}}
\caption{Umbralización.} \label{fig:disco_4}
\end{figure}

A la imagen umbralizada se le aplica una operación de erosión con  un elemento estructurante con forma de disco, con un tamaño de 12 píxeles, con el fin de acentuar la forma circular del disco óptico ( FIGURA \ref{fig:disco_5}). Como paso siguiente esta imagen se dilata  con un elemento estructurante en forma de disco, con un tamaño de 20 píxeles ( FIGURA \ref{fig:disco_6}).
\begin{figure}[H]
\centering
\subfigure[Imagen umbralizada.]{\includegraphics[width=50mm]{./Figures/cap4/disco/disco4.png}}
\subfigure[Imagen erosionada.]{\includegraphics[width=50mm]{./Figures/cap4/disco/disco6.png}}
\caption{Erosión de imagen.} \label{fig:disco_5}
\end{figure}

\begin{figure}[H]
\centering
\subfigure[Imagen erosionada.]{\includegraphics[width=50mm]{./Figures/cap4/disco/disco6.png}}
\subfigure[Imagen dilatada.]{\includegraphics[width=50mm]{./Figures/cap4/disco/disco8.png}}
\caption{Dilatación de imagen.} \label{fig:disco_6}
\end{figure}

Asumiendo que el disco óptico es el objeto con mayor área, se eliminan todos los otros componentes conectados, dejando solo el elemento de mayor área, el cual es el  disco óptico$f_{do}$ ( FIGURA \ref{fig:disco_7}).

\begin{figure}[H]
\centering
\subfigure[Imagen dilatada.]{\includegraphics[width=50mm]{./Figures/cap4/disco/disco8.png}}
\subfigure[Imagen final $f_{do}$]{\includegraphics[width=50mm]{./Figures/cap4/disco/disco9.png}}
\caption{Imagen final del disco óptico.} \label{fig:disco_7}
\end{figure}


 La secuencia de pasos para la detección y segmentación de disco óptico puede verse en la FIGURA \ref{fig:diado}.
 % y  la secuencia de imágenes generadas son desplegadas en la FIGURA \ref{fig:discoptico}.

\begin{figure}[H]
	\centering
		\includegraphics[width=0.7	\textwidth]{./Figures/cap4/dia_doo.jpeg}
	\caption{Diagrama de bloques de detección y segmentación de disco óptico.}
	\label{fig:diado}
\end{figure}

\subsubsection{Detección y segmentación del borde circular}
Con el objetivo de delimitar correctamente el área en la cual se encuentran los exudados duros, es  necesaria  la creación de un borde circular de las imágenes de retinas, ya que en algunas imágenes los bordes circulares son más claros y estos pueden ser confundidos por exudados duros.

 Para tal efecto la imagen original$f_{in}$ es umbralizada utilizando el método de Otsu de manera a tener una clara diferencia entre el área de la retina y el fondo de la imagen (FIGURA \ref{fig:borde_1}). Sobre esta imagen umbralizada se realiza la operación del gradiente morfológico utilizando un elemento estructurante con forma de disco de tamaño de 10 píxeles para obtener el borde circular (FIGURA \ref{fig:borde_2}). 
 
 \begin{figure}[H]
\centering
\subfigure[Imagen de retina.]{\includegraphics[width=50mm]{./Figures/cap4/vasos/vaso1.jpg}}
\subfigure[Imagen umbralizada.]{\includegraphics[width=50mm]{./Figures/cap3/borde/bordeUmb.png}}
\caption{Umbralización.} \label{fig:borde_1}
\end{figure}
 


 \begin{figure}[H]
\centering
\subfigure[Imagen umbralizada.]{\includegraphics[width=50mm]{./Figures/cap3/borde/bordeUmb.png}}
\subfigure[Gradiente morfológico.]{\includegraphics[width=50mm]{./Figures/cap3/borde/bordeRes.png}}
\caption{Gradiente morfológico.} \label{fig:borde_2}
\end{figure}

La secuencia de pasos para la detección y segmentación del borde circular puede verse en la FIGURA \ref{fig:bordeCircular}

 \begin{figure}[H]
	\centering
		\includegraphics[width=0.7	\textwidth]{./Figures/cap4/bordeCircular.png}
	\caption{Diagrama de detección del Borde circular.}
	\label{fig:bordeCircular}
\end{figure}

\subsubsection{Detección y Segmentación de Exudados Duros}
Los exudados duros son lesiones claras, por lo que es más fácil detectarlos utilizando la intensidad de la imagen (FIGURA \ref{fig:exu_1}).
%Como primer paso del algoritmo, se obtiene el canal de intensidad de la imagen original$f_{in}$, luego 
%Sobre esta imagen
Se aplica un cierre morfológico con un elemento estructurante en forma de disco de 6 píxeles, eliminando pequeñas zonas que pueden ser identificadas como exudados debido a su variación de intensidad (FIGURA \ref{fig:exu_2}).
\begin{figure}[H]
\centering
\subfigure[Imagen de retina.]{\includegraphics[width=50mm]{./Figures/cap3/exu/exu1.jpg}}
\subfigure[Intensidad de la Imagen.]{\includegraphics[width=50mm]{./Figures/cap3/exu/exu2.jpg}}
\caption{Intensidad de la imagen.} \label{fig:exu_1}
\end{figure}



\begin{figure}[H]
\centering
\subfigure[Intensidad de la Imagen.]{\includegraphics[width=50mm]{./Figures/cap3/exu/exu2.jpg}}
\subfigure[Cierre de imagen.]{\includegraphics[width=50mm]{./Figures/cap3/exu/exu3.jpg}}

\caption{Cierre de imagen.} \label{fig:exu_2}
\end{figure}

Seguidamente, se obtiene las componentes brillantes aplicando la transformada de Top-Hat utilizando un elemento estructurante en forma de disco con un tamaño de 6 píxeles (FIGURA \ref{fig:exu_3}).
La imagen resultante es umbralizada por el método de entropía máxima (FIGURA \ref{fig:exu_4}).
\begin{figure}[H]
\centering
\subfigure[Cierre de Imagen.]{\includegraphics[width=50mm]{./Figures/cap3/exu/exu3.jpg}}
\subfigure[Top-hat de la imagen.]{\includegraphics[width=50mm]{./Figures/cap3/exu/exu4.jpg}}

\caption{Top-hat de la imagen.} \label{fig:exu_3}
\end{figure}


\begin{figure}[H]
\centering
\subfigure[Top-hat de la imagen.]{\includegraphics[width=50mm]{./Figures/cap3/exu/exu4.jpg}}
\subfigure[Umbralización de la imagen.]{\includegraphics[width=50mm]{./Figures/cap3/exu/exu5.jpg}}

\caption{Umbralización de la imagen.} \label{fig:exu_4}
\end{figure}

De esta imagen se extraen las partes restantes de los vasos sanguíneos (FIGURA \ref{fig:exu_5}). Seguidamente se extrae el borde circular (FIGURA \ref{fig:exu_6}).

\begin{figure}[H]
\centering
\subfigure[Vasos sanguíneos.]{\includegraphics[width=50mm]{./Figures/cap3/exu/exuVena.jpg}}
\subfigure[Vasos sanguíneos extraídos.]{\includegraphics[width=50mm]{./Figures/cap3/exu/exuSinVena.jpg}}
\caption{Extracción de vasos sanguíneos.} \label{fig:exu_5}
\end{figure}


\begin{figure}[H]
\centering
\subfigure[Borde circular.]{\includegraphics[width=50mm]{./Figures/cap3/exu/exuBorde.jpg}}
\subfigure[Borde circular extraídos.]{\includegraphics[width=50mm]{./Figures/cap3/exu/exuSinBorde.jpg}}
\caption{Extracción de borde circular.} \label{fig:exu_6}
\end{figure}
Por último, se extrae el disco óptico, obteniendo así la imagen final $f_{ed}$ (FIGURA \ref{fig:exu_7}).
\begin{figure}[H]
\centering
\subfigure[Disco óptico.]{\includegraphics[width=50mm]{./Figures/cap3/exu/exuDisco.jpg}}
\subfigure[Disco óptico extraído.]{\includegraphics[width=50mm]{./Figures/cap3/exu/exuSinDisco.jpg}}
\caption{Extracción de disco óptico.} \label{fig:exu_7}
\end{figure}
 
La secuencia de pasos para la detección y segmentación de exudados duros puede verse en la FIGURA \ref{fig:diaExu}.
%y la secuencia de imágenes generadas son desplegadas en la FIGURA \ref{fig:secExu}.
\begin{figure}[H]
	\centering
		\includegraphics[width=0.7	\textwidth]{./Figures/cap4/exudaEspa2.png}
	\caption{Diagrama de bloques de detección y segmentación de exudados duros.}
	\label{fig:diaExu}
\end{figure}

%\begin{figure}[H]
%	\centering
%		\includegraphics[width=0.7	\textwidth]{./Figures/cap4/secExu2.PNG}
%	\caption{Secuencia de imágenes de detección y segmentación de Exudados duros (a) Imagen de retina, (b) canal de intensidad de la imagen, (c) Cierre de imagen, (d) Transformada Top-Hat de imagen, (e) Imagen umbralizada, (f) Imagen de vasos sanguíneos, (g) Imagen de borde circular, (h) Imagen de disco óptico y (i) Imagen de  exudados duros segmentada $f_{ed}$}
%	\label{fig:secExu}
%\end{figure}

\subsection{Detección y segmentación de microaneurismas} 

Para la detección y segmentación de microaneurismas se obtiene del canal verde de la imagen de retina $f_{in}$ debido a que los microaneurismas contienen características más contrastadas en este canal (FIGURA \ref{fig:sec_ma_0}). 
\begin{figure}[H]
\centering
\subfigure[Imagen de retina.]{\includegraphics[width=50mm]{./Figures/cap4/vasos/vaso1.jpg}}
\subfigure[Canal verde de la imagen.]{\includegraphics[width=50mm]{./Figures/cap4/micro0.jpg}}
\caption{Canal verde de la imagen.} \label{fig:sec_ma_0}
\end{figure}

Con el objetivo de reducir el ruido de la imagen, aplicamos un filtro estadístico, específicamente el filtro de la mediana. Ver FIGURA \ref{fig:sec_ma_1}. Una vez filtrada esta imagen es normalizada (FIGURA \ref{fig:sec_ma_2}).


\begin{figure}[H]
\centering
\subfigure[ Canal verde de la imagen.]{\includegraphics[width=50mm]{./Figures/cap4/micro0.jpg}}
\subfigure[Imagen generada por el filtro de la mediana.]{\includegraphics[width=50mm]{./Figures/cap4/micro1.jpg}}
\caption{Filtro de la mediana.} \label{fig:sec_ma_1}
\end{figure}
%\subsubsection{Normalización de la intensidad}

\begin{figure}[H]
\centering
\subfigure[Filtro de la mediana.]{\includegraphics[width=50mm]{./Figures/cap4/micro1.jpg}}
\subfigure[Normalización de la intensidad.]{\includegraphics[width=50mm]{./Figures/cap4/micro2.jpg}}
\caption{Imagen con intensidad ajustada.} \label{fig:sec_ma_2}
\end{figure}

%\subsubsection{CLAHE}
Para finalizar el proceso de mejora se estira el contraste utilizando el algoritmo de CLAHE (FIGURA \ref{fig:sec_ma_3}). Luego esta imagen mejorada es erosionada por un elemento estructurante en forma de disco con un tamaño de 5 píxeles (FIGURA \ref{fig:sec_ma_4}).
\begin{figure}[H]
\centering
\subfigure[ Imagen con intensidad ajustada.]{\includegraphics[width=50mm]{./Figures/cap4/micro2.jpg}}
\subfigure[ Imagen ecualizada.]{\includegraphics[width=50mm]{./Figures/cap4/micro3.jpg}}
\caption{Ecualización del histograma.} \label{fig:sec_ma_3}
\end{figure}
 %\subsubsection{Imagen erosionada}
 

\begin{figure}[H]
\centering
\subfigure[Imagen ecualizada.]{\includegraphics[width=50mm]{./Figures/cap4/micro3.jpg}}
\subfigure[Imagen erosionada.]{\includegraphics[width=50mm]{./Figures/cap4/micro4.jpg}}
\caption{Erosión de una imagen.} \label{fig:sec_ma_4}
\end{figure}

 %\subsubsection{Resta de imágenes}
Después de esto se obtiene el borde, haciendo la diferencia entre la imagen mejorada y la erosionada (FIGURA \ref{fig:sec_ma_5}).  %\subsubsection{Imagen umbralizada}
 Esta imagen es umbralizada en intensidad para generar una nueva imagen conteniendo los posibles microaneurismas (FIGURA \ref{fig:sec_ma_6}).
 \begin{figure}[H]
\centering
\subfigure[Imagen ecualizada.]{\includegraphics[width=50mm]{./Figures/cap4/micro4.jpg}}
\subfigure[Resta de imágenes.]{\includegraphics[width=50mm]{./Figures/cap4/micro66.jpg}}
\caption{Resta de imágenes.} \label{fig:sec_ma_5}
\end{figure}


 \begin{figure}[H]
\centering
\subfigure[Imagen a umbralizar.]{\includegraphics[width=50mm]{./Figures/cap4/micro66.jpg}}
\subfigure[Imagen umbralizada.]{\includegraphics[width=50mm]{./Figures/cap4/micro6.jpg}}
\caption{Umbralización.} \label{fig:sec_ma_6}
\end{figure}
 
 %\subsubsection{componentes conectados}
 De esta imagen se mantienen los componentes conectados en un rango de área entre 65 y 200 píxeles (FIGURA \ref{fig:sec_ma_7}).  Finalmente, se aplica un cierre con un disco para sobresaltar la forma circular de los componentes conectados (FIGURA \ref{fig:sec_ma_8}).
 
 \begin{figure}[H]
\centering
\subfigure[ Imagen umbralizada.]{\includegraphics[width=50mm]{./Figures/cap4/micro6.jpg}}
\subfigure[Imagen con componentes conectados eliminados.]{\includegraphics[width=50mm]{./Figures/cap4/micro9.jpg}}
\caption{Componentes conectados.} \label{fig:sec_ma_7}
\end{figure}
 %\subsubsection{Cierre de imagen}
 


 \begin{figure}[H]
\centering
\subfigure[Imagen binaria.]{\includegraphics[width=50mm]{./Figures/cap4/micro9.jpg}}
\subfigure[Cierre de imagen.]{\includegraphics[width=50mm]{./Figures/cap4/micro10.jpg}}
\caption{Cierre de imagen.} \label{fig:sec_ma_8}
\end{figure} 
 
 %\subsubsection{Imágenes con componentes de forma circular $f_{ma}$}
Se obtienen los componentes conectados de forma circular que son los microaneurismas detectados en  la imagen final $f_{ma}$ (FIGURA \ref{fig:sec_ma_9}).

 \begin{figure}[H]
\centering
\subfigure[Imagen erosionada.]{\includegraphics[width=50mm]{./Figures/cap4/micro10.jpg}}
\subfigure[Imagen final con MA.]{\includegraphics[width=50mm]{./Figures/cap4/micro12.jpg}}
\caption{Imágenes con componentes de forma circular $f_{ma}$} \label{fig:sec_ma_9}
\end{figure}

La secuencia de pasos para la detección y segmentación de microaneurismas puede verse en la FIGURA \ref{fig:diaMa}.
%y la secuencia de imágenes generadas son desplegadas en la FIGURA \ref{fig:secMa}.

\begin{figure}[H]
	\centering
		\includegraphics[width=0.7	\textwidth]{./Figures/cap4/dia_ma.png}
	\caption{Diagrama de bloques de detección y segmentación de microaneurismas.}
	\label{fig:diaMa}
\end{figure}


%\begin{figure}[H]
%	\centering
%		\includegraphics[width=0.7	\textwidth]{./Figures/cap4/sec_ma.png}
%	\caption{Secuencia de imágenes de detección y segmentación de Micro aneurismas (a) Imagen de Retina, (b) Canal verde de la imagen, (c) Imagen generada por el filtro de la mediana, (d) Imagen con intensidad ajustada, (e) Imagen ecualizada, (f) Imagen erosionada, (g) Resta de imágenes, (h) Imagen umbralizada, (i) Imagen sin componentes conectados y (j) Imágenes con componentes de forma circular $f_{ma}$.}
%	\label{fig:secMa}
%\end{figure}

\section{Módulo 2: Extracción de características}
Una vez finalizada la etapa de segmentación es necesario extraer las áreas de interés de las imágenes, para este trabajo el vector característica utilizado está determinado por las siguientes características:
\begin{itemize}
\item Área de vasos sanguíneos ($C_{vs}$).
\item Área de exudados duros ($C_{ed}$).
\item Área de microaneurismas ($C_{ma}$).
\end{itemize}

\nomenclature[49]{$f_{vs}$}{Imagen binaria final de vasos sanguíneos segmentados.}
\nomenclature[50]{$f_{ed}$}{Imagen binaria final de exudados duros segmentados.}
\nomenclature[51]{$f_{ma}$}{Imagen binaria final de microaneurismas segmentados.}

Las cuales se obtienen sumando los píxeles de las imágenes $f_{vs}$, $f_{ed}$ y $f_{ma}$.

%La sumatoria de las áreas de los componentes conectados $A$ se define como:

%\begin{equation}
%\label{eq:AreaObjetoSeg}
%A=\sum_{i=0}^{m} \sum_{j=0}^{n} f(i,j)
%\end{equation}

\begin{equation}
\label{eq:AreaObjetoSeg1}
C_{vs}=\sum_{i=0}^{i=M-1} \sum_{j=0}^{j=N-1} f_{vs}(i,j)
\end{equation}

\nomenclature[52]{$C_{vs}$}{Área de los vasos sanguíneos calculado a partir de la imagen binaria $f_{vs}$.}
\nomenclature[53]{$C_{ed}$}{Área de los exudados duros calculado a partir de la imagen binaria $f_{ed}$.}
\nomenclature[54]{$C_{ma}$}{Área de los microaneurismas calculado a partir de la imagen binaria  $f_{ma}$.}


\begin{equation}
\label{eq:AreaObjetoSeg2}
C_{ed}=\sum_{i=0}^{i=M-1} \sum_{j=0}^{j=N-1} f_{ed}(i,j)
\end{equation}

\begin{equation}
\label{eq:AreaObjetoSeg3}
C_{ma}=\sum_{i=0}^{i=M-1} \sum_{j=0}^{j=N-1} f_{ma}(i,j)
\end{equation}

En ese contexto, el vector de características $x$ está conformado por $C_{vs}$, $C_{ed}$ y $C_{ma}$, esto es $x=\{C_{vs},C_{ed},C_{ma}\}$.
%El área $A$ se calcula sumando todos los píxeles de la imagen e indica la suma de las áreas de los objetos segmentados. 

\section{Módulo 3: Clasificación}
Se realiza la clasificación de imágenes por el clasificador binario Support Vector Machine (SVM) \cite{cortes1995support}, el mismo ya alcanza un nivel de  exactitud favorable luego de tres o cuatro rondas de buena retroalimentación. Básicamente el clasificador SVM recibe un conjunto de características de entrenamiento $D=\{x,y\}$, donde $x$ es el conjunto de vectores de características e $y$ es el conjunto de etiquetas, cada etiqueta pertenece a una de dos categorías, en nuestro caso 0 para una retina sana y 1 en caso de ser una retina con retinopatía diabética, a partir del conjunto de entrenamiento se construye un modelo que posteriormente será usado para clasificar las imágenes de prueba.%Considerando los resultados obtenidos según el clasificador, en el capítulo siguiente se realiza un análisis de los mismos en base a las métricas de comparación. 


\section{Resumen}


En este capítulo se detalla la manera de obtener un diagnóstico a partir de una imagen de retina. Para lograr entender el funcionamiento de la metodología propuesta antes se debe entender como funciona cada uno de sus módulos. El primero de ellos es el módulo de detección y segmentación, en este módulo se describe los pasos realizados sobre la imagen de entrada y obtener a partir de esos pasos imágenes binarias segmentadas. En este módulo se comentan tres segmentaciones principales que son la de vasos sanguíneos, exudados duros y microaneurismas y dos segmentaciones secundarias que son  la de borde circular y disco óptico.
 En el segundo módulo se realiza la extracción de características de las imágenes segmentadas. El proceso se reduce a hallar el área en píxeles de las patologías y estructuras del ojo segmentadas.
 Como último paso se hace la clasificación, este módulo se encarga de agrupar las imágenes en sanas o con retinopatía diabética en base a las características extraídas. 
En el siguiente capítulo se presenta las métricas de evaluación utilizadas para evaluar el desempeño de la metodología propuesta, se expone el ambiente experimental y los resultados obtenidos del trabajo base del estado del arte y el de la metodología propuesta. Finalmente se comparan y discuten los resultados obtenidos.
\newpage{\ } 
\thispagestyle{empty} 

\chapter{Pruebas experimentales}
\lhead{Capítulo 5. \emph{Pruebas experimentales}} % This is for the header on each page - perhaps a shortened title
En este capítulo se menciona las métricas de evaluación utilizadas para medir el desempeño de la metodología propuesta. Por otro lado, se detallan los experimentos realizados y los  resultados tanto de la metodología propuesta y  del estado del arte, seguidamente se esboza la comparación y la posterior discusión de los mismos.


\section{Métricas de evaluación}

Como se había mencionado anteriormente, un diagnóstico consiste en determinar la presencia o ausencia de una enfermedad, para este caso particular es la  presencia o ausencia de la retinopatía diabética. Un diagnóstico positivo determina la presencia de retinopatía diabética y un diagnóstico negativo determina la ausencia de la misma.

Dado un clasificador y un diagnóstico, hay cuatro resultados posibles. Si el diagnóstico es positivo y se clasifica como positivo, se cuenta como un verdadero positivo (VP); si se clasifica como negativo, se cuenta como un falso negativo (FN). Si el diagnóstico es negativo y se clasifica como negativo, se cuenta como un verdadero negativo (VN); si se clasifica como positivo, se cuenta como un falso positivo (FP) \cite{fawcett2006introduction}. En la FIGURA \ref{fig:clasificacion} se puede ver la gráfica de una matriz en donde se ve los cuatro casos posibles del par diagnóstico y clasificación. 
%Las limitaciones de la precisión de diagnóstico como una medida del desempeño de decisión, requiere la introducción de los conceptos de sensibilidad y especificidad de una prueba diagnóstica \cite{metz1978basic,fawcett2006introduction}.

\begin{figure}
	\centering
		\includegraphics[width=0.65	\textwidth]{./Figures/cap5/matrizd.PNG}
	\caption{Esquema de clasificación.}

	\label{fig:clasificacion}
\end{figure}

Para evaluar los resultados obtenidos se hace uso de las siguientes métricas de evaluación: sensibilidad, especificidad y exactitud. A continuación las mismas se explican con más detalles.  
\subsection{Sensibilidad}
Es la probabilidad de clasificar correctamente a un paciente con retinopatía diabética, es decir, la probabilidad de que  un paciente con retinopatía diabética sea diagnosticado con resultado positivo. La sensibilidad es, por lo tanto, la capacidad para detectar pacientes con retinopatía diabética \cite{pita2003pruebas}.

\begin{equation}
\label{eq:sensibilidad}
\small	Sensibilidad=\frac{\text{VP}}{\text{VP} + \text{FN}}
\end{equation}

\subsection{Especificidad}
Es la probabilidad de clasificar correctamente a un paciente sin retinopatía diabética, es decir, la probabilidad de que para un sujeto sano se obtenga un resultado negativo. En otras palabras, se puede definir la especificidad como la capacidad para detectar pacientes sin retinopatía diabética \cite{pita2003pruebas}. 
\begin{equation}
\label{eq:especificidad}
\small	Especificidad=\frac{\text{VN}}{\text{VN} + \text{FP}}
\end{equation}

\subsection{Exactitud}
Es la probabilidad de clasificar correctamente a un paciente, dicho de otra manera, la probabilidad de que un paciente sano o enfermo obtenga un diagnóstico correcto. En otras palabras, se puede definir la exactitud como la capacidad para detectar pacientes con o sin  retinopatía diabética \cite{pita2003pruebas}. 
\begin{equation}
\label{eq:exactitud}
\small	Exactitud = \frac{VN + VP}{VN + FP + FN + VP}
\end{equation}

\section{Pruebas y resultados experimentales}
En esta sección  se profundiza el trabajo base del estado del arte, motivo por el cual fue seleccionado y sus resultados. Igualmente  se exponen los resultados obtenidos por nuestra metodología y se realiza una comparación con respecto a las métricas de evaluación.

\subsection{Base de imágenes MESSIDOR}
Para las pruebas se utiliza la base de imágenes pública MESSIDOR  \cite{messidor}. Estas imágenes poseen una resolución de 2240 x 1488 píxeles. Las imágenes seleccionadas para las pruebas incluyen imágenes borrosas y con poca iluminación con el fin de probar la robustez del detector.

\subsection{Metodología del trabajo base}
Selvathi et al. \cite{selvathi2012automated} proponen un sistema  de detección automática de retinopatía diabética a partir de imágenes de retina, cuyo trabajo se compara con nuestra metodología propuesta.
Las razones por las cuales este trabajo fue seleccionado, se basa en las siguientes similitudes:
\begin{itemize}
\item Utilizaron segmentaciones de vasos sanguíneos, exudados duros y microaneurismas.
\item Utilizaron  una base de imágenes pública MESSIDOR \cite{messidor} con imágenes de retina previamente diagnosticada por profesionales médicos.
\item Estructuraron su metodología en tres módulos, el primero de ellos el de detección y segmentación, luego el módulo de extracción de características y por último el módulo de clasificación. 
\item Hicieron uso del clasificador binario SVM.
\item Es el trabajo con mejores resultados para el diagnóstico automatizado de presencia o ausencia de retinopatía diabética para la base de imágenes de MESSIDOR  \cite{messidor}.
\end{itemize}

Además de estas semejanzas, en \cite{selvathi2012automated} los resultados reportados indican que el enfoque de Selvathi et al. es el más promisorio, por tal motivo, es el fundamento principal de la comparación con este trabajo.


%la propuesta de este trabajo  base a la etapa de investigación se concluye que ellos obtuvieron los mejores resultados en  el diagnóstico de la retinopatía diabética en relación a otros similares, lo que representa el fundamento principal de la comparación con este trabajo.
\subsubsection{Resultados del estado del arte}
Selvathi et al. \cite{selvathi2012automated} utilizaron 200 imágenes para sus experimentaciones de las cuales  50 fueron  imágenes de entrenamiento, 25 imágenes sin retinopatía diabética y 25 imágenes con retinopatía diabética. 
En la etapa de pruebas emplearon 150 imágenes, 75 imágenes con retinopatía diabética y 75 imágenes sin retinopatía diabética, diagnosticadas por el clasificador binario SVM.

De las 75 imágenes con retinopatía diabética, 4 no fueron clasificadas correctamente, dando una tasa de sensibilidad 94,67\%, mientras que 7 de 75 imágenes sin retinopatía diabética fueron mal diagnosticadas, teniendo así una tasa de especificidad del 90,67\%, para dar  un total de 139 imágenes correctamente diagnosticada de 150 imágenes de prueba logrando una tasa de exactitud del 92,67\%.

Los resultados  obtenidos por \cite{selvathi2012automated} se pueden apreciar en la TABLA \ref{tab:resultados1}. 

\begin{table}[!hbtp]
\begin{center}
\caption{Resultados de Clasificación de Selvathi 2012 \cite{selvathi2012automated}.}
\resizebox{15cm}{!} {
\begin{tabular}{|p{2.2cm}|p{1.9cm}|p{2.2cm}|p{2.4cm}|p{2cm}|}


\hline
Métricas &  Imágenes de entrenamiento & Imágenes de prueba  & Imágenes clasificadas correctamente  & Tasa \\ 
\hline
Sensibilidad & 25 & 75 & 71 & 94,67 \\
Especificidad & 25 & 75 & 68 & 90,67  \\
Exactitud  & 50 & 150  & 139  & 92.67 \\
%\hlineExactitud  & 50 & 150  & 139  & 93 \\
\hline
\end{tabular}
}

\label{tab:resultados1}
\end{center}
\end{table}
 
\subsection{Prueba experimental I}
Para los experimentos se hace uso de 200 imágenes de la base pública MESSIDOR. El software utilizado para la implementación de los módulos fue el \textit{toolbox} de procesamiento de imágenes de MATLAB versión 2013b. El primer paso a realizarse es la segmentación de los vasos sanguíneos, exudados duros y microaneurismas. Luego se realiza la extracción de características en base a las imágenes segmentadas. Por último, la clasificación y los resultados obtenidos  se exponen a continuación:

% 0.969333 0.952  0.9866667

\begin{table}[!hbtp]
\begin{center}
\caption{Resultados de Clasificación.}
\resizebox{15cm}{!} {
\begin{tabular}{|p{2.2cm}|p{1.9cm}|p{2.2cm}|p{2.4cm}|p{2cm}|}

\hline
Métricas &  Imágenes de entrenamiento & Imágenes de prueba  & Imágenes clasificadas correctamente  & Tasa \\ 
\hline
Sensibilidad & 25 & 75 & 71 & 94,67 \\
Especificidad & 25 & 75 & 74 & 98,67  \\
%\hline
Exactitud  & 50 & 150  & 145  & 96,67\\
\hline
\end{tabular}
}

\label{tab:resultados}
\end{center}
\end{table}


En la TABLA \ref{tab:resultados}, como se puede observar, se hace uso de 50 imágenes de entrenamiento que se seleccionaron aleatoriamente de un grupo de imágenes de entrenamiento, 25 imágenes sin retinopatía diabética y 25 imágenes con retinopatía diabética. Para probar el sistema se utiliza 150 imágenes de prueba, 75 imágenes con retinopatía diabética y 75 imágenes sin retinopatía diabética.

 Con respecto a los resultados obtenidos de las 75 imágenes con retinopatía diabética, 4 no fueron clasificadas correctamente, dando una tasa de sensibilidad 94,67\%, mientras que 74  de las 75 imágenes sin retinopatía diabética fueron diagnosticadas correctamente, es decir una tasa de especificidad de 98,67\%, para dar un total de 145 imágenes correctamente diagnosticadas de 150 imágenes de prueba logrando una tasa de exactitud del 96,67\%.


% Selvathi, D., N. B. Prakash, and Neethi Balagopal \cite{selvathi2012automated}.
%\begin{itemize}
%\item 200 imágenes: 100 normales, 100 con retinopatía diabética.
%\item Sensibilidad: 95\%.
%\item Especificidad: 91\%.
%\item Exactitud: 93\%.
%\end{itemize}
%\subsection{}


\subsection{Comparación y discusión de resultados}
% \cite{selvathi2012automated}
 En la TABLA \ref{tab:comparacion} se puede ver la comparación de los resultados y del estado del arte:
 \begin{table}[!hbtp]
 \caption{Comparación de clasificación.}
\begin{center}
\resizebox{15cm}{!} {

\begin{tabular}{|p{3.5cm}|p{2cm}|p{2cm}|p{2cm}|p{2cm}|p{2cm}| p{2cm}}
\hline
 &  Imágenes de entrenamiento &  Imágenes de Prueba & Sensibilidad & Especificidad & Exactitud   \\ 
 
\hline
Selvathi 2012 & 50 & 150& 94,67 & 90,67 & 92,67  \\
Método propuesto& 50 & 150 & 94,67 & 98,67 & 96,67   \\
\hline
\end{tabular}
}

\label{tab:comparacion}
\end{center}
\end{table}


Se puede apreciar que el método propuesto ha superado en especificidad e igualado en sensibilidad al propuesto por Selvathi et al. \cite{selvathi2012automated} obteniendo así mayor exactitud. 
  En el proceso de entrenamiento del clasificador, se emplea la misma cantidad de imágenes, de igual manera, se obtiene una mejora con respecto a las métricas: especificidad y exactitud.


El segundo experimento denota los valores de exactitud obtenidos en base a cierta cantidad de imágenes de entrenamiento.
\subsection{Prueba experimental II}
Para el segundo experimento, se usan 200 imágenes de la base de imágenes MESSIDOR, se toman 100 imágenes de entrenamiento y 100 imágenes de prueba.
El experimento consiste en tomar aleatoriamente $v$ imágenes de entrenamiento y hallar la exactitud obtenida al clasificar las 100 imágenes de prueba. Esto se realiza en 10 iteraciones. Luego, este procedimiento se repite para valores de $v=\{10,20,30,40,50,60,70,80\}$.

 \begin{figure}[H]
	\centering
		\includegraphics[width=0.65	\textwidth]{./Figures/cap5/boxplot.png}
	\caption{Diagrama de caja de los valores de exactitud obtenidos.}

	\label{fig:bp}
\end{figure}

En la FIGURA \ref{fig:bp} se puede ver que para menores cantidades de entrenamiento se obtuvieron resultados altos tanto como resultados bajos, es decir se tienen resultados inestables. A medida que fue aumentando la cantidad de imágenes de entrenamiento los resultados obtenidos se estabilizaron.

 \begin{figure}[H]
	\centering
		\includegraphics[width=0.65	\textwidth]{./Figures/cap5/exacTra.png}
	\caption{Porcentaje de exactitud por imágenes de entrenamiento.}

	\label{fig:exacNtrta}
\end{figure}
 


En la FIGURA \ref{fig:exacNtrta} se puede ver la curva generada por los valores de exactitud obtenidos en promedio de las 10 iteraciones para la clasificación en función de la imágenes de entrenamiento.
Se puede notar en la gráfica que por más de que aumente la cantidad de imágenes de entrenamiento la exactitud en la clasificación ya no se mejora.



%\section{Discusión de resultados}
   
 % Entre las características utilizadas en la detección de retinopatía diabética, la de mayor relevancia fueron los microaneurismas  debido a que su presencia o ausencia en la imagen de retina fue determinante en el diagnóstico realizado \cite{veritticlose} por el clasificador SVM.
  
\subsection{Dificultades encontradas} 
 Debido a la mala calidad de algunas imágenes se hace necesario evaluar automáticamente la calidad de las imágenes de retina para aumentar la exactitud en la detección de estructuras y patologías de la retina para un sistema de detección basado en técnicas de visión por  computadora. Una imagen se considera de mala calidad cuando es difícil o imposible hacer un  juicio clínico confiable con respecto a la presencia o ausencia de retinopatía diabética en la imagen \cite{niemeijer2006image}.  

 En la FIGURA \ref{fig:badQuality} se muestra una de las imágenes de mala calidad utilizada como imagen de prueba, en ella se puede notar poca iluminación debido al pequeño tamaño de la pupila y pérdida de contraste por el movimiento del ojo.
 
 \begin{figure}[H]
	\centering
		\includegraphics[width=0.65	\textwidth]{./Figures/cap5/badQ.png}
	\caption{Imágen digital de mala calidad.}

	\label{fig:badQuality}
\end{figure}
 



 
%Durante el proceso de desarrollo se comprobó que existe una relación  de costo entre la sensibilidad  y la especificidad del método, las mismas son inversamente proporcionales, lo que significa que a medida que aumenta la sensibilidad, la especificidad disminuye y viceversa. \cite{parikh2008understanding}.
Teniendo en cuenta los resultados favorables en base a los objetivos propuestos, en el siguiente y último capítulo se exponen las conclusiones de este trabajo  y finalmente trabajos futuros que puedan dar continuidad a este trabajo final de grado.
\newpage{\ } 
\thispagestyle{empty} 

\chapter{Conclusiones y trabajos futuros}
\lhead{Capítulo 6. \emph{Conclusiones y trabajos futuros}}
%\lhead{Capítulo 6. \emph{Conclusiones y trabajos y futuros}} % This is for the header on each page - perhaps a shortened title
En base a los resultados obtenidos en la estimaci\'on de perdida de carbono en nuestra \'area de estudio, este cap\'itulo nos presenta las conclusiones y recomendaciones para investigaciones futuras derivadas del trabajo:
\section{Conclusiones}

\begin{itemize}
\item Una vez evaluado las diferentes zonas de nuestro caso de estudio, podemos darnos cuenta que la metodolog\'ia propuesta posee una mejor respuesta, respecto a la calidad, en \'areas rurales. Esto es debido a que el Coeficiente kappa o los indices de acuerdo var\'ian entre 0.57-0.67 y su precisi\'on global sobrepasan el umbral optimo de 85\%, para cada coeficiente de tolerancia $ n $. Por lo que se considera satisfactorio la metodolog\'ia propuesta, ya que la perdida de carbono es un fen\'omeno frecuente en \'areas con vegetaci\'on predominante.

\item Para zonas donde la vegetaci\'on no predomina, esta metodolog\'ia podr\'ia no resultar suficientemente conveniente. Las pruebas experimentales hechas en zonas urbanas, la precisi\'on global y el coeficiente kappa no son \'optimos por el cual se llega a esa interpretaci\'on.

\item En \'areas cercanas a r\'ios o sujetas a inundaci\'on, se observaron resultados aceptables para  estudios con tolerancias medias y altas en la detecci\'on de perdida forestal. Por lo que el monitoreo en estos tipos de zonas con la metodolog\'ia propuesta podr\'ia ser aun de gran utilidad, ya que la presencia de agua en la vegetaci\'on modifica la respuesta espectral, dificultando su clasificaci\'on como cobertura vegetal.

\item Mediante los an\'alisis estad\'isticos empleados tanto para la determinaci\'on de umbrales vegetaci\'on/no vegetaci\'on como en el hallazgo de ecuaciones de transformaci\'on a carbono, nos indica que empleando extracciones de indices vegetales y variables estad\'isticas es posible generar metodolog\'ias no complejas destinadas al monitorio ambiental. Esta sencillez nos libera de necesarias supervisiones y entrenamientos normalmente empleadas en teledetecci\'on.

\item El mapa global de carbono \cite{saatchi2011benchmark} constituyo un factor importante para la automatizaci\'on, al permitir determinar una ecuaci\'on que transforme el indice vegetal a carbono. De no existir, hubiese sido necesario aplicar previos muestreos forestales en el terreno.

\item La correcci\'on geom\'etrica implica procesos que engloba visitas al terreno para levantamientos de puntos de control requeridas en las interpolaciones. Gracias a la utilizaci\'on de im\'agenes Landsat L1T prove\'idas por la USGS, no fue necesario sumar ese costo a la metodolog\'ia, automatizando-la por no haber necesidad de realizar dicho procedimiento.

\end{itemize}

La idea al elegir como caso de estudio parte del chaco paraguayo, es la de actuar de impulsora en la generaci\'on de herramientas para el monitoreo ambiental, donde con el empleo de procesamientos digital de im\'agenes satelitales que conlleven t\'ecnicas computacionalmente sencillas y autom\'aticas podamos identificar alertas referentes a perdida en el contenido de carbono forestal. De manera que una vez detectado, a trav\'es de las estimaciones, se puedan generar pol\'iticas de acci\'on o prevenci\'on contra los da\~{n}os posibles al ambiente. El chaco paraguayo es una regi\'on muy afectada actualmente por la degradaci\'on y de-forestaci\'on en los bosques, donde la falta de recursos y el costo  elevado en el monitoreo dificulta las intervenciones a tiempo, constituyendo un caso ideal e impulsora para la aplicaci\'on de metodolog\'ias como la propuestas en esta investigaci\'on.



\section{Trabajos futuros}
Se pretende que la metodolog\'ia propuesta siga mejorando en t\'erminos de pre-procesamiento de las im\'agenes satelitales, ante factores que influyan en el momento de captura de los datos hechas por sensores remotos como tambi\'en en t\'ecnicas que permita mejora la detecci\'on de cambio forestal, por lo que mencionamos como trabajos futuros:
\begin{itemize}
\item Proponer t\'ecnicas que permitan detectar y eliminar nubosidad en las imagen satelitales.
\item Mejorar la precisi\'on global y el coeficiente kappa para zonas urbanas.
\item Dise\~{n}ar mejores t\'ecnicas que clasifique cobertura vegetal mediante la extracci\'on de indices en todas las bandas.
\item Adaptar la metodolog\'ia, de manera a que permita recibir im\'agenes satelitales con diferentes resoluciones radiom\'etricas.

\end{itemize}

\newpage{\ } 
\thispagestyle{empty} 

%----------------------------------------------------------------------------------------
%	BIBLIOGRAPHY
%----------------------------------------------------------------------------------------

\lhead{\emph{Bibliografía}} % Change the page header to say "Bibliography"
%\bibliographystyle{apacite}
%\bibliographystyle{unsrtnat} % Use the "unsrtnat" BibTeX style for formatting the Bibliography
\bibliographystyle{alpha} % Es el estilo BiBTeX estandard más sencillo. Las entradas son ordenadas alfabéticamente con etiquetas numéricas.
%\bibliographystyle{unsrt} % Es muy parecido al estilo plain, pero en este estilo las referencias no se ordenan alfabéticamente sino que se citan en orden de aparición. Se utilizan etiquetas numéricas para referenciarlas.
%\bibliographystyle{apacite}
\bibliography{Bibliography} % The references (bibliography) information are stored in the file named "Bibliography.bib"
%\input{./Chapters/Anexo}




\end{document}  
